\chapter{Representing Information using Ontologies}
\label{chapter:vocabularies}

Intro: this chapter X Y Z

* presents the created ontologies for achieving aims listed in RO2

\section{Methodology for Ontology Engineering}\label{sec:voc:methodology}

* process - Ontology 101, NeOn, compare with PrOnto, list steps

* use of competency questions identified in Ch 4

* aim to fill gaps in the state of the art

* Evaluation through creation of use-cases based on real-world situations

* Evaluation through peer-reviewed publishing

* Protege to create ontology, test for inconsistencies using reasoner

* Ontology quality - follow community guidelines, OOPS!

* Documentation - follow best practices, WIDOCO guidelines, use WIDOCO to publish HTML and other serialisations on the web

* Use persistent identifiers - purl.org at first, then switch to w3id

* License: open license CC-by

* Open access: upload to public repositories - Zenodo, institution website - OpenScience, repositories showing work and code - Github, OpenGogs (institution)

\section{GDPRtEXT}\label{sec:voc:GDPRtEXT}

With the information identified using the compliance questions, assumptions, and constraints in \autoref{sec:info:compliance-questions}, competency questions were established to identify necessary concepts and relationships to create ontological representations using established procedures \cite{noy,neon}.
The competency questions presented in the following sections areare based on the three objectives as outlined within $RO3$ in \autoref{sec:intro:RO}.  
\begin{enumerate}
    \item concepts and text of the GDPR: representation of concepts within GDPR and associated with its compliance, representation of links between information and specific clauses of GDPR
    \item activities associated with personal data: representation of activities associated with processing of personal data in both ex-ante and ex-post phases
    \item information associated with consent: representation of information necessary to evaluate compliance of consent in its various states (given or otherwise)
\end{enumerate}

\subsection{Competency Questions for GDPR concepts and representation}
These competency questions are concerned with the representation of concepts within GDPR and its compliance, and association of information with specific clauses of the GDPR.

\subsubsection{Structure of GDPR text}
\begin{enumerate}[label={\textit{CQ.\theenumi}}]
    \item How many Recitals are there within GDPR?
    \item How many Chapters are there within GDPR?
    \item How many Sections are there within GDPR?
    \item How many Articles are there within GDPR?
    \item How many Paragraphs are there within GDPR?
    \item How many Sub-paragraphs are there within GDPR?
    \item How many References or Citations are there within GDPR?
    \item Article 4 belongs to which Chapter?
    \item Which clause contains the definition of 'personal data'?
    \item What is the structural hierarchy of the document?
    \item Where is the principle of `Accountability' defined?
    \item Which articles, paragraphs, and sub-paragraphs are relevant to the validity of given consent?
    \item How to associate information regarding given consent to relevant clauses in the GDPR?
    \item How to associate information regarding compliance to a specific article of the GDPR?
\end{enumerate}

\subsubsection{Concepts associated with GDPR}
\begin{enumerate}[label={\textit{CQ.\theenumi}},resume]
    \item What type of data does the GDPR define?
    \item What types of consent does the GDPR define?
    \item What are the different entities referred to within GDPR?
    \item Which activities are associated with processing of personal data?
    \item Which activities are associated with consent?
    \item What are the conditions or criteria associated which affect sensitivity of processing?
    \item What actions are relevant to a data breach?
    \item Which actions are relevant regarding compliance?
    \item What are the principles defined in GDPR?
    \item What are the rights provided by the GDPR?
    \item Which criteria does the GDPR mention for right to data portability?
    \item Which criteria does the GDPR mention for right to be informed?
    \item What are the obligations mentioned within GDPR?
    \item What are the obligations of the Controller?
    \item What are the obligations of the Processor?
    \item What are the obligations of a DPO?
    \item What are the lawful basis for processing of personal data specified in the GDPR?
    \item What are the conditions for valid consent under GDPR?
    \item Which obligations are mentioned in relation to data collection?
    \item Which obligations are mentioned in relation to obtaining consent?
    \item Which obligations are mentioned in relation to retaining personal data?
    \item Which obligations are mentioned in relation to security of personal data?
    \item What concepts are defined regarding seals and certifications?
\end{enumerate}

\subsection{Competency Questions for processing activities involving personal data}
\begin{enumerate}[label={\textit{CQ.\theenumi}},resume]
    \item x.
\end{enumerate}

\subsection{Competency Questions for consent}
\begin{enumerate}[label={\textit{CQ.\theenumi}},resume]
    \item x.
\end{enumerate}

\section{GDPRov}\label{sec:voc:GDPRov}

\section{GConsent}\label{sec:voc:GConsent}

\section{Data Privacy Vocabulary (DPV)}\label{sec:voc:DPV}