\chapter{Representing Information using Ontologies}
\label{chapter:vocabularies}

Intro: this chapter X Y Z

* presents the created ontologies for achieving aims listed in RO2

* \autoref{sec:voc:methodology} - Methodology
* \autoref{sec:voc:GDPRtEXT} - GDPRtEXT
* \autoref{sec:voc:GDPRov} - GDPRov
* \autoref{sec:voc:GConsent} - GConsent
* \autoref{sec:voc:DPV} - DPV

\section{Methodology for Ontology Engineering}\label{sec:voc:methodology}
This section expands on the ontology engineering methodology described in \autoref{sec:intro:ontology-engineering} regarding construction of ontologies described in this chapter.

\subsubsection*{Utilisation of Existing Ontology Engineering Methodologies}
The creation of ontologies followed guidelines and methodologies deemed `best practice' by the semantic web community. In this, `Ontology development 101: A guide to creating your first ontology' by Noy and McGuiness \cite{} - was utilised as the seminal guide for ontology engineering. It provides a series of steps for the creation of ontologies, which includes attention to avoid bad design decisions and pitfalls. It also advocates the use of competency questions to determine the scope of an ontology and evaluate it after creation. For this, the compliance questions presented in \autoref{} were used as competency questions. The guide also mentions use of Protégé\footnote{\url{}} \cite{} - a popular tool for ontology development - which also provides a semantic reasoner to detect inconsistencies in the ontology.

Another commonly adapted resource that was used is the NeOn methodology \cite{} which provides a flexible workflow through the use of scenarios which can be adapted for ontology development. The scenarios include from implementing using a specification, reusing and re-engineering existing ontological and non-ontological resources, and utilisation of ontological design patterns.

UPON Lite \cite{} is a lightweight methodology for rapid ontology engineering that was used in combination with NeOn for iteratively developing the presented ontologies. It consists of six steps from identification of domain terminology, construction of domain glossary, creating a taxonomy, predication as properties, meronymy for complex components, and conceptualisation into an ontology.

Compared with methodologies used in comparative work such as SPECIAL \autoref{} and MIREL \autoref{} within the SotA, the ontology development methodology used in this research lacked resources and access to legal experts which could be utilised to shape the legal interpretation of the developed work. To compensate for this, the developed ontologies have been sufficiently documented to indicate their aims, motivation, methodology, resources used to shape conceptualisations and rationalisations, and published in an open and accessible manner.

\subsubsection*{Summarisation of Methodology}
From above, the methodology used for ontology engineering and development can be summarised through the following steps:
\begin{enumerate}
    \item \textbf{Identification of aims, objectives, scope:} The first step was to identify the aim and objectives of information to be represented, followed by deciding on the scope regarding relation to GDPR compliance. For the ontologies presented in this chapter, the aims and objectives are listed in \autoref{}. % introduction
    \item \textbf{Identify and analyse relevant information:} Using the scope, relevant information was gathered from various sources - including authoritative, community, and publications - and analysed to identify terms of importance and requirements regarding GDPR compliance. The information is presented partially as background of the GDPR in \autoref{} and analysed with regards to compliance in \autoref{}.
    \item \textbf{Create use-cases and competency questions:} From the analysed information, different use-cases were identified to better understand the application of information in compliance scenarios and the requirements of different stakeholders in this process. This was done using the information interoperability model presented in \autoref{}. The analysed information was used to create compliance questions, as presented in \autoref{}, which identify relevant information for evaluation of compliance. These compliance questions were utilised as competency questions in the development and evaluation of ontologies.
    \item \textbf{Identify concepts and relationships:} Relevant concepts and relationships were identified to express information required to answer compliance questions in identified use-cases. This was an iterative and cyclic process where identified concepts and relationships were re-purposed to better suit some design pattern or compliance requirements.
    \item \textbf{Create Ontology:} The identified concepts and relationships were formalised as an ontology in OWL2 using the Protégé ontology development environment. In this process, a semantic reasoner - such as Pellet\footnote{\url{}} and HermiT\footnote{\url{}} - was used to identify inconsistencies in the ontology. Minor inconsistencies were fixed by changing the appropriate relationships between concepts, while major inconsistencies required evaluation of information identified in step 4. Development of the ontology utilised best practices advocated by the semantic web community in terms of ontology metadata \cite{}, documentation \cite{}, design patterns \cite{}, publication \cite{}, and dissemination \cite{}.
    \item \textbf{Evaluate:} The ontology was evaluated for sufficiency using competency questions, and by publishing as peer-reviewed publications. It was also evaluated for quality using guidelines and tools provided by the community.
    \item \textbf{Iteratively develop ontology using steps 2 to 6:} Following an iteration of an ontology and its evaluation, changes were integrated by following steps 2 to 6 in an development cycle. New concepts and relationships as well as changes to existing ones were integrated through this method. Previous versions of the ontology were documented for provenance where relevant and possible.
\end{enumerate}

\subsubsection*{Ontology Quality Considerations}
The quality of an ontology refers to the quality of its design of concepts and relationships. While following a suitable ontology engineering methodology provides a structured ontology, it still needs to be inspected for quality in terms of ontology as well as for intended use-cases and scenarios. For this, existing publications \cite{jeremy quality paper, vredicic thesis} list various methods of ontology quality detection, evaluation, and suggest solutions to fix identified problems.

OOPS!\footnote{\url{}} \cite{} is an useful tool for ontology evaluation which detects common pitfalls in the design of concepts and relationships and provides a documented output which can be persisted for provenance of ontology development. Each pitfall detected by OOPS! is categorised along  structural, functional, and usability-profiling dimensions. The tool also provides an indicative measure of importance regarding the pitfall in terms of critical, important, and minor levels.

OOPS! was used for detecting catalogued common pitfalls in the preiodic evaluation of developed ontologies. Identified pitfalls were corrected by changing the underlying relationships and iteratively developing the ontology to remove them. Other pitfalls were inspected manually from published sources \cite{}. 

\subsubsection*{Ontology Documentation}
Ontology documentation was created by using the WIDOCO\footnote{\url{}} \cite{} tool which uses ontology metadata to create HTML documents listing its classes and properties. Ontology metadata consists of information regarding the ontology as well as its concepts and properties integrated into the serialisation as annotations. WIDOCO provides a document of suggested metadata indicating best practices for ontology documentation, which the developed ontologies utilised. It builds upon the LODE\footnote{\url{}} tool and library\footnote{\url{}} which is also a popular ontology documentation service.

The output of the WIDOCO tool is a HTML document along with various serialisations of the ontology for content negotiation which were published as an online resource. Additional information was manually added to the HTML documentation regarding aims and methodologies used in the development of ontologies, as well as examples of use-cases and diagrams. WIDOCO integrates OOPS! to detect pitfalls in the ontology and documents the output. It also provides an interactive visualisation of the ontology using WebVOWL\footnote{\url{}}.

\subsubsection*{Dissemination}
The ontology was published on the internet using a stable IRI through persistent identifiers on servers hosted by ADAPT Research Centre and School of Computer Science \& Statistics within Trinity College Dublin. Initially, these were provided through the purl\footnote{\url{}} service, which later had issues regarding maintenance and frequent problems with URL resolution. Later the ontologies utilised the w3id\footnote{\url{}} persistent identifiers which are the current community recommendation and see active maintenance and development. The ontologies published in this manner followed the best practices and principles related to use of Linked Open Data\footnote{\url{}}, Linked Open Vocabularies\footnote{\url{}}, and FAIR\footnote{\url{}}.

Each ontology was added to the Linked Open Vocabularies community listing which catalogues vocabularies in the semantic web community. Each ontology was published in Zenodo which is a public open repositories and through which each iteration of the ontology was provided a DOI. The ontoloy development and resources were also added to public hosting services such as Github and an instance of OpenGogs hosted on institution servers. Each ontology was provided under an open and permissive license (CC-by-4.0\footnote{\url{}}) to promote its use and adoption.

\subsubsection*{Evaluation}
Evaluation was carried out by analysing whether the information expressed using the ontologies was sufficient to answer the collected competency questions. This was carried out in an interactive manner where the ontology was first developed and evaluated, and then the results of evaluation were used as feedback to further develop the ontology. Following the evaluation, changes to the ontology were made based on missing concepts and relationships, or incorrect assumptions expressed in existing ones. 

The ontology was also evaluated against common pitfalls as expressed in the section about ontology quality. Documentation and publishing standards were evaluated by assessing whether the ontologies met existing criteria advocated by the community (such as 5-star principle for linked data \cite{} and the FAIR principles \cite{}). Finally, each ontology was published and presented in a peer-reviewed venue and publication.

\section{GDPRtEXT - Linked Open Dataset of GDPR text \& Glossary of Concepts}\label{sec:voc:GDPRtEXT}

This section describes the GDPRtEXT vocabulary and dataset, which provides a linked open data version of the text of the GDPR and a SKOS glossary of concepts associated with its compliance. The section describes the motivation and creation of this work, its publication and dissemination, and a comparison with relevant approaches in the state of the art. GDPRtEXT is available online\footnote{\url{https://w3id.org/GDPRtEXT/}} with its documentation and code repository\footnote{\url{https://github.com/coolharsh55/GDPRtEXT/}}.

\subsection{Motivation}
GDPR as a legislation consists of text which is structured into 173 Recitals, 99 Articles (further grouped into Chapters and Sections), and 21 Citations. Each Article may have one or more Paragraphs, which itself may have one or more Sub-Paragraphs. As is the norm for legislations, each individual clause - whether an article, paragraph, or sub-paragraph - is identified with an alphanumeric number if provided. These are commonly referenced in textual notation by mentioning these identifiers present in the text of the legislation, such as for Article 8 Paragraph 2 Sub-Paragraph c the following textual notations can be used: A8(2-c), A(8-2c), A8-2(c), Art.8 2(c), Art-8-2-c. As there is no standard or accepted commonality in the approach, and because such notations are intended towards human readability and interpretation - there is no defined set of notations. This presents difficulty when representing such information in machine-readable formats.

The EU Publications Office currently publishes legislation metadata at the document level - which provides information about GDPR as a legislation using the ELI ontology and standard \cite{} - but do not specify granular information about its contents. While the EU Publications Office has indicated its intention to provide such granular metadata in the future, currently the state of the art contains primarily two approaches as presented and analysed in \autoref{}. While these provide granular information representation of GDPR clauses, the information model used is not compatible with the ELI model used by EU Publications Office.

With this as the motivation, and the research question establish in \autoref{}, this section presents the GDPRtEXT vocabulary and dataset which fulfils research objective $RO3(a)$ regarding creation of OWL2 ontology for expressing information about concepts and text of the GDPR. It also fulfils the aim/objective of linking information with individual clauses of the GDPR as well as with terms and concepts related to GDPR compliance.

\subsection{Ontology Engineering and Creation of Resource}\label{sec:voc:gdprtext-engineering}
Following the methodology described in \autoref{sec:voc:methodology}, the development of competency questions was achieved through understanding and analysis of how legal articles are referenced in text in relation to compliance. The competency questions concern representation of concepts within GDPR and its compliance, and association of information with specific clauses of the GDPR. They are categorised as pertaining to structure of GDPR text and regarding concepts associated with GDPR. The collected competency questions are outlined below:

\subsubsection{Structure of GDPR text}
\begin{enumerate}[label={\textit{CQ.\theenumi}}]
    \item How many Recitals are there within GDPR?
    \item How many Chapters are there within GDPR?
    \item How many Sections are there within GDPR?
    \item How many Articles are there within GDPR?
    \item How many Paragraphs are there within GDPR?
    \item How many Sub-paragraphs are there within GDPR?
    \item How many References or Citations are there within GDPR?
    \item Article 4 belongs to which Chapter?
    \item Which clause contains the definition of 'personal data'?
    \item What is the structural hierarchy of the document?
    \item Where is the principle of `Accountability' defined?
    \item Which articles, paragraphs, and sub-paragraphs are relevant to the validity of given consent?
    \item How to associate information regarding given consent to relevant clauses in the GDPR?
    \item How to associate information regarding compliance to a specific article of the GDPR?
\end{enumerate}

Based on these, the following requirements were identified for the ontology with regards to structure:
\begin{itemize}
    \item Structure of text must be specified with granularity and a hierarchy of Document, Chapter, Section, Article, Paragraph, Sub-Paragraph along with Recitals and Citations.
    \item Relations between clauses must be specified e.g. Paragraph belongs to an Article.
    \item Relations must be transitive e.g. Paragraph belonging to an Article must also belong to the Chapter the Article is in.
    \item Each individual clause must have a unique IRI to enable linking of information to it.
\end{itemize}

\subsubsection{Concepts associated with GDPR}
\begin{enumerate}[label={\textit{CQ.\theenumi}},resume]
    \item What type of data does the GDPR define?
    \item What types of consent does the GDPR define?
    \item What are the different entities referred to within GDPR?
    \item Which activities are associated with processing of personal data?
    \item Which activities are associated with consent?
    \item What are the conditions or criteria associated which affect sensitivity of processing?
    \item What actions are relevant to a data breach?
    \item Which actions are relevant regarding compliance?
    \item What are the principles defined in GDPR?
    \item What are the rights provided by the GDPR?
    \item Which criteria does the GDPR mention for right to data portability?
    \item Which criteria does the GDPR mention for right to be informed?
    \item What are the obligations mentioned within GDPR?
    \item What are the obligations of the Controller?
    \item What are the obligations of the Processor?
    \item What are the obligations of a DPO?
    \item What are the lawful basis for processing of personal data specified in the GDPR?
    \item What are the conditions for valid consent under GDPR?
    \item Which obligations are mentioned in relation to data collection?
    \item Which obligations are mentioned in relation to obtaining consent?
    \item Which obligations are mentioned in relation to retaining personal data?
    \item Which obligations are mentioned in relation to security of personal data?
    \item What concepts are defined regarding seals and certifications?
\end{enumerate}

Based on these, the following requirements were identified for the ontology with regards to structure:
\begin{itemize}
    \item The ontology should express concepts in a hierarchy of relation as associated with compliance. This hierarchy is based on which additional concepts are relevant to the given concept. For example, all principles are referred to when referring to the concept of 'Principle', and - activities and actions associated with compliance are referred to when using the concept of 'Compliance'.
    \item The ontology should reference concepts with their definitions within the clauses of the GDPR.
    \item The ontology should indicate concepts of relevance for a given concept.
    \item The ontology provide concepts regarding:
    \begin{itemize}
        \item types of data
        \item types of consent
        \item types of entities
        \item types of activities associated with - consent, data, processing, data breaches
        \item actions associated with compliance
        \item principles defined in the GDPR
        \item rights provided by the GDPR
        \item obligations mentioned in the GDPR
        \item conditions required for valid consent
        \item conditions associated with seals and certifications
    \end{itemize}
\end{itemize}

\subsubsection{Extending ELI}
Following this, the suitability of extending the existing ELI ontology for representing hierarchy of clauses in GDPR was evaluated and found to be feasible. Existing resources associated with the information model used by ELI - namely the FORMEX\footnote{\url{https://op.europa.eu/en/web/eu-vocabularies/formex//}} and Common Data Model\footnote{\url{https://op.europa.eu/en/web/eu-vocabularies/model/-/resource/dataset/cdm}} - were used to formulate an extension mechanism that ensured compatibility.
The extension intended to be language agnostic and therefore did not follow the language specifications provided by FRBR model\footnote{\url{https://www.ifla.org/publications/functional-requirements-for-bibliographic-records}}, though such functionality can be integrated in the future if needed. For the glossary of concepts, SKOS\footnote{\url{}} was selected based on its status as a standard and accepted usage by the community towards specification of thesauri.

\subsubsection{Creation of datasets}
Three outputs were decided based on the requirements - an OWL2 ontology for representing structure of GDPR text, a dataset of GDPR text using this ontology, and a SKOS glossary of concepts associated with compliance. The OWL2 ontology and SKOS glossary were combined within a single deliverable - namely the GDPRtEXT ontology, and the representation of GDPR text was provided as a RDF dataset with metadata defined using the DCAT\footnote{\url{https://www.w3.org/TR/vocab-dcat/}} and VoID\footnote{\url{https://www.w3.org/TR/void/}} standards.

Extracting the text of the GDPR in the form of individual clauses proved to be a challenge due to the way the official legislation is structured. The official publication provides the legislation as HTML, PDF, and XML. The task of extracting individual clauses and annotating their structure (e.g. chapter, article, paragraph) was automated using a python script.

Three datasets were produced and published through this process. The first provides description of canonical versions of the official legislation i.e. published by the EU Publications Office, and describes the GDPR legislation in its HTML, PDF, and XML formats. The second provides a copy of the GDPR text hosted on institution servers and provides identifiers for individual clauses in the GDPR in HTML, JSON, and plain-text formats. The third provides RDF serialisations of the text of GDPR using the GDPRtEXT ontology in RDF/XML, N3, Turtle, JSON-LD formats.

The glossary published using SKOS utilised the IRIs of individual clauses of the GDPR to indicate source, definitions, and related concepts. Definitions were declared using the \textit{rdfs:isDefinedBy} property, and a new property called \textit{involves} was created to indicate associations between concepts. 

\subsubsection{Publication \& Dissemination}
The ontology and dataset were provided through a SPARQL end-point on a triple-store hosted on the institution servers. Pubby\footnote{\url{http://wifo5-03.informatik.uni-mannheim.de/pubby/}} was used to provide a front-end for browsing the legislation. The dataset was provided under the CC-by-4.0 license to provide resources in an open and reusable manner. It was also published in the Irish Open Data Portal\footnote{\url{https://data.gov.ie/dataset/gdprtext}} which rated it as a 5-star linked open dataset. The documentation for the ontology and dataset was generated using WIDOCO, and published on institution servers.

The GDPRtEXT ontology and datasets (in various serialisations) were hosted on institution servers and provided through a public a SPARQL endpoint\footnote{\url{}} hosted using the OpenLink Virtuoso triple-store.
Along with a HTML version of GDPR, the RDF version of text was provided for human consumption using the Pubby\footnote{\url{}} framework, as shown in Fig.\ref{fig:vocab:gdprtext-pubby}
\begin{figure}[htbp]
    \centering
    \includegraphics[width=\linewidth]{img/gdprtext-pubby}
    \caption{Article 12(3) in GDPRtEXT as RDF displayed using Pubby \cite{}}
    \label{fig:vocab:gdprtext-pubby}
\end{figure}

\subsection{Resource Description \& Application}
A visual summary of the concepts within GDPRtEXT is presented in Figures \ref{fig:vocab:gdprtext-summary-a} and \ref{fig:vocab:gdprtext-summary-b}.

\begin{figure}[htbp]
    \centering
    \includegraphics[width=0.75\linewidth]{img/gdprtext-summary-a}
    \caption{Visual overview of concepts in GDPRtEXT - part (a) \cite{}}
    \label{fig:vocab:gdprtext-summary-a}
\end{figure}
\begin{figure}[htbp]
    \centering
    \includegraphics[width=0.75\linewidth]{img/gdprtext-summary-b}
    \caption{Visual overview of concepts in GDPRtEXT - part (b) \cite{}}
    \label{fig:vocab:gdprtext-summary-b}
\end{figure}

\subsubsection{Concepts for description structure of text}
GDPRtEXT extends the European Legislation Identifier (ELI) \cite{} ontology published by the European Publications Office with granular concepts to represent each individual clause within the GDPR. 
ELI provides the class \textit{LegalResource} to indicate a legislative document and its sub-class \textit{LegalSubResource} to indicate a component or part of that resource. GDPRtEXT extends the \textit{LegalSubResource} concept with classes Chapter, Section, Article, Point (indicates Paragraph), SubPoint (indicates Sub-Paragraph), Recital, and Citation using the sub-class mechanism.

The ELI provides properties \textit{has\_part} and its inverse \textit{is\_part\_of} to indicate connections between two legal resources. GDPRtEXT extends these to indicate hierarchical relations between chapters, sections, articles, points, and sub-points.

\subsubsection{Concepts about Data}
GDPR mentions different types of data, based on which various obligations and requirements are set forth in relation to compliance. GDPRtEXT provides the top-level concept of \textit{Data} to indicate the abstract term of `data'.

The crux of GDPR is based on personal data, which is defined in Article 4(11) and is represented by  \textit{PersonalData}, with special categories defined in Article 9(11) requiring additional obligations regarding its processing and handling, and represented by \textit{SpecialCategoryPersonalData} in GDPRtEXT. Types of special categories mentioned include criminal data, genetic data, health data, and racial data which are defined as sub-classes in GDPRtEXT.

GDPR also mentions data in the context of anonymising and pseudo-anonymising processes, with their outcomes specified using the \textit{AnonymousData} and \textit{PseudoAnonymousData} concepts.

\subsubsection{Concepts about Consent}
The top-level concept of `consent' is represented by \textit{Consent} class in GDPRtEXT with its definitions based in Articles 4(11), 6(1) and Recitals 32, 40. It is sub-classed to indicate `given consent' - which is a legal basis and is therefore a sub-class of both \textit{Consent} and \textit{LegalBasis}. \textit{GivenConsent} is further sub-classed to indicate `valid consent' which carries obligations of ensuring the consent is valid and meets the requirements of GDPR and is therefore also defined as the sub-class of \textit{ObligationForObtainingConsent}. These obligations are represented by sub-classing the \textit{ValidConsent} concept regarding conditions of - freely given, informed, specific, voluntary, and opt-in.

\subsubsection{Concepts about Entities}
\textit{Entity} represents the top-level concept of `entity' indicating any institution, company, corporation, partnership, government agency, university, or any other organization including individuals. 
It is sub-classed to indicate entities mentioned in GDPR, which are - Data Subject, Controller, Processor, Sub-Processor, Data Protection Officer (DPO), and Data Protection Authority (DPA). Additionally, relevant concepts associated with entities are also defined, which are -  Representative of Controller, Representative of Processor, Certification Body, and Regulatory Authority.

\subsubsection{Concepts about Activities}
`Activity' refers to some process or action mentioned, referred, implied, or defined by the requirements of GDPR compliance. To represent these, GDPRtEXT defines activities regarding consent and personal data processing, as well as other activities related to the functioning of the GDPR such as reporting data breach and demonstrating consent. The top-level concept `Activity' represents the abstraction of all activities, which is specialised into `ConsentActiity' and `DataActivity' to indicate involvement of consent and personal data respectively.

Consent activities defined within GDPRtEXT consist of obtaining consent and withdrawing consent. Regarding data, activities are defined to represent use, archival, collection, cross-border transfer, erasure, copying, rectifying, sharing, and storage. In these, the activity associated with usage of personal data is equivalent to its common and synonymous usage with the term `processing'. Activities for indicating context of processing include - automated processing,  automated decision making with significant effects, confirming or matching datasets, large scale processing, processing affected or vulnerable individuals, processing sensitive data, processing using untested technologies, and unlawful processing.

GDPRtEXT also provides activities associated with reporting of data breach, which include obligations and actions such as - report data breach, maintain record of breach, notify data subject of breach, report breach to controller (for processors), and report breach to DPA within 72 hours. Other activities provided are - security of personal data, appointment of processors, demonstrating consent, exercise rights, identification of data subject, impact assessment, marketing, direct marketing, monitor compliance, propagate rights to third parties, and systematic monitoring.

\subsubsection{Concepts about Compliance}
Concepts associated with compliance are provided to indicate actions or terms used in the process of maintaining, documenting, evaluating, and demonstrating compliance. The top-level concept \textit{Compliance} represents the abstract notion of compliance. Other terms derived from this include - Demonstration of Consent, Monitor Compliance, and Report Data Breach.

\subsubsection{Concepts about Principles}
GDPRtEXT represents principles using the top-level concept of \textit{Principle}, which are specialised to indicate principles associated with - Accountability; Accuracy; Data Minimisation; Integrity and Confidentiality; Lawfulness, Fairness, and Transparency; Purpose Limitation; and Storage Limitation.

\subsubsection{Concepts about Rights}
To represent rights, GDPRtEXT provides top-level concepts representing each of the rights, with further concepts associated with the rights represented as sub-classes. The right of data portability is represented by the concept \textit{RightOfDataPortability} with related concepts regarding providing copy of personal data, commonly used data format, machine readable format, structured, and supporting reuse.

The right of erasure is represented by the concept \textit{RightOfErasure} with related concepts provided regarding obligation to erase data when consent is withdrawn, or when data is no longer needed for original purpose. The right to access personal data is represented by the concept \textit{RightToAccessPersonalData} with related concepts for indicating if and where controller is processing data, whether there is automated processing with significant effects on data subject, categories of data being processed, categories of recipients data is shared with, existence of rights, information about processing, source of data, storage period, and ensuring no charges are levied for provision of rights.

The right to basic information about processing is represented by the concept \textit{RightToBasicInformationAboutProcessing} and is accompanied with its related concept regarding information about third parties. The concept \textit{RightToRestrictProcessing} represents the right to restrict processing, and is accompanied with conditions such as - accuracy is contested, data no longer needed for original purpose, and processing is unlawful. The right to transparency is represented by the concept \textit{RightToTransparency} with related concepts regarding conditions of concise, easily accessible, intelligible, and transparent. Other rights provided correspond with right to - not be evaluated through automated processing, object to direct marketing,  object to processing, and right of rectification.

\subsubsection{Concepts about Obligations}
GDPRtEXT defines concepts regarding obligations of controllers, processors, DPOs, consent, and compliant processing of personal data based on a legal basis. Obligations of controllers are represented by the concept \textit{ControllerObligation} with related concepts also provided regarding  appointment of processors, accountability, controller responsibility, co-operation with DPA, data protection by design and default, data security, liability of joint controller(s), maintaining records of processing activities, privacy by design, propagate rights to third parties, and reporting data breach.

For rights of processors, the concept \textit{ProcessorObligaion} is provided along with its related concepts for appointing sub-processors, assisting in complying with rights, compliance with controller's instructions, co-operate with DPA, data security, impose confidentiality on personnel, inform controller of conflict with law, maintain records of processing activities, only act on documented instructions, propagate rights to third parties, provide controller with information for compliance, report data breach to controller, restrictions on cross-border transfers, and return or destroy personal data at end of term.

The concept \textit{DPOObligation} represents obligations of a DPO, for which the concept \textit{MonitoringCompliance} is provided to indicate monitoring of compliance. The obligations related to lawful basis for processing are represented by \textit{LawfulBasisForProcessing} along with related concepts for contract with data subject, exempted by national law, employment law, given consent, historic, statistical, or scientific purposes, legal claims, legal obligation, legitimate interest, made public by data subject, medical, diagnostic, or treatment, not for profit org., public interest, purpose of new processing, and vital interest.

Obligations regarding valid consent are represented by \textit{ValidConsent} with related concepts to indicate that consent should be freely given, informed, specific, voluntary, and opt-in. 
Obligations for obtaining consent are represented by \textit{ObligationForObtainingConsent} and include concepts for information about third parties, indicating consent can be withdrawn easily, and conditions regarding information provided for obtaining consent such as - it should be clear, providing explanation of processing, should not be from silence or inactivity, should be demonstrable, should be distinguishable from other matters, and that is should produce valid consent.

Obligations for data collection are represented by \textit{ObligationForDataCollection}, which is accompanied with related concepts for indicating accurate collection, specification of explicit purpose, ensuring legitimate purpose, ensuring it is not further processed than original purpose, and ensuring it is limited to specified purpose.
Obligations for retention of personal data are represented by \textit{ObligationForRetentionOfPersonalData} and include related concepts about    retention of personal data, ensuring it is adequate for processing, ensuring it is identifiable for required processing, obligation to kept it up to date, ensuring it is limited for processing, obligation to rectify inaccuracies, and ensuring it is relevant for processing. 
The concept \textit{ObligationForSecurityOfPersonalData} represents the obligation regarding security of personal data, which consists of related concepts regarding accidental loss, damage, destruction, and unlawful processing.

\subsubsection{Concepts about Seals and Certifications}
GDPRtEXT provides concepts of \textit{Seal} and \textit{Certification} for representing seals and certifications as provided by GDPR to assist with the maintenance and demonstration of compliance. It also provides concepts to represent conditions for seals and certifications, represented by the concept \textit{ConditionsForSealsAndCertifications} which consist of concepts related to adherence to seal/certification, having a maximum validity of 3 years, and a voluntary system of accreditation. 

\subsubsection{Example Use-Case: Compliance Reporting}
This example use-case takes a look at how references to GDPR can aid in creation of reports which document information regarding compliance. Consider a system for creation of compliance reports that stores information related to each of the obligations it addresses from the GDPR. It uses the EARL\footnote{\url{}} vocabulary for expressing results of conformance checks within the report. GDPRtEXT is used to link the resources in EARL reports with articles and points within the GDPR as well as to express and define concepts related to compliance in a suitable and comprehensible manner. Through this, the information about compliance checks is linked and associated with the specific articles of GDPR.

EARL provides a standardized vocabulary to describe specific resources and relationships that are relevant to test reporting. The core construct of EARL is an \textit{Assertion}, which describes the context and outcome of an individual test execution. It contains the following information (copied verbatim from EARL website):

\begin{itemize}
    \item \textit{Assertor} - This can include information about who or what ran the test. For example human evaluators, automated accessibility checkers, or combinations of these.
    \item \textit{Test Subject} - This can include web content (such as web pages, videos, applets, etc.), software (such as authoring tools, user agents, etc.), or other things being tested.
    \item \textit{Test Criterion} - What are we evaluating the test subject against? This could be a specification, a set of guidelines, a test from a test suite, or some other testable statement.
    \item \textit{Test Result} - What was the outcome of the test? A test result could also include contextual information such as error messages or relevant locations within the test subject.
\end{itemize}

Taking the example of Right to Data Portability, the EARL report in Listing.\ref{listing:vocab:gdprtext-earl} represents compliance checks for conditions associated with linked articles in GDPR (Article 20). The compliance system has a module \textit{\_system\_dataportability} that checks the software that handles the provision of a copy of personal data \textit{\_org\_dataportability} through the test case \textit{\_test\_provide\_data\_copy} and generates the report which shows that the test has passed through \textit{\_result\_pass}.

\begin{lstlisting}[label={listing:vocab:gdprtext-earl},caption={Use of GDPRtEXT to link tests with GDPR Articles in EARL report}]
@prefix earl: http://www.w3.org/ns/earl# .
@prefix dct:  http://purl.org/dc/terms/ .
@prefix gdprtext: http://purl.org/adaptcentre/resources/GDPRtEXT# .

:_org_dataportability
    a    earl:TestSubject, earl:Software ;
    dct:description """System that handles data portability requests"""@en ;
    dct:title "Data Portability Handler"@en .

:_system_dataportability
    a    earl:Assertor ;
    dct:description """Module checking data portability obligations"""@en ;
    dct:hasVersion "1.4" ;
    dct:title "DataPortability Module"@en ;
    earl:asserts { :_org_dataportability :_result_pass :_test_provide_data_copy } .

:_result_pass
    a    earl:ResultProperty ;
    earl:date "2018-01-01" ;
    earl:validity earl:Pass ;
    earl:confidence earl:High .

:_test_provide_data_copy
    a    earl:TestCase ;
    earl:testMode earl:automatic ;
    dct:title "Test provision of data copy"@en ;
    dct:description """Tests whether system provides a copy of personal data on exercising right to data portability"""@en ;
    dct:subject gdprtext:article20 .
\end{lstlisting}

Now to gather such related resources together, a SPARQL query (simplified) would focus on the link between \textit{TestCase} and its result using \textit{earl:validity}, as shown in Listing.\ref{listing:vocabs:gdprtext-sparql}.
These tests can be further combined into test suites to group compliance checks related to each article or a particular concept and structure the documentation around this form of logical grouping of concepts.
In this manner, the use of GDPRtEXT to link tests and results with documentation enables automation of information retrieval and management.

\begin{lstlisting}[label={listing:vocabs:gdprtext-sparql},caption={SPARQL query and results showing retrieved GDPR test results by article}]
SELECT ?gdpr ?result ?confidence ?mode WHERE {
    ?assertor a earl:Assertor .
    ?assertor earl:asserts ?assertion .

    ?testcase rdf:predicate ?assertion .
    ?testcase a earl:TestCase .
    ?testcase dct:subject ?gdpr .
    ?testcase ear:testMode ?mode .

    ?testresult rdf:object ?assertion .
    ?testresult a earl:ResultProperty .
    ?testresult earl:validity ?result .
    ?testresult earl:confidence ?confidence .
}

| gdpr          | result     | confidence     | mode          |
|-----------    |--------    |------------    |-----------    |
| article16     | pass       | low            | automatic     |
| article17     | pass       | high           | automatic     |
| article18     | fail       | high           | manual        |
| article19     | pass       | high           | automatic     |
\end{lstlisting}

\subsubsection{Example Use-Case: Mapping between DPD and GDPR obligations}
The second application of GDPRtEXT demonstrates the linking of obligations between the GDPR and Data Protection Directive (DPD), which is the previous data protection legislation. Given that DPD was adopted in 1995, and was superseded by the GDPR in 2016, there are a large number of solutions and approaches regarding compliance with DPD that already exist and are used in practice. By linking the obligations between DPD and GDPR it is possible to investigate reuse of these existing solutions for GDPR compliance. To that end, a mapping from DPD obligations to GDPR obligations containing annotations that describe the nature of change between the two is constructed by linking the articles of DPD and GDPR.

To model the annotations as a RDF resource, a linked data version of the DPD was created similar to GDPRtEXT which assigned URIs for every resource in the legislation. This enabled referring to each individual clause in the DPD and linking it with relevant clauses in the GDPR. 
The annotations, available online\footnote{\url{https://openscience.adaptcentre.ie/projects/GDPRtEXT/dpd_mapping.html}}, are consist of references from a clause in DPD to its corresponding clause in the GDPR with an expression of change between the two. The nature of change is represented by the values of: same - indicating no change, reduced - indicating reduction of obligation, slightly changed - indicating minor change, completely changed - indicating major change, and extended - indicating addition of obligations.

To demonstrate the application of existing work regarding DPD compliance towards meeting GDPR obligations, previous existing work using XACML rules to denote DPD compliance \cite{}\cite{} were utilised to assess their suitability in meeting GDPR requirements.
For each link between DPD and GDPR obligations in the annotation, a record was also added to indicate whether the corresponding XACML rule for DPD compliance needed to be changed. The notation \textit{N/A} was used to denote the case where no XACML rules existed for a particular DPD obligation and the corresponding obligation in GDPR had changed and had additional requirements. 
% The value \textit{No} was used to indicate no changes in the GDPR obligation compared to the DPD obligation, so that the existing XACML rule would be sufficient to meet GDPR requirements. Similarly \textit{Yes} was used to indicate a change required in the XACML rule to handle the obligation.

The class \textit{DPDToGDPR\_Annotation} represents annotations between DPD and GDPR, with an example instance depicted in Listing.\ref{listing:vocabs:gdprtext-xacml}. The property \textit{resourceInDPD} is used to refer to the particular clause within DPD through its URI. Similarly, the property \textit{resourceInGDPR} is used to refer to the corresponding clause in GDPR. The nature of change is defined using the property \textit{hasChange} whose value is an instance of the class \textit{ChangeInObligation}, with defined instances for Extended, Same, Reduced, CompletelyChanged, and SlightlyChanged. Similarly, the change in the XACML rules is defined as a property whose values are one of Yes, No, and N/A defined as instances of the class \textit{ChangeInXACMLRule}. Comments are defined using the \textit{rdfs:comment} property.
\begin{lstlisting}[label={listing:vocabs:gdprtext-xacml},caption={Example annotation of associating obligation between DPD and GDPR with indication of corresponding changes required to reuse DPD compliance XACML rules for GDPR requirements}]
@prefix gdpr: https://w3id.org/GDPRtEXT/gdpr# .
@prefix dpd: https://w3id.org/GDPRtEXT/dpd# .
@prefix rdfs: http://www.w3.org/2000/01/rdf-schema# .

dpd:mappingrule6
    a dpd:DPDToGDPR_Annotation ;
    dpd:hasChange dpd:ChangeExtended ;
    dpd:hasXACMLChange dpd:XACMLNoChange ;
    dpd:resourceInDPD dpd:Article7 - a ;
    dpd:resourceInGDPR gdpr:Article6-1-a ;
    rdfs:comment "added consent given to ..." .
\end{lstlisting}


\subsection{Evaluation}
The assessment of the extent to which GDPRtEXT meets the requirements outlined in \autoref{sec:voc:gdprtext-engineering} consists of evaluating whether the ontology and dataset produced is sufficient to enable granular linking of information with the concepts and clauses of the GDPR. Based on the presented work, GDPRtEXT meets these requirements and enables answering the competency questions related to use of GDPR in the compliance process.

In terms of ontology assessment, the methodology outlined in \autoref{sec:voc:methodology} provides the criterion for evaluation of the quality of ontology as well as its documentation. GDPRtEXT fulfils these, based on tests undertaken using the OOPS! tool and by following the best practices community guidelines for ontology documentation.
The publication of GDPRtEXT in Irish open data portal demonstrates the quality of the work, based on the 5 star rating given to it as a linked open dataset.

GDPRtEXT, and the work described in this section, was published in the resource track at Extended Semantic Web Conference \cite{}. The publication described the creation of the resource, summarised its contents, and also provided mapping of DPD obligations with GDPR using a linked data approach and XACML to denote which obligations from DPD could be re-used towards GDPR compliance. Through this, the research and developed resources were peer-reviewed and adopted.
Further validation is provided through a survey of legal approaches and works in the state of the art \cite{} which includes GDPRtEXT as one of the resources surveyed and demonstrates that GDPRtEXT is unique in its provision of GDPR as a machine-readable resource.

\subsection*{Summary}
The GDPRtEXT resource represents the first major contribution of this thesis. It provides a linked data version of the text of GDPR and a vocabulary of its concepts, and fulfils research objectives $RO3(a)$ and $RO5(b)$ - as outlined in \autoref{sec:intro:RQ}. It enables exposing each individual article or point within the GDPR as a unique resource through URIs provided using semantic web notations.
GDPRtEXT thus enables machine-readable links to be established between information and the text of GDPR as well as concepts pertaining to its compliance.

The use of GDPRtEXT makes it possible to create approaches that automate the generation and querying of information associated with GDPR - such as for compliance, management of business processes, or generation of privacy policies. The compatibility offered by use of ELI ontology ensures alignment with official documents produced by the European Publications Office in the future.
Finally, GDPRtEXT fills an important gap in the state of the art regarding machine-readable approaches for linking information with legal text.
GDPRtEXT has been released as an open resource under the permissive CC-by-4.0 license. It has been published in Zenodo, Datahub, and has been incorporated into Ireland’s open data portal as a 5-star linked open dataset.

% GDPRov
\section{GDPRov - Ontology for GDPR activities associated with Personal Data and Consent}\label{sec:voc:GDPRov}
This section describes the GDPRov ontology for representing activities in ex-ante and ex-post phases associated with processing of personal data and consent for GDPR compliance. GDPRov stands for GDPR Provenance - which is a reference to the requirement of maintaining provenance information of processes in both ex-ante and ex-post phases representing for demonstrating GDPR compliance. This section presents the motivation, description, publication, and evaluation of the GDPRov ontology. It also presents comparisons with relevant approaches in the state of the art. GDPRov is available online\footnote{\url{}} with its documentation and code repository\footnote{\url{}}.

Following the motivation arising from importance of documenting activities associated with personal data processing and consent, the research objectives $RO3(b)$ and $RO3(c)$ in \autoref{sec:intro:RQ} outline the requirement for an ontology to represent this information.
To ensure such an ontology meets the requirements of the GDPR, an analysis of information regarding GDPR compliance requirements was carried out to identify the requirements.
The result of this exercise were the information regarding GDPR presented in \autoref{chapter:background} and the compliance questions presented in \autoref{sec:info:compliance-questions}.
These compliance questions were then used as competency questions to identify requirements for the ontology.
Simultaneously, an analysis of state of the art was carried out to identify existing semantic web representations of activities which were suitable for representing the information for GDPR compliance, which could be reused or extended compatible manner, and which were published in an open and permissive license permitting their usage in this research.

The identified concepts and relationships were combined to create the GDPRov ontology. It utilised GDPRtEXT to provide definitions from within the text of the GDPR, thereby linking the ontology to relevant clauses and concepts of the GDPR. 

\subsection{Identification of requirements from competency questions}
The compliance questions presented in \autoref{} provided the competency questions for deriving concepts and relationships regarding processes associated with personal data and consent in the context of GDPR compliance requirements. 
These concepts and relationships were collected, combined, and analysed to ensure their cohesion as an ontology and evaluated against the compliance questions to ensure they satisfied requirements regarding GDPR compliance and documentation of associated processes.
In this, the aspect of ex-ante and ex-post processes provides a form of duplication as most processes have their counterparts in both phases, and which is linked and documented in a manner so as to demonstrate the prior planning of processes to ensure their compliance and their execution which also needs to be documented to demonstrate compliance.
Therefore, while GDPR requirements and the compliance questions do no explicitly mention or provide clues to the ex-ante and ex-post phases, the development of ontology explicitly considers each activity to have representations in both phases as a compliance requirement.

In each of the sub-sections below, the relevant concepts arising from the denoted compliance question are described along with relevant relationships associated with those concepts. This is followed by an analysis of their role in ex-ante and ex-post phases. The section provides the analysis of concepts and relationships which act as requirements towards the construction of the GDPRov ontology, and serve to describe the motivation behind its design and implementation.

\subsubsection{Actors and Agents involved in activities}
\begin{itemize} 
    \item \textit{CMQ2} - Provides the concept of Controller as an agent controlling the processes as defined by and its representative Data Protection Officer (DPO).
    \item CMQ17 - Describes the Processor as an executor of processes and its representative DPO. In this relationship, the Controller provides such processes to the Processor to execute, which is governed by a Data Processing Agreement (DPA) between the two.
    \item CMQ35 - Describes Data Subject as an agent who is associated with the provision of personal data, consent, and who is related to the exercising of rights.
\end{itemize}

\subsubsection{Details of processing}
\begin{itemize}
    \item \textit{CMQ3} and \textit{CMQ37} provide the concept of Purpose which describes the purpose of personal data processing. Each purpose can incorporate multiple processing operations, and each processing operation taking place can be associated with multiple purposes.
    \item \textit{CMQ4} describe the necessity to specify data subject categories whose personal data is being processed.
    \item \textit{CMQ36} describes personal data, while \textit{CMQ5} describes categories of personal data being processed. \textit{CMQ34} specifies special categories of personal data as a sub-category of personal data which needs to be explicitly stated as being processed.
    \item \textit{CMQ38} defines processing of personal data as defined by Article 4-2 of GDPR. The GDPR definition of processing provides types of operations considered under processing, as specified by ``any operation or set of operations which is performed on personal data or on sets of personal data, whether or not by automated means, such as collection, recording, organisation, structuring, storage, adaptation or alteration, retrieval, consultation, use, disclosure by transmission, dissemination or otherwise making available, alignment or combination, restriction, erasure or destruction;''.
    \item \textit{CMQ6} defines sharing of data as a type of processing. Additional information associated with sharing of data is provided by - \textit{CMQ7} and \textit{CMQ20} for categories of recipients; \textit{CMQ8},\textit{CMQ21} for identifies of recipients, \textit{CMQ9} and \textit{CMQ22} for location where data is being sharing to; \textit{CMQ10} and \textit{CMQ23} for safeguards associated with data transfer; \textit{CMQ15} and \textit{CMQ25} for purposes of sharing, which is the same concept as purpose of processing except applied for sharing of personal data.
    \item \textit{CMQ11} defines data storage, with additional concepts provided by \textit{CMQ12} for existence of time limits or conditions for erasure and \textit{CMQ13} for specification of time limits or conditions for erasure for categories of data.
    \item \textit{CMQ26} defines legal basis for justifying the processing of personal data, and \textit{CMQ27} specifies legal basis associated with a particular purpose. Each purpose can have one or more legal basis associated with it.
\end{itemize}

\subsubsection{Lifecycle of data}
\begin{itemize}
    \item \textit{CMQ28} and \textit{CMQ30} describe source of personal data which in turn implies an activity that collects data and specifies the actor or agent providing the data.
    \item \textit{CMQ29} specifically refers to personal data collected from data subject.
\end{itemize}

\subsubsection{Anonymisation}
\begin{itemize}
    \item \textit{CMQ31} specifies anonymisation of personal data, with \textit{CMQ32} inquiring about different `levels' of anonymisation which affect the application of obligations and requirements of compliance. 
    \item The levels are specified based on their application in the process of compliance, and include data which is completely anonymised, data which is pseudo-anonymised, and data which is not anonymised. In this, data that is pseudo-anonymised can be considered and used as anonymous data under the condition that the organisation does not have additional information to de-anonymise it. 
    \item From this, the processing associated with anonymisation and de-anonymisation of personal data are defined.
\end{itemize}

\subsubsection{Activities associated with Consent}
\begin{itemize}
    \item Regarding consent, \textit{CMQ48} inquires about activities associated with the provision and collection of consent. This includes information about how the consent is requested and collected, used within processes as a legal basis, and is archived for future demonstration of compliance.
    \item \textit{CMQ49} and \textit{CMQ50} inquire about artefacts associated with the collection of consent as determination of validity of consent under GDPR require investigation of how choices for consent were offered. This also includes the form in which consent is provided or collected from the data subject. The artefacts are associated with the processes where consent choices are offered or requested and whose result is the collection of consent.
\end{itemize}

\subsubsection{Provision of Rights}
\begin{itemize}
    \item The rights associated with GDPR need processes to internally (from the perspective of the organisation) handle their execution as well as for interaction with the data subject. Therefore, such processes need to be defined and documented for compliance purposes.
    \item In the case of right to be informed, \textit{CMQ88 - CMQ105} provide the competency questions regarding how the right is provided and how it is executed or implemented.
    \item This includes activities associated with the provision of information to the data subject, artefacts associated with information provision, inclusion of details such as controller and DPO, purposes, processing, legal basis, personal data categories.
    \item It also includes information about sources of personal data (where not obtained directly from the data subject), and whether the legal basis is legitimate interest.
    \item Regarding data sharing, the information to be specified includes categories of recipients , and their location.
    \item The right to be informed also includes provision of information regarding the existence and application of rights.
    \item The information associated with the right to be informed is common to other information documented in the due course of processing of personal data, and therefore does not specifically require separate notation or representation of this information in order to execute the right. the existing information or concepts can be reused for specifying the required information. However, activities associated with the right need to be defined to demonstrate the existence of processes for handling the right.
\end{itemize}

\subsubsection{Compliance procedures such as Reporting of Data Breach}
\begin{itemize}
    \item The reporting of data breach requires information about data breach to be maintained, as specified by \textit{CMQ106 - CMQ120}.
    \item This includes information about the data breach, which includes timestamp of when the breach occurred (\textit{CMQ106}), timestamp of when the controller became aware of it (\textit{CMQ107}), timestamp and method of it being notified to supervisory authority (\textit{CMQ108}).
    \item Information about contents of breach include information about its affected personal data and categories of data subject (\textit{CMQ112}).
    \item This information is associated with the process of reporting and documenting data breach in the form of artefacts. This information also needs to be provided to the supervisory authority and in some cases to the data subject based on the extent of the breach (\textit{CMQ113}) and therefore requires prior plans to execute the process and handle a data breach and send the information to data subjects along with any remedial measures (\textit{CMQ116}).
\end{itemize}

\subsubsection{Specifying requirements for ex-ante and ex-post phases}
Process logs are a convenient and demonstrable form of information to store and document the compliant processing of personal data. By verifying such logs, it is possible to evaluate, document, and demonstrate that the executed processes were compliance with the requirements of the GDPR. This constitutes as ex-post documentation of compliance. Along with this, it is also essential to demonstrate that the executed processes were based on a pre-conceived plan or template that was ensured to be compliant before the actual execution. Storing such plans is essential to demonstrate prior planning and maintaining compliance. This constitutes as ex-ante documentation and demonstration of compliance.

Associating the executed processes with their plans allows demonstration of compliance throughout the lifecycle of the process. It also enables to document the change in plans and its effects on execution of processes - i.e. when a plan changes, it also brings about corresponding changes in the executed processes. In the context of GDPR compliance, the requirements of compliance require documentation, maintenance, and demonstration of processes across both ex-ante (planning) and ex-post (execution) phases. The ex-ante plans of processes are described as an organisational measure and their compliance is associated with ensuring processes meet legal requirements before they are actually carried out. In some instances, such as for DPIA, the existence of ex-ante information about processes is essential to the evaluation of compliance.

With this, the requirements for representing processes can be summarised with the following points:
\begin{enumerate}
    \item Represent process in ex-ante phase as log or record.
    \item Represent process in ex-post phase as plan or template.
    \item Link ex-post plan with its instantiations or executions in ex-post phase as logs.
    \item Track the provenance of ex-ante plans i.e. changes in plans of processes.
    \item Enable tracking changes in ex-post logs based on corresponding changes in ex-ante plans.
    \item Associate information with processes as artefacts.
    \item Associate actors/agents with processes.
    \item Link processes based on:
        \begin{enumerate}
            \item dependency - where one process is dependant on another,
            \item order of execution - where one process is or will be executed before or after another, and
            \item composition - where one process is constituted by several sub-processes.
        \end{enumerate}
\end{enumerate}

\subsection{Extending PROV-O and P-Plan}
Based on the above stated requirements for representing activities or processes in ex-ante and ex-post phases, the existing semantic web ontologies of PROV-O \cite{} and P-Plan \cite{} were extended with relevant GDPR concepts and relationships to create the GDPRov ontology. The necessity of this process and a brief overview of PROV-O and P-Plan ontologies is described below along with the process of extending the two ontologies.

\subsubsection{PROV - W3C standard for representing provenance information}
Provenance is information about entities, activities, and people (or software)
involved in producing data or a component which can be used to form an
assessment about its quality, reliability, or trustworthiness. The PROV-O ontology \cite{} along with PROV family\footnote{\url{https://www.w3.org/TR/2013/NOTE-prov-overview-20130430/}} of schemas and documents is the W3C recommendation since 30\textsuperscript{th} April 2013 for representing provenance information.
It provides definitions for interchange of provenance information by representing entities
and relations between them such as generated by, derived from, and attributions.

The core concepts of PROV-O are summarised in \autoref{fig:prov-o-model}, and consist of interactions between Activities, Entities, and Agents.
An Entity in PROV-O is defined as being physical, digital, conceptual, or other
kind of thing with some fixed aspects. PROV-O defines an Activity as something
that occurs over a period of time and acts upon or with entities; it may include
consuming, processing, transforming, modifying, relocating, using, or generating
entities.
\begin{figure}[htbp]
    \centering
    \includegraphics[width=\linewidth]{img/prov-o-model.png}
    \caption{Overview of PROV-O model \cite{}}
    \label{fig:prov-o-model}
\end{figure}

PROV-O is a generic and domain independent ontology for representing provenance information.
In order for it to be applied to the domain of GDPR compliance, it needs to incorporate the relevant terminology and enable distinction between different types of activities and entities.
Further, PROV-O as a provenance ontology, is intended to represent information about activities that have been executed in the past, and is therefore suitable to represent only the ex-post aspect of GPDR compliance processes.
While PROV-O does provide the concept of `Plan' to represent ex-ante information, it does not provide further concepts or relationships to associate the plan with activities and entities.
Therefore, in order to adopt PROV-O for representing information for GDPR compliance, it needs to be extended with the relevant concepts and relationships.

\subsubsection{P-Plan - extending PROV-O Plans as Workflows}
P-Plan \cite{} extends the concepts of `Plan' in PROV-O towards representing scientific
workflows, which enables creating a template of a `step' and linking it to executions of activities.
A \textit{p-plan:Plan} is a subclass of \textit{prov:Plan} and is composed of smaller activities or steps (\textit{p-plan:Step}) that use and generate (as inputs or outputs of steps) variables (\textit{p-plan:Variable}).
An overview of the relationship between PROV-O and P-Plan is described in \autoref{fig:p-plan-model}.
P-Plan enables the representation of provenance information associated with both ex-ante and ex-post processes by representing them as scientific workflows. It also enables associating plans with their executions, thereby providing a link between ex-ante and ex-post provenance information.
\begin{figure}[htbp]
    \centering
    \includegraphics[width=0.75\linewidth]{img/p-plan-model.png}
    \caption{Overview of P-Plan model and its relationship with PROV-O \cite{}}
    \label{fig:p-plan-model}
\end{figure}

A \textit{p-plan:Plan} represents information of `how’ something should happen or a ‘template’ for executions. A \textit{p-plan:Activity} is a subclass of \textit{prov:Activity} and represents the execution of the process described in a \textit{p-plan:Step}.
A \textit{p-plan:Entity} is a subclass of \textit{prov:Entity} that corresponds to a \textit{p-plan:Variable} in the \textit{p-plan:Plan}. Therefore, a
\textit{p-plan:Step} may describe the template including inputs and outputs which can
then be instantiated into multiple instances of \textit{p-plan:Activity} that can have
distinct inputs to produce different outputs.
As \textit{p-plan:Plan} extends \textit{prov:Plan}, which itself extends \textit{prov:Entity}, it can be
used to treat the \textit{p-plan:Plan} as an object whose provenance can be tracked using
PROV-O or P-Plan. This makes it possible to express provenance of processes that themselves also describe provenance, thereby creating a history of how plans were formulated and executed over time.

\subsubsection{Extending ontologies for GDPR}
The PROV-O and P-Plan ontologies were extended to represent concepts and relationships of ex-ante and ex-post activities associated with personal data and consent based on requirements of GDPR compliance.
The decision to extend PROV-O and P-Plan with GDPR concepts was made as both ontologies contain generic concepts associated with activities and workflows which can be used for representing information about GDPR compliance, but doing so would be not be intuitive due to the difference in terminology and structuring of information as expected for GDPR compliance.

Extending existing ontologies of PROV-O and P-Plan enables expressing a `template'
or `plan' using \textit{p-plan:Plan} describing ex-ante activities (as \textit{p-plan:Step}) that can take place. This template can then be used to denote execution of activities in ex-post phase using \textit{p-plan:Activity}.
This provides a machine-readable and documented data model of both ex-ante and ex-post activities, whose provenance itself can be expressed (using PROV-O and P-Plan) to record how they were created and how they  change over time.
This is beneficial in documenting the state of a system at a given time as a set of activities that deal with consent and personal data, and
can be helpful in determining changes when the interactions between personal data and an activity change over time.

The extended ontology derived from PROV-O and P-Plan incorporates concepts and relationships associated with GDPR in order to normalise the terminology for representing information associated with GDPR compliance.
The concepts and relationships are derived from the competency questions and linked with their relevant clauses within the GDPR through the use of GDPRtEXT by using \textit{rdfs:isDefinedBy} and \textit{rdfs:seeAlso}.
This provides a form of documentation regarding the origin of concepts and their use in the representation of information associated with those clauses of the GDPR.
It also provides a machine-readable link from the ontology to GDPR, which can be used to compare, analyse, and align relevant ontologies.

The extension consists of subclassing existing concepts in PROV-O and P-Plan to represent specific activities associated with GDPR compliance. 
The use of subclass mechanism preserves the existing concepts and relationships of PROV-O and P-Plan so as to provide compatibility and reuse. This is particularly important in the case of PROV-O as it is the W3C standard for representing provenance information and therefore is more likely to used and expected within the community.
The compatibility also enables packaging the information defined using the ontology as an artefact and defining its provenance as well as planning to provide meta-documentation about how compliance is to be planned and maintained. This is particularly useful to maintain periodic snapshots of organisational processes associated with compliance, and provides the opportunity to automate the querying and validation of information checks within a use-case.

\subsection{Ontology Description \& Application}
The resulting ontology is named GDPRov (GDPR Provenance Ontology) and is published online along with its documentation at \url{https://w3id.org/GDPRov/} under the open and permissive CC-by-4.0 license.
The ontology was created, documented, and published using the methodology presented in \autoref{sec:voc:methodology}.
The aim of GDPRov is to provide representations of ex-ante and ex-post activities regarding personal data and consent for GDPR compliance.
It uses the GDPRtEXT ontology to define concepts based on their origin and relevance to clauses within the text of GDPR.

\subsubsection{Overview of GDPRov concepts}
GDPRov extends concepts from PROV-O and P-Plan to represent activities associated with GDPR compliance, with a visual overview provided in \autoref{fig:vocabs:gdprov-overview}.
To that end, it extends the concept of \textit{p-plan:Plan} in the form of \textit{Process} to represent ex-ante plans of activities that will take place.
Each \textit{Process} can contain `step' (represented by \textit{p-plan:Step}) which represent activities that interact with data and agents (actors in provenance terminology).
To associate steps with a process, the property \textit{p-plan:isStepOfPlan} is extended as \textit{isPartOfProcess}.
Another additional property - \textit{refersToProcess} is also used to enable referring to a process without being a part of it.
Similarly, to associate data (defined in P-Plan as \textit{p-plan:Variable}) the properties \textit{p-plan:hasInputVar} and \textit{p-plan:isOutputVarOf} are extended for activities using inputs and producing outputs respectively.
The ex-post activities in P-Plan are represented by \textit{p-plan:Activity}.
Data interactions with these activities is represented by \textit{p-plan:Entity} and the properties \textit{prov:used} and \textit{prov:wasGeneratedBy} are used to indicate inputs and outputs respectively.
GDPRov defines steps to indicate automated execution and user interactions regarding collecting data from the user (input) and providing data (output).
To indicate the legal basis associated with a process or a step, the property \textit{hasLegalBasis} is provided.
\begin{figure}[htbp]
    \centering
    \missingfigure[figcolor=white]{}
    \caption{Figure describing GDPRov concepts by extending PROV-O and P-Plan}
    \label{fig:vocabs:gdprov-overview}
\end{figure}

\subsubsection{Depicting Data Lifecycle}
Activities associated with the lifecycle of personal data constitute of collecting, processing or using it, storing, sharing, deleting, transferring, transforming, anomymising, and rectifying it. GDPR defines several more categories of actions on data in Article 4-2 in its definition of `processing'.
GDPRov provides broad and abstract processes to represent data access, data archival, data erasure, and data rectification given the need to execute these using several more steps.
GDPRov also provides representations of actions in ex-ante phase as \textit{DataStep} which extends \textit{p-plan:Step} and in ex-post phase as \textit{DataActivity} which extends \textit{p-plan:Activity}.
These are further extended to distinguish between data collection, data deletion, data sharing, data storage, data archival, data transfer, data transformation, data usage, and rectification of data.
A visual summary of these is provided in \autoref{fig:vocabs:gdprov-data-lifecycle}.

The anonymisation of data is defined as a sub-class of data transformation to indicate the transformation of data that takes place when anonymising it.
As GDPR obligations are based on the level of anonymity and the capability of de-anonymising it from an organisation's point of view, GDPRov provides the concept of \textit{anonymisation level} to indicate the state of anonymity the data is in.
GDPRov defines four levels of anonymisation based on existing work in representing anonymous data \cite{}, which constitute of data that is completely anonymised, completely deanonymised, pseudo-anonymised, and pseudo-organisational-anonymised where the organisation does not have the data required to de-anonymise it and can thus internally utilise it as if it were completely anonymous data.
The sharing of data consists of interactions with actors or agents, which are represented by \textit{prov:Agent} and associated with the respective steps and activities using extended properties.
\begin{figure}[htbp]
    \centering
    \missingfigure[figcolor=white]{}
    \caption{Figure describing data lifecycle defined using GDPRov}
    \label{fig:vocabs:gdprov-data-lifecycle}
\end{figure}

The personal data used within activities is represented by \textit{PersonalData} which is sub-classed from \textit{p-plan:Variable} for ex-ante representation and by \textit{PersonalDataEntity} which is sub-classed from \textit{prov:Entity} for ex-post representation.
Further categorisation of personal data into anonymised, sensitive, and representing user identifier is provided through sub-classes.

\subsubsection{Depicting Consent Lifecycle}
Activities associated with consent and its lifecycle are represented in ex-ante phase by sub-classing \textit{p-plan:Step} as \textit{ConsentStep} and in ex-post phase by sub-classing \textit{p-plan:Activity} as \textit{ConsentActivity}.
These are further sub-classed to represent the acquisition, archival, modification, and withdrawal of consent.
Amongst these, withdrawal of consent is defined as sub-class of modification since it modifies the state of consent.
A visual summary of these is provided in \autoref{fig:vocabs:gdprov-consent-lifecycle}.

The artefacts associated with consent and used in activities include the choices or offer of consent provided to the individual and the subsequent consent given by the individual.
To represent these in ex-ante phase, GDPRov provides the concepts of \textit{ConsentAgreementTemplate} to represent the template offered to collect consent, \textit{ConsentAgreement} to indicate the given consent, and \textit{TermsAndConditions} to indicate the policies or terms and conditions.
The corresponding concepts in ex-post phase are \textit{GivenConsentTemplate}, \textit{GivenConsent}, and \textit{TermsAndConditionsEntity}.
\begin{figure}[htbp]
    \centering
    \missingfigure[figcolor=white]{}
    \caption{Figure describing consent lifecycle defined using GDPRov}
    \label{fig:vocabs:gdprov-consent-lifecycle}
\end{figure}

\subsubsection{Depicting Compliance-related processes}
In addition to representing activities associated with personal data and consent, GDPRov also provides reprentations for compliance-related processes.
These include actions such as appointing processor (by a controller), carrying out an impact assessment, marketting and its special case of direct marketting, and monitoring compliance.
Processes are also provided for handling data breaches, which include notifying the controller (by a processor), notifying the data subject, and notifying the data protection authority.
The handling of right provided by the GPDR is represented through sub-classes of \textit{Process} for data portability, erasure, access personal data, basic info about processing, no automated processing, object to direct marketting, object processing, rectification, restrict processing, transparency, SAR (subject access request).

% \subsubsection{Actors and Agents}
% Controller Representative
% Data Subject
% DPO
% Processor Representative
% Third Party
%     Controller
%         Joint Controller
%     Processor
%         SubProcessor

% \subsubsection{Documentation \& Dissemination}

\subsection{Evaluation}
The ontology assessment was based on the methodology outlined in \autoref{sec:voc:methodology} regarding the criterion for evaluation of the quality of ontology and its documentation.
The ontology satisfies the requirements of the competency questions in representing activities associated with personal data and consent.
It goes further in providing relevant and useful concepts associated with other activities and processes associated with GDPR compliance such as those associated with data breaches and handling of rights, which shows its applicability in other use-cases.
GDPRov was published \cite{} in a peer-reviewed publication and has been cited in other relevant works within the state of the art, which lends credence to its contributions.
Subsequent revisions included addition of new concepts associated with practical real-world implementation of GDPR compliance, particularly those associated with consent mechanisms on the internet.

The applicability and usefulness of GDPRov is demonstrated through its use for querying and validation of information for GDPR compliance in \autoref{chapter:testing}.
A simplified example demonstrating such an application was published along with the ontology in the peer-reviewed publication \cite{}, and which is presented in \autoref{listing:vocabs:gdprov-query} demonstrate how GDPRov can assist in the answering of compliance questions for GDPR.

\begin{lstlisting}[label={listing:vocabs:gdprov-query},caption={SPARQL query demonstrating use of GDPRov in representing and retrieving steps, data, anonymity level, and anonymisation process based on sharing of data \cite{}}]
PREFIX gdprov: <https://w3id.org/GDPRov#>

SELECT ?data ?sharestep ?isAnonymised ?anonymisationStep
WHERE {
    ?data a gdprov:Data .
    ?sharestep a gdprov:DataSharingStep .
    ?sharestep gdprov:sharesData ?data. 
    BIND (
        EXISTS { ?data a gdprov:AnonymisedData . }
        as ?isAnonymised ) .
    OPTIONAL {
        ?anonymisationStep
        gdprov:generatesAnonymisedData ?data .
    }
}
\end{lstlisting}

\subsection*{Summary}
GDPRov represents the second major contribution of this thesis by providing an ontological representation of ex-ante and ex-post activities associated with personal data and consent for GDPR compliance.
It thus fulfils the research objective $RO3(b)$ as outlined in \autoref{sec:intro:RQ}. 
The use of GDPRov makes it possible to indicate the plans associated with how personal data and consent is collected, used, stored, shared, and erased.
It also enables the representation of logs for activities that act over personal data and consent.


At the time of this undertaking (in 2016-2017), no other vocabulary was found that represented information about activities associated with GDPR compliance.
The work presented as state of the art in \autoref{chapter:sota} and demonstrating existence of approaches for representing information about GDPR processes were published after this period, and were thus subsequently used to compare and analyse the approaches. Of these approaches, some also utilise PROV-O to represent provenance of activities. The differentiating factor of GDPRov is the use of PROV-O and P-Plan to represent ex-ante and ex-post activities based on the scientific workflow model. Another differentiating factor is the use of GDPRtEXT to define concepts and relationships within GDPRov, thereby linking the use of ontology with the legal concepts it was derived from.

Approaches within the state of the art, such as SPECIAL \cite{}, demonstrate the applicability of provenance vocabularies in maintaining, querying, and assessing provenance logs represented using PROV-O for GDPR compliance.
While SPECIAL also provides ex-ante compliance assessment by using the same data model and logging it as a request instead of execution \cite{}, GDPRov further expands on the use of provenance to include the representation of plans or templates to indicate the association between activities in ex-ante and ex-post phases of compliance.

% GConsent
\section{GConsent - Ontology for Consent Information for GDPR Compliance}\label{sec:voc:GConsent}
GConsent is a semantic web ontology for representing contextual information about consent based on requirements of GDPR compliance. 
GConsent aims to model the context, state, and provenance of consent as an entity.
Its scope is limited to consent as defined in the GDPR towards assisting in the modelling of information associated with compliance of given consent.
It uses GDPRtEXT to denote the origin and relevance of its concepts within the text of the GDPR.
GConsent is the outcome of applying the methodology presented in \autoref{sec:voc:methodology} to identfiy and represent information about consent required to determine compliance.
For this, the information presented in \autoref{chapter:background} was used to identify the validity of consent defined by GDPR, along with the requirements and compliance questions presented in \autoref{chapter:information} used as competency questions.
GConsent is published online\footnote{\url{https://w3id.org/GConsent}} with its documentation under the open and permissive license of CC-by-4.0 and its code repository\footnote{\url{https://github.com/coolharsh55/GConsent/}}.

\subsubsection{Distinction with existing work in state of the art}
Information about consent needs to be maintained and shared by multiple parties - data subject, controller, processor, and authorities - which requires its representation to be interoperable between them, as outlined in the model in \autoref{}.
From the existing work analysed in state of the art in \autoref{chapter:sota}, the focus of approaches for consent is mostly on `given' consent i.e. consent provided by the data subject. There is a lack of work regarding representing other `states' of consent wtihin its lifecycle as an entity or representation of agreement which are relevant to its use as a legal basis in the determination of processing of personal data and its compliance under GDPR.
Examples of such states are - `not given' or `refused' or `withdrawn' - and these cannot be modelled in the same manner as `given consent' since they are merely a state or annotation associated with a particular instance of processing the data subject's personal data.
Apart from the notion of states, the existing approaches also a lack modelling representations for events such as delegation or associations with third parties regarding consent which have an effect on its validity regarding compliance.
GConsent aims to fill this gap, and therefore to provide novelty and contribution by representing a more cohesive and complete representation of information associated with consent under the GDPR.

\subsubsection{Relationship with GDPRov}
GDPRov builds upon and is complimentary to the representation of consent in GDPRov.
The definition of consent as an entity involved in activities is sufficient to express its lifecycle and provenance by using GDPRov. This includes the ex-ante representation of information provided to collect consent and its subsequent agreement by the individual to produce given consent, which is then used within activities as a legal basis, and may be modified, withdrawn, or revoked - signalling its effective end of lifecycle.
While GDPRov is sufficient to represent these states of consent as an entity along with information about activities acting on it, the primary focus of GDPRov is about expressing the information from the point of view of activities.
Managing consent as a legal basis involves consideration of information such as purpose of processing, recipients, contextual information such as medium of provision and collection, and situations such as delegation.

GConsent aims to provide a consent-centric representation of these information categories by providing concepts relevant to the resolution of valid consent as defined by requirements of GDPR compliance. In this, the use of provenance concepts show an overlap with GDPRov.
This is resolved through the differing scopes of the two ontologies, where concepts defined within GDPRov can also be defined using corresponding concepts in GConsent and vice-versa. An example of such cohesive usage is demonstrated through the application of developed ontologies for querying and validation of information in \autoref{chapter:testing}.

\subsection{Requirements Gathering and Establishment of Competency Questions}
The scope of consent as represented within GConsent is limited to the definition of consent as provided by Article 4-11 of GDPR.
Other special cases of consent not included within the scope consist of consent defined by Article 9 regarding use of special categories of personal data, Recital 33 regarding use of personal data for scientific research, and Article 8 with Recital 38 regarding use of children’s personal data.
These were not included due to additional requirements and complexity regarding their interpretation and representation using semantic web as well as lack of legal guidance on their compliance requirements.

GConsent is primarily based on the notion of consent as a legal basis under Article 6 of the GDPR. The conditions defined in Article 7, Recital 42, and Recital 43 provide the requirements for consent to considered or determined valid as per the requirements of GDPR.
The Data Controller bears the burden of demonstrating proof and satisfaction of requirements for consent to be considered valid as per Recital 42. This requires demonstrable proof that the data subject provided the consent and that it was valid as per the obligations specified in the GDPR.

For consent to be informed it is necessary to provide information to the data subject which includes the specific purposes of processing the personal data.
GDPR also provides data subjects with the right to modify or withdraw consent as defined in Article 7-3.
When consent is withdrawn the processing carried out done prior to the withdrawal is considered valid under the given consent which was in effect as the legal basis during that period of time.

Through this, a rudimentary summary of information associated with consent includes:
\begin{itemize}
    \item Data Subject the consent is about
    \item Personal Data associated with consent
    \item Processing operations the consent is about
    \item Purposes the consent is about
    \item Entity/Agent/Actor the consent is provided to
    \item Specific operations, including data storage and data sharing
    \item Recipients of data if any
\end{itemize}
These attributes are sufficient to provide a simplified representation of consent, and are used in existing approaches within the state of the art - such as the model of consent in SPECIAL vocabularies. However, these attributes are not sufficient by themselves to determine the validity of given consent, and lack information about the context the consent was given in, as well as tracking its state across changes. Further attributes associated with consent include:
\begin{itemize}
    \item Entity/Agent/Actor that provided the consent - relevant in the case of delegation
    \item Status of consent at a given period in time
    \item Contextual information regarding request for consent such as - location, medium, timestamp, expiry
    \item Contextual information regarding giving of consent such as - location, medium, timestamp, expiry
\end{itemize}
In addition to these, the provenance of how consent was requested and obtained is also important.
Specifically, the information about the specific process and artefacts used in the provision of request for consent - which must satisfy the GDPR qualitative requirements such as the request being clearly stated and being umabigious.

To identify the application of these attributes in various scenarios within the real world, 15 categories of use-cases were collected to determine the necessary information required to demonstrate valid consent or state of consent, and identify their ontological representations.
These also guided the development of the ontology by serving as test scenarios where GConsent could be applied.
A summary of these use-cases is as follow:
\begin{enumerate}
% \tightlist
\item
  Obtaining / Declaring Consent (its state)

  \begin{enumerate}
%   \tightlist
  \item
    The consent is given
  \item
    Consent was given, but is now invalidated (by the controller)
  \item
    Consent was given, but was withdrawn (by the Data Subject)
  \item
    Consent was requested (by the controller)
  \item
    Consent was requested, but was refused (by the Data Subject)
  \item
    Consent state is unknown (e.g. when importing data about consent)
  \end{enumerate}
\item
  Entity the consent is about

  \begin{enumerate}
%   \tightlist
  \item
    The consent is about a Data Subject who is not a minor
  \item
    The consent is about a Data Subject who is a minor
  \end{enumerate}
\item
  Activity for Data Subject

  \begin{enumerate}
%   \tightlist
  \item
    There was an age verification process associated with the consent
    (such as for minors)
  \item
    There was an identity verification process associated with the
    consent
  \end{enumerate}
\item
  Entity that provided consent

  \begin{enumerate}
%   \tightlist
  \item
    Consent was provided by the Data Subject it is about
  \item
    Consent was not provided by the Data Subject it is about, but was
    provided by a Delegation

    \begin{enumerate}
    % \tightlist
    \item
      Consent in the Delegation was provided by another Data Subject
    \item
      Consent in the Delegation was provided by a Person
    \item
      Consent in the Delegatiton was provided by another Delegation
    \end{enumerate}
  \end{enumerate}
\item
  Role within Delegation

  \begin{enumerate}
%   \tightlist
  \item
    Entity is the Parent/Guardian of the Data Subject
  \item
    Entity is a third-party to the Data Subject
  \end{enumerate}
\item
  Activity of Delegation

  \begin{enumerate}
%   \tightlist
  \item
    There was some verification process to assert the authentication of
    the delegation
  \end{enumerate}
\item
  Personal Data associated with consent

  \begin{enumerate}
%   \tightlist
  \item
    Consent was given for specific instances of personal data
  \item
    Consent was given for categories of personal data (note: not
    instances)
  \end{enumerate}
\item
  Medium of Consent

  \begin{enumerate}
%   \tightlist
  \item
    consent is given via a web-form
  \item
    consent is given as a signed paper document
  \item
    consent is given as a verbal confirmation
  \item
    consent is given implicitly in some form (medium)
  \item
    consent is given via delegation in some form (medium)
  \end{enumerate}
\item
  Activity responsible for consent

  \begin{enumerate}
%   \tightlist
  \item
    Activity created consent as a new entity
  \item
    Activity modified existing consent
  \end{enumerate}
\item
  Previous consent and relationship

  \begin{enumerate}
%   \tightlist
  \item
    Consent has no previous instance
  \item
    Consent has a previous instance, it replaces it
  \end{enumerate}
\item
  Differences between consent instances

  \begin{enumerate}
%   \tightlist
  \item
    Something changes between two consent instances (e.g. personal data
    category is added)
  \end{enumerate}
\item
  Time constraints

  \begin{enumerate}
%   \tightlist
  \item
    consent expires (has a tangible expiry such as a specific date or
    duration)
  \item
    consent does not expire (is valid for ``as long as required'')
  \end{enumerate}
\item
  Third party Association

  \begin{enumerate}
%   \tightlist
  \item
    Personal Data is collected from a third party
  \item
    Personal Data is stored with a third party (processor)
  \item
    Personal Data is shared with a third party
  \item
    Processing involves third party
  \item
    Purpose involves third party
  \end{enumerate}
\item
  Role of Third Party

  \begin{enumerate}
%   \tightlist
  \item
    Third Party is a Processor contracted by the Controller
  \item
    Third Party is another Controller
  \item
    Third Party is another entity (regulatory/supervisory/governmental)
  \end{enumerate}
\item
  Storage Duration and Locations for Personal Data

  \begin{enumerate}
%   \tightlist
  \item
    Data is stored for a fixed time (specific instance or duration)
  \item
    Data is stored for an indefinte duration (``for as long as
    required'')
  \end{enumerate}
\end{enumerate}
Based on these use-cases, a set of competency questions were developed from the compliance questions presented in \autoref{} pertaining to given consent (\textit{CMQ-CMQ}) and change in consent state (\textit{CMQ-CMQ}).
These were further categorised as questions about consent itself, questions about how the consent was created/given/changed/invalidated, questions about the context of how consent was created/given/invalidated, and questions related to third parties associated with the consent.

\subsection{Ontology Description \& Application}
\subsubsection{Core Concepts}
The core concepts and relationships in GConsent describe the common and primary attributes associated with consent.
In this case, `consent' by itself does not refer to only the state of `given consent' but also stands as a representation of consent whose state is unknown or is refused, withdrawn, or invalidated by the Data Subject, Controller, or an authority such as the court. This definition of consent is based on managing consent as a data entity rather than as a semantic concept reflecting the data subjects choices. 
The core concepts associated with consent in all its state refer to the information necessary to express what the consent is about, which comprises of the 5 attributes visualised in \autoref{fig:vocabs:gconsent-core} - Data Subject, Personal Data, Purpose, Processing, and Status.
\begin{figure}[htbp]
    \centering
    \includegraphics[width=0.8\linewidth]{img/gconsent_core.png}
    \caption{Core concepts in GConsent \cite{}}
    \label{fig:vocabs:gconsent-core}
\end{figure}

The \textit{DataSubject} is the person the consent is associated with as an agreement of their choices. This person may or may not be the same entity that gave the consent, as in the case of parental or guardian giving consent for a child or the act of delegation. The \textit{DataSubject} class is defined as a subclass of \textit{prov:Person}, and has the subclass \textit{MinorDataSubject} to denote a data subject that is legally a minor or a child.
Each instance of \textit{Consent} can only be associated with one and only one \textit{DataSubject}, and any further changes or modifications to the state of consent will continue to be associated with the same \textit{DataSubject}.
The \textit{PersonalData} is the set of personal data associated with the consent. Where multiple personal data are associated with a single instance of consent, it is interpreted to mean the union of these sets of personal data. Similarly, \textit{Purpose} and \textit{Processing} are also to be interpreted as union rather than intersection.
The `status' or `state' of consent indicates the suitability of using that specific instance of consent as a legal basis for the processing of personal data defined by the consent attributes.

\textit{Purpose} and \textit{Processing} are concepts that have semantic meaning based upon their use within the GDPR.
`Processing' is defined by Article 4-2, while `Purpose' has no specific definition provided but can be summarised as the intent or aim of why the set of personal data is needed or to be used for. In practice, purpose is generally defined at a higher abstract level, and often encompasses several types or categories of data. An example of this is a privacy policy specifying `account information' and `location of service use' - which are data categories, and that are `collected' and `used' - which are processing operations on the personal data, `to ensure security of the account' - which is the purpose the personal data will be processed for. The relation between a purpose and its associated processing operations is quite opaque. It is difficult to determine which processing operations a purpose entails and vice versa, and their usage may not always be implied or commonly understood. Therefore, GConsent provides purpose and processing as self-declarative high-level concepts which can be extended with additional information.

% To faciliate this, GConsent provides several instances of useful status types divided across two subclass to indicate the valid and invalid states. Additional categories can be defined by extending any of the classes as required.

\subsubsection{Context of Consent}
The context of consent refers to the environmental attributes such as the location or time when the instance of consent was created, invalidated, generated, changed, modified, given, or recorded.
GConsent provides concepts for location, medium, and timestamp to indicate instant of creation or invalidation along with capturing the `expiry' of consent as either a instant of time or a duration using the Time vocabulary \cite{}. The context also represents how consent was `provided' by a Person or Data Subject or Delegation. 
The provided contexts in GConsent are visualised in \autoref{fig:vocabs:gconsent-context}.

% A delegation is the transfer of responsiblity to provide consent to an entity other than the data subject the consent is about. This includes a parent or a guardian giving the consent in lieu of a minor. 
Context is associated with an instance of consent using the generic property \textit{hasContext}, with specialised properties extending it to indicate provision, expiry, location, time, and medium. 
Additional contexts can be represented and associated by extending \textit{hasContext} in a similar manner.
\begin{figure}[htbp]
    \centering
    \includegraphics[width=0.8\linewidth]{img/gconsent_context.png}
    \caption{Concepts for representing context of consent in GConsent \cite{}}
    \label{fig:vocabs:gconsent-context}
\end{figure}

\subsubsection{Consent States}
The state of consent determines the suitability of its usage as a legal basis in the processing of personal data.
From a compliance perspective, there are only two categories of states - one which permits legal processing of personal data, and one which is not sufficient or prohibits such processing.
GConsent represents these using the concepts by sub-classing \textit{Consent} as \textit{StatusValidForProcessing} and \textit{StatusInvalidForProcessing} to indicate use of an consent instance as a valid or invalid legal basis, as depicted in \autoref{fig:vocabs:gconsent-status}.
Instances provided to represent states of valid consent to indicate legal processing, and include - explicitly given, implicitly given, and given by delegation.
Instances provided that represent invalid states of consent to indicate processing should not be carried out include - unknown, not given, withdrawn, expired, invalidated, refused, and requested.

The use of state refers to tracking the consent of a data subject from the legal perspective, and is aimed to aid in the management of consent as an entity. For example, `unknown' reflects a situation where the status of consent is not known - which can occur when importing consent information data from another source.
This is distinct from `not given' which indicates an offer has been made for obtaining consent but the data subject has not yet provided any actionable response that could indicate acceptance or refusal - which are themselves represented by the states for `given' and `refused' respectively.
For meeting the obligations and requirements of GDPR compliance, it is not necessary to represent consent instances with states such as unknown or refused.
GConsent provides them based on the practical management of consent information where a Controller may wish to track the consent status of its processing operations throughout its lifecycle.
\begin{figure}[htbp]
    \centering
    \includegraphics[width=0.8\linewidth]{img/gconsent_status.png}
    \caption{Concepts representin state/status of consent in GConsent \cite{}}
    \label{fig:vocabs:gconsent-status}
\end{figure}

GDPR requires keeping track of state change for consent. For example, when the status changes from given to withdrawn or when the state if changed to invalidated because of the choice of the controller or some legal requirement. Therefore, whenever a consent status changes, this results in a new consent instance being created, which also assists in capturing the context of the new consent (such as time instant). Therefore, this leads to a chain of consent instances, where the "latest" consent is at the "end" of this chain, and will have the most recent timestamp of creation. It is vital to record such provenance to demonstrate past processing was in compliance with the state of consent at that point in time.

\subsubsection{Example Application}
The documentation of GConsent provides example applications of the ontology in four use-cases to demonstrate  how information can be represented. The examples provided include (i) change in consent state, (ii) capturing given consent, (iii) capturing consent given via delegation, and (iv) capturing consent when data is shared with a third party.
The fourth example is presented here to explain the application of GConsent and the use of its concepts to represent information towards GDPR compliance.

The example, as visually represented in \autoref{fig:vocabs:gconsent-example}, shows association of a third party in the role of a data processor\footnote{Under GDPR, a processor is not considered a third party, but has its own defined role as an entity associated with the Controller. However, from a lay person's perspective, the individual is the first party, the Controller is the second party, and any other entity is a third party. GConsent reflects this use in its structuring of entities where a Processor is considered a special type of Third Party.}
with whom the data is shared for the purpose of advertising. The association is captured by the instance \textit{ex:AdvertisingArrangement} of type \textit{prov:Association}, and has \textit{ex:AdPartner} defined as a \textit{gdprov:Processor} defined with the role as \textit{gdprtext:Processor}. It is also possible to list out the specific arrangement for this association using the \textit{prov:hadPlan} property and a \textit{gdprov:Process} instance to list out the specific steps and entities involved in the data sharing arrangement.

The example serves to demonstrate the practical use of GConsent in representing information about consent, where PROV-O is used to specify relationships with a Processor. GConsent can be combined or supplemented by other ontologies to define such associations and practical reflections of data sharing agreements between parties. The defined instance of consent in the example enables a Controller to track the state of consent as the data subject is provided with the choice of whether to agree to this arrangement or to refuse it, and where upon agreement the option to exercise the right to modify and withdraw the consent is also provided.
\begin{figure}[htbp]
    \centering
    \includegraphics[width=0.8\linewidth]{img/gconsent_third_party_datasharing.png}
    \caption{Example application of GConsent in use-case about third party data sharing \cite{}}
    \label{fig:vocabs:gconsent-example}
\end{figure}

\subsection{Evaluation}
GConsent as an ontology was evaluated regarding its capability to express information about consent using the methodology outlined in \autoref{sec:voc:methodology}.
This was an iterative process where the ontology was tested and modified to accommodate the requirements of the competency questions. Changes were made to the ontology where information was found to be missing or incorrectly modelled.

GConsent was published in the Extended Semantic Web Conference as a peer-reviewed publication \cite{} which described the motivation, methodology, and creation of the ontology. Along with this, the documentation of GConsent, available online, also provides extensive information about the ontology and its potential applications. It also provides a brief comparison of the ontology with relevant approaches within the state of the art.

Compared to existing work regarding consent in the state of the art presented in \autoref{chapter:sota}, GConsent provides novel contributions regarding the representation of consent for GDPR compliance.
In particular, the depiction of delegation and consent states are important gaps filled by GConsent.
Other than these, GConsent also received input through a semi-formal consultation with a legal expert who is a law professor at Trinity College Dublin, further providing some valuable input from a domain-expert's perspective.

\subsection*{Summary}
GConsent is an ontology for the representation of consent and its associated information for GDPR compliance.
It is currently the only approach within the state of the art to provide representations of attributes of consent and its states as required by the GDPR.
GConsent has been published in a peer-reviewed publication, and is available online as an open and reusable resource along with an extensive and descriptive documentation.
It fulfils the research objective $RO3(c)$ and forms the third major contribution of this thesis.

There is currently an ongoing effort to combine the existing standard of Consent Receipt \cite{} with the representation of consent for GDPR as offered by GConsent. The aim of this work is the creation of a GDPR-version of Consent Receipt, which is to then be further extended to create an international version of Consent Receipt that would be applicable to the emerging privacy laws across the globe. 
% * ??? to mention ??? consultaition with research Mastercard on semantic representation of consent

% DPV
\section{Data Privacy Vocabulary (DPV)}\label{sec:voc:DPV}
\subsection{Relevance of DPV to this thesis}
The Data Protection Vocabulary is a semantic web ontology for representing information about personal data handling based on the framework provided by requirements for GDPR compliance.
It is the outcome of the work done by the W3C Data Privacy Vocabularies and Controls Community Group (DPVCG) which provides collaboration between a community of academics, researchers, industry stakeholders, and legal experts as initially described in \autoref{}.
It aims to work towards establishment of \textit{interoperable standards} regarding representing information about personal data processing, for which there is no suitable standard present.

The DPV reflects a community consensus in its representation of information regarding data protection and personal data processing. While being a generic vocabulary, much of its design is based on and reflected by the requirements of the GDPR. The DPV, and by extension the DPVCG, reflect an ongoing effort to provide practical and useful semantic representations of information in an open and interoperable machine-readable form. 

The ontologies presented in this thesis as research contributions - namely GDPRtEXT, GDPRov, and GConsent - were part of the state of the art analysed by the DPVCG in its methodology \cite{}.
In addition, by being an active member of the DPVCG and a contributor the creation of the DPV, the author of this thesis has applied the experience of developing presented research to influence the design and modelling of information within the DPVCG.

The DPVCG currently has published two peer-reviewed publications, where the first \cite{} was published at the initiation of the community group and described the goal of the CG along its plans and aims. The second \cite{} publication presents the DPV along with its creation methodology and concepts, where the author of this thesis was a co-first author.
In addition to these, the vocabulary specification lists the author of this thesis as an co-editor and author of the ontology.

The DPVCG was initiated as an outcome of the SPECIAL project \cite{}, and therefore bears close association and alignment with the SPECIAL vocabularies.
In particular, the SPECIAL core vocabulary was used as the basis to create the DPV core vocabulary, providing the basis for alignment and compatibility.
The deliverable D6.5 \cite{} of the SPECIAL project presents the work of the DPVCG by describing the DPV.
It is based on the publication of DPV \cite{} and features the author of this thesis as one of the lead authors for the deliverable.

This section describes the DPV and compares it with the ontologies presented in this thesis.
It demonstrates the differences in representation of information, scope, and methodology of the two approaches of work, and their complimentary nature in addressing GDPR.
\subsection{Overview of DPV}
\subsubsection{Description of Data Privacy Vocabulary}
The DPV ontology is published at the W3C namespace \url{http://w3.org/ns/dpv} with its documentation and uses the namespace prefix \textit{dpv}. 
The current iteration of vocabulary provides classes and properties to annotate and categorise information about legally compliant personal data handling. In this context, personal data handling refers to all operations associated with the processing of personal data and its management - which includes organisational measures which do not directly affect the data or its processing.

The DPV is a pseudo-modular ontology, with a set of core concepts referred to as the `Base Ontology', and modular extensions further expanding each concept in the form of a taxonomy. 
The core concepts, visualised in \autoref{fig:vocabs:dpv-core}, consist of personal data category, processing, purpose, legal basis, data controller, recipient, data subject, technical and organisational measures, and the top-level concept of personal data handling which ties them together.
The base ontology represents the top-level classes for defining a policy of legal personal data handling. Each of its concepts are further elaborated using sub-vocabularies within the DPV.
\begin{figure}[htbp]
    \centering
    \includegraphics[width=\linewidth]{img/dpv-personaldatahandling.png}
    \caption{Core concepts in DPV \cite{}}
    \label{fig:vocabs:dpv-core}
\end{figure}

\subsubsection{Personal Data Categories}
DPV uses the taxonomy provided by EnterPrivacy~\cite{enterprivacy} to define a broad hierarchy of personal data categories based on the nature of information (financial, social, tracking) and to its inherent source (internal, external). 
In addition to these, the class \texttt{dpv:Special\-Category\-Of\-PersonalData} represents categories that are `special' or `sensitive' based on GDPR’s Article 9.

These personal data categories can be further extended by using the sub-class mechanism to depict specialised concepts such as `likes regarding movies'.
Sub-classing also enables representation of specific contexts such as derivation of personal data as represented by the class \texttt{dpv:DerivedPersonalData}.
This is useful to represent practical representation of personal data categories such as inference of opinions from social media.
Similar classes can be additionally added to specify contexts such as use of machine learning, accuracy, and source.
The aim of the providing such high-level concepts is to provide a sufficient coverage of abstract categories of personal data which can be extended using the subclass mechanism to represent concepts used in the real-world. 

\subsubsection{Purposes}
Purposes in DPV are organised hierarchically by using the sub-class mechanism to represent high-level and generic purposes of data handling.
Purposes provided in DPV include service provision, R\&D, commercial interest, security, service optimisation, and service personalisation. These are further extended to provide a total of 31 generic purposes.
These may be extended by further subclassing to create more specific purposes as applicable to a scenario.
As GDPR requires a specific purpose to be declared in an understandable manner, specific purposes can be created using subclasses of one or several \texttt{dpv:Purpose} categories to make them as specific to the use-case as possible.
 Purposes can be restricted to specific \emph{contexts} using the class \texttt{dpv:Context} and the property \texttt{dpv:hasContext}.
Purposes can also be restricted to a specific \emph{business sector} using the class \texttt{dpv:Sector} and the property \texttt{dpv:hasSector}.

\subsubsection{Processing Categories}
DPV provides a hierarchy of processing categories based on the requirements of regulations such as GDPR. 
DPV defines top-level classes to represent the following broad categories of processing - Disclose, Copy, Obtain, Remove, Store, Transfer, Transform, and Use.
Each of these are then again further expanded using subclasses to provide 33 processing categories, which includes terms defined in the definition of processing in GDPR (Article 4-2).
The DPV provides properties with a boolean range to indicate the nature of processing regarding \emph{Systematic Monitoring}, \emph{Evaluation or Scoring}, \emph{Automated Decision-Making}, \emph{Matching or Combining}, \emph{Large Scale processing}, and \emph{Innovative use of new solutions} - as these affect assessment of legal data processing under GDPR.

\subsubsection{Technical and Organisational Measures}
GDPR Article 32 requires implementing appropriate measures by taking into account the state of the art, the costs of implementation and the nature, scope, context and purposes of processing, as well as risks, rights and freedoms.
These are represented as technical and organisational measures in the DPV.
Examples include pseudonymisation and encryption of personal data, the ability to restore the availability and access to personal data in a timely manner in the event of a physical or technical incident, and a process for regularly testing, assessing and evaluating the effectiveness of technical and organisational measures for ensuring the security of the processing.
The generic property \texttt{dpv:measureImplementedBy} enables referencing the implementation measure as a comment or IRI.
The class \emph{StorageRestriction} provides expression of measures used for storage of data with two specific properties provided for storage location and duration restrictions.

\subsubsection{Consent and Legal Bases}
DPV provides \texttt{dpv:LegalBasis} as a top-level concept to represent various legal bases that can be used for justifying processing of personal data.
The definition of a `legal basis' is based on the justification for processing which has a provision in the law. The concept itself is not based on any specific jurisdiction, but needs to be interpreted in terms of the legal bases defined and provided within the laws applicable within a jurisdiction.

For the GDPR, which is a EU specific law, and therefore is not binding in the interpretation of legal bases across all use-cases, the DPV provides the legal bases specific to GDPR as a separate aligned vocabulary under the \url{https://www.w3.org/ns/dpv-gdpr} and namespace (prefix: \texttt{dpv-gdpr:}). 
This vocabulary defines the legal bases defined by Articles 6 and 9 of the GDPR to represent the legal justification of processing personal data.

Consent as a special case of legal basis provided by the GDPR is provided with additional properties and classes within the core DPV vocabulary to reflect the information requirements associated with its validity as a legal basis.
The concepts associated with consent provide terms to describe consent provision, withdrawal, and expiry.
The structure of these was adapted from an analysis of existing work in the form of Consent Receipt \cite{lizar_consent_2017} and GConsent \cite{pandit_gconsent_2019} with the intention to enable documenting attributes associated with consent which can demonstrate and evaluate its validity based on requirements of the GDPR.

\subsection{Comparison of GDPRtEXT, GDPRov, and GConsent with and DPV}

\section{Comparing developed ontologies with SotA}

\subsection*{Summary}

\section{Chapter Summary}