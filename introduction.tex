\chapter{Introduction}
\label{chapter:introduction}

\section{Background \& Motivation}\label{sec:intro:background}
% privacy laws across the world
% disconnect with technological progress
To date, 132 of the 206 states listed by the United Nations (UN) have a privacy law which regulates the usage of personal data \cite{greenleaf_global_2019}.
However, their intended application suffers from a disconnect with the rapid progresses in technology. In particular, the use of internet as a medium for data exchange and its pervasiveness and connectivity to individuals via devices such as the smartphone has led to industrial data harvesting at large scales \cite{christl_networks_2016}. 
To counter this problem, lawmakers in the European Union (EU) passed the General Data Protection Regulation (GDPR) \cite{noauthor_regulation_2016} in 2016 with the aim of providing individuals with the right to information and control over use of their personal data, and to simplify requirements for organisations through a unified regulation across the EU.

The GDPR has received a large amount of attention due to its prospective fines which can be up to 4\% of an organisation's annual turnover or €20 million - whichever is greater. 
To date, there have been over 70 publicly known instances of fines associated with the GDPR \cite{noauthor_gdpr_nodate}, the largest of which was the disruptive €50 million fine to internet giant and Android developer Google \cite{noauthor_cnils_2019}.
Being a regulation and replacing the Data Protection Directive (DPD) \cite{noauthor_directive_1995}, GDPR provides a uniform set of compliance requirements across the EU, and is the basis of national privacy laws implemented in its member states \cite{mccullagh_national_2019}.
Furthermore, GDPR has influenced other privacy laws, such as the California Consumer Protection Act (CCPA) \cite{noauthor_assembly_2018}, thereby further expanding similarities in compliance requirements across the globe.

The most visible change of the GDPR for most individuals is the almost ubiquitous `consent dialogue' on websites, required for obtaining `consent' - one of the legal basis for processing of personal data in GDPR.
Despite being a legal requirement, the dialogue is notorious for being difficult to use effectively and subverting the spirit of the GDPR \cite{machuletz_multiple_2019,utz_informed_2019}.
The issue of consent itself has further received significant interest in terms of technological solutions developed for its compliance due to the right to withdraw consent afforded by the GDPR, which enables an individual to revoke their previously given consent, after which the processing of personal data based on it must be halted.
While there is no known public knowledge of practices regarding consent being found unlawful, opinion provided by legal authoritative has expressed the need for greater transparency regarding the activities associated with acquiring consent \cite{noauthor_opinion_2019}.

% information associated with gdpr compliance
Compared to other privacy laws, including its predecessor DPD, GDPR provides significantly stricter and verbose requirements for the processing of personal data and requires organisations to explicitly document information specific to its obligations in order to be compliant.
This information consists of identification of which clauses are applicable to the specific use-case of an organisation, and the steps taken to achieve compliance with their requirements and obligations.
From a technical or informational viewpoint, GDPR specifies interactions between entities in a clear manner, for example - a controller using the legal basis of consent is required to provide certain information about processing activities to the data subject, or how the controller needs to inform supervisory authorities of data breach. Furthermore, this information is also necessary to be maintained, evaluated, and documented in order to demonstrate compliance to a Supervisory Authority as part of an investigation. At the same time, this information is also associated with other stakeholders - such as through privacy policies, user agreements, terms and conditions, or even controller - processor agreements. This shows that information associated with GDPR compliance is also useful or used in other applications and involves several stakeholders.

As GDPR is a data protection law, its compliance is concerned primarily with information associated with the processing of personal data, its legality, and associated operations within an organisation. 
This includes processing in both tenses - past as well as future - where an organisation is obligated to first determine its requirements and activities involving processing of personal data will be valid under the GDPR, and to then maintain a record of such activities as the processing takes place.
These are defined in the legal domain by the terms `ex-ante' to specify compliance assessment before activity takes place (preventative) and `ex-post' to specify compliance assessment after the activity has taken place (corroborative).

The combination of new requirements and significant fines has provided an incentive to utilise technology in meeting the obligations and requirements stipulated by GDPR towards its compliance.
Existing efforts, such as the International Organization for Standardization (ISO), have addressed this change by updating standards to meet increased requirements with global privacy laws.
In the context of GDPR, ISO/IEC 27001 defines requirements for an information security management system, and its extension ISO/IEC 27701 defines a privacy information management system, which together provide a framework for managing privacy risks associated with personal data processing.
Adherence to such standards provides a commonality in the information management practices of an organisation, and assists in the compliance process by providing a structured interpretation of practices based on the standard specification, which is demonstrated through certification with the standard.

% challenges in developing technological solutions
Technological development of solutions for legal compliance face two problems in general: the first being algorithmic interpretation of the requirements associated with legal compliance - which is difficult as the text used in a legal document such as the GDPR does not readily lead to algorithmic compliance due to ambiguity and uncertainty in its legal interpretation, especially in domain specific use-cases.
In addition, because GDPR has been enforced for a comparatively short period, the interpretation of its clauses in terms of requirements and compliance relies on clarification through legal opinions and decisions by supervisory authorities and courts.
The second problem is that regardless of how technology is used in the compliance process, formal investigations of legal compliance require information to be documented and associated with the specifics of the law they intend to comply with - in this case the specific articles and clauses of GDPR.
Traditionally, this is carried out through creation of documentation by legal experts, lawyers, and legal departments.
Therefore, technological solutions addressing GDPR compliance should also incorporate information documentation in addition to assessment of compliance. 

% problem with existing text based approaches
Incorporating legal compliance into a requirements of a business has led to several approaches, which include use of symbolic (mathematical) logic, knowledge representation of legal text as logical rules, deontic rights specifying rights and obligations, defeasible logic based on exceptions, first order temporal logic, access control, markup based representations, and goal modelling of obligations \cite{otto_addressing_2007}.
While there has been significant work in the use of technology to adopt these approaches towards addressing and evaluating compliance in the last decade \cite{sadiq_modeling_2007,otto_addressing_2007,gordon_rules_2009,fellmann_state---art_2014,benyoucef_information_2015,elgammal_formalizing_2016,kirrane_access_2016}, the issue of associating information with legal documents has received relatively lesser focus.
While contemporary methods are sufficient to meet legal requirements, their use of text-based document formats prevents effective technological solutions that can be scaled, automated, or utilised in an information management system. To enable such approaches, information must be represented using machine-readable formats which can enable the use of querying to retrieve information as well as validation methods to check its correctness. Furthermore, the need to share information between stakeholders leads to the necessity of it being interoperable, which is also important towards transparency in the compliance process. By using open and interoperable standards, the commonality in representation and interpretation of information benefits stakeholders and reduces costs associated with innovation in the legal community.

% linked data principles and ELI, Akoma Ntoso
Governmental agencies across the globe have addressed this issue by adopting the principles of Linked Open Data \cite{bizer_linked_2011}, and have produced interoperable standards \cite{palmirani_akoma_2018,european_union_eli_2015,van_opijnen_european_2011} which facilitate use of information in technological solutions.
In addition to Linked Data, these standards also implement the principles of the Semantic Web \cite{noauthor_semantic_nodate} by utilising the Resource Description Framework (RDF) \cite{noauthor_rdf_2014} to specify information in an interoperable and machine-readable manner.
This has paved the way for development of technologies that address challenges associated with legal compliance through greater use of automation and scale.
Consequently, the use of Linked Data and Semantic Web within the legal domain has resulted in the development of ontologies along the dimensions of organising and structuring information, reasoning and problem solving, semantic indexing and search, semantic integration and interoperability, and understanding the domain \cite{rodrigues_legal_2019}.
Semantic Web is also being used to address the challenges associated with GDPR compliance through commercial solutions as well as large-scale European research projects such as SPECIAL \cite{noauthor_scalable_nodate}, MIREL \cite{noauthor_mirel_nodate}, DAPRECO \cite{noauthor_data_2018-1}, BPR4GDPR \cite{noauthor_bpr4gdpr_nodate}, RestAssured \cite{noauthor_restassured_nodate}.
The technological solutions developed within these utilise ontologies to represent the information required for compliance, and a corresponding approach that expresses and evaluates obligations for compliance.

%%% link between GDPR compliance and research question???
The work presented in this thesis concerns utilising the Semantic Web to address GDPR compliance by representing activities involved in the processing of personal data in both ex-ante and ex-post manner.
This includes activities associated with acquiring consent as well owing to the role of consent as a legal basis and the assertion that consent itself is also personal data.
In particular, the focus of the thesis is on representing information relevant for GDPR compliance in a manner that can be associated with the text of the GDPR following Linked Data Principles.
It uses existing standards such as Resource Description Framework (RDF) \cite{noauthor_rdf_2014} and Web Ontology Language (OWL2) \cite{noauthor_owl_2012} to represent information as ontologies, SPARQL \cite{noauthor_sparql_nodate} for querying information, and Shapes Constraint Language (SHACL) \cite{knublauch_shapes_2017} to validate information.
The use of semantic web standards and technologies enables the information to be persisted in a machine-readable, interoperable, and queryable form - which readily lends itself to automation using technological solutions in the areas of legal compliance and its documentation.

% Research Scope 
In terms of scope, the work presented in this thesis addresses only the representation and management of information associated with GDPR compliance, and is not intended to provide an authoritative assessment of  compliance as the only entities with legal authority to do so are supervisory authorities and courts.
In the same vein, the research presented in this thesis is also not intended to replace professional opinions such as that offered by lawyers and legal experts.
Instead, the intention of the work is to demonstrate the applicability and feasibility of using technology as a tool to assist with the compliance process.

\section{Research Question}\label{sec:intro:RQ}
% The aim of the thesis is to enable representation of information associated with ex-ante and ex-post activities involving processing of personal data and consent by using semantic web technologies for GDPR compliance.
The research question investigated in this thesis is:
\begin{framed}
\small{Research Question}
\begin{quote}
\textbf{To what extent can information regarding activities associated with processing of personal data and consent be represented and associated with the GDPR using Semantic Web technologies with for compliance?}
\end{quote}
\end{framed}

\subsection{Definitions}\label{sec:intro:definitions}
The following definitions are used in the context of the research question outlined above and this thesis:
\begin{itemize}
    \item \textit{information regarding activities}: information about how processes, services, tasks, or other similar concepts were planned, executed or carried out, resulting outcomes, and the artefacts used or required
    \item \textit{activities associated with processing of personal data}: information about how personal data will be or has been obtained (its source), its usage - including storage, sharing, analysis, or other forms of processing 
    \item \textit{activities associated with consent}: information about consent will be or has been obtained, its usage as a legal basis, the information represented by consent, and its planned or recorded withdrawal
    \item \textit{querying}: retrieving information using a structured representation based on the underlying representation of information
    \item \textit{validation}: assessment of information to meet a constraint or requirement
    \item \textit{associate or link information with GDPR}: to establish an association or link between information and clauses or concepts of the text of GDPR
    \item \textit{subset of GDPR}: a subset of the clauses defined in the text of the GDPR
    \item \textit{ex-ante compliance}: compliance regarding processing before it has taken place, i.e. \textit{A priori}
    \item \textit{ex-post compliance}: compliance regarding processing after it has taken place, i.e. \textit{A posteriori}
    \item \textit{compliance questions}: questions that retrieve information relevant for determination of compliance
\end{itemize}

\subsection{Research Objectives}\label{sec:intro:RO}
% RO1
The GDPR is a legal document structured into 173 Recitals, 99 Articles, and 21 Citations. Of these, not all parts are relevant to provenance information of personal data and consent. Therefore, the first research sub-question concerns investigation and identification of the sub-set of GDPR regarding activities associated with personal data and consent, along with information on the ex-ante and ex-post aspects of such activities towards compliance.
\begin{framed}
$RO1$: Identify the subset of GDPR relevant for activities associated with personal data and consent regarding ex-ante and ex-post compliance.
\end{framed}

% RO2
Following the identification of a sub-set of relevance, information required to represent the activities is obtained using guided questions that identify the actors, entities, and relationships for compliance with the GDPR.
\begin{framed}
$RO2$: Identify information required to represent activities associated with personal data and consent towards investigations of GDPR compliance.
\end{framed}

% RO3
The identified information is then represented using semantic web ontologies in the form of concepts and relationships. This representation acts as the information model upon which questions or queries are then  executed in an automated fashion to obtain the required information for determining compliance. The formalisation of information as an ontology also provides a controlled vocabulary for validation of information to determine its sufficiency and correctness before determining compliance. 

Rather than assimilating all information within a singular ontology, good practice dictates creation of modular ontologies specific to a particular task of domain. The information requirements can thus be divided into three distinct areas, each of which correspond to a specific domain, and lead towards the creation of an ontology within it. The first sub-objective therefore concerns creating an ontology to associate information with the concepts and clauses within the text of the GDPR. The second and third objectives then utilise this ontology to represent activities associated with personal data and consent respectively.
\begin{framed}
$RO3$: Create OWL2 ontologies for expressing information about:
\newline\indent\indent\textbf{(a)}: concepts and text of the GDPR
\newline\indent\indent\textbf{(b)}: activities associated with personal data
\newline\indent\indent\textbf{(c)}: activities associated with consent
\end{framed}

% RO4
'Compliance questions' retrieve the relevant information required to determine compliance, and are important in the documentation process. These can be executed in an automated fashion by expressing the information using developed ontologies, and utilising SPARQL queries to represent compliance questions using corresponding concepts and relationships from the ontologies.
\begin{framed}
$RO4$: Represent compliance questions using SPARQL to query information about activities associated with personal data and consent
\end{framed}

% RO5
The determination of compliance not only includes assessment whether a given information satisfies all the obligations and requirements, but also involves validation of the information itself in terms of correctness and completeness.
In software engineering processes, such assessments are automated in the form of tests that validate data and produce a report to record documentation.

The same principle can be utilised to assess a given information for correctness and completeness based on requirements of the GDPR.
This is done using SHACL which enables expressing validation requirements over developed ontologies and produces a report which can be persisted and linked back to the GDPR for documentation of compliance.
\begin{framed}
$RO5$: Utilise SHACL to:
\newline\indent\indent\textbf{(a)}: validate information for GDPR compliance regarding activities associated with personal data and consent
\newline\indent\indent\textbf{(b)}: link validation results with GDPR
\end{framed}

\section{Research Methodology}\label{sec:intro:research-methodology}
% state of the art
\subsection{Reviewing the State of the Art}
A review of the state of the art (SotA) was conducted at several stages of the research given the evolving nature of approaches towards GDPR compliance.
The publications associated with the research objectives were driving factors in providing motivation to conduct a SotA review to capture the approaches and progress at that particular time.
In addition, a general review of legal models for compliance was also conducted to identify relevant approaches which could be reused towards addressing the requirements of the GDPR.

The review included understanding of the GDPR, which was obtained and studied from sources including the official text of the GDPR, its interpretation and clarification provided by authoritative bodies such as Data Protection Commissions, Article 29 Working Party (A29WP), and European Data Protection Board (EPDB).
In addition, guides and expert opinions provided by legal experts and organisations were also utilised to better understand the requirements for GDPR compliance.
Information requirements associated with compliance were based on these, as well as studying case law related to interpretation of the GDPR where accessible.

Approaches associated with GDPR compliance were reviewed where information was openly available, such as through academic publications and project deliverables.
Where such information was not accessible, such as in commercial products and some resources in academic projects, the available information was used to compile and review the SotA.
In particular, the review focused on the use of semantic web technologies, and the extent of their applicability towards addressing the requirements of the GDPR.

Publications and resources were discovered through Google Scholar, Scopus, IEEExplore, ACM Digital Library, as well as through events such as conferences and events, and by following relevant profiles on Twitter.
Zotero was used as a tool for managing references and related notes.

% construction of ontologies
\subsection{Ontology Engineering}\label{sec:intro:ontology-engineering}
The development of ontologies required for research objective $RO3$ was done using accepted methodologies within the semantic web community, which included a general guide for creating ontologies \cite{noy_ontology_2001}, the NEON methodology for ontology engineering \cite{suarez-figueroa_neon_2012}, and UPON Lite for agile ontology development \cite{de_nicola_lightweight_2016}.

The general methodology used for ontology engineering for the ontologies presented in this thesis can be summarised as follows:
\begin{enumerate}
	\item Gather questions related to information associated with GDPR compliance - with a particular focus on activities related to personal data and consent. These were used as competency questions for ontology development. In addition, each question was associated with relevant clauses of the GDPR.
	\item Identify concepts and relationships to represent the information required to answer each competency questions. Concepts and relationships from addressing all competency questions were consolidated and analysed to identify additional relationships and remove inconsistencies between them.
	\item The identified concepts and relationships were represented as an ontology by using OWL2. Simultaneously, relevant prior work regarding existing ontologies was identified to represent the concepts and relationships, and utilised where applicable.
	\item The ontology was assessed in terms of effectiveness to represent the information for answering competency questions, and limitations were identified. These steps was repeated to address limitations in a cyclic process until sufficient capability was achieved.
    \item Incorporate new information regarding GDPR compliance periodically by constructing additional competency questions and requirements, and identifying suitability of developed ontologies in addressing them.
\end{enumerate}

The ontology was documented with metadata based on best practices advocated by the community\footnote{\url{https://dgarijo.github.io/Widoco/doc/bestPractices/index-en.html}}, which enabled automatic generation of documentation using the WIDOCO tool \cite{garijo_widoco:_2017}.
The namespace IRI was defined with persistent identifiers through the use of W3ID\footnote{\url{http://w3id.org/}}.
The ontology itself was archived in the public open repository Zenodo\footnote{\url{https://zenodo.org/}}, which provided it with versioned DOIs.
Finally, the ontology and related resources were hosted in a SPARQL end-point on Trinity College Dublin servers.

Methodologies associated with other GDPR related ontologies, such as within the SPECIAL, MIREL, and DAPRECO projects, include the participation of legal experts who guide the interpretation of GDPR and its compliance.
In comparison, the methodology in thesis is focused on representation of information rather than interpretation of compliance, which reduces the limitations associated with lack of participation from legal experts,
though the research developed in this thesis was informally presented and discussed with legal experts in the area of data protection.

% querying and validation framework
\subsection{Information Validation Framework for GDPR Compliance}
For the information validation framework associated with research objective $RO5$, the methodology consisted of compiling constraints and assumptions from questions associated with GDPR compliance, and defining them using SHACL to assess data based on ex-ante and ex-post measures.
In general, the methodology followed the following steps:
\begin{enumerate}
	\item Gather questions related to information associated with GDPR compliance questions with a particular focus on activities associated with personal data and consent, and associate them with relevant clauses of the GDPR. These contained questions reused from Section \ref{sec:intro:ontology-engineering}.
	\item Identify constraints and assumptions for each question
    \item Identify concepts and relationships from developed ontologies required for satisfying constraint
    \item Identify constraints as applicable to ex-ante and ex-post phases
    \item Identify constraints common to ex-ante and ex-post phases
    \item Express constraints using SHACL and persist results for three cases:
    \begin{enumerate}
        \item for ex-ante phase
        \item for ex-post phase
        \item for common constraints in ex-ante phase, and reuse results with instance specific constraints for ex-post phase
    \end{enumerate}
    \item Express questions about compliance by executing queries using SPARQL over persisted validation results 
\end{enumerate}

% evaluation strategy
\subsection{Evaluation Methodology}\label{sec:intro:evaluation}
\subsubsection{Evaluating Ontologies}
%% ontologies
The developed ontologies were assessed in their sufficiency and completeness to answer the competency questions. 
In addition, rudimentary use-cases related to compliance were also used to assess the ontology in terms of sufficient representation of related information. These use-cases were compiled from GDPR-related case law, SotA, and synthetic situations.
Ontologies were also evaluated using best practices advocated by the community throughout its cycle of development by using OOPS! \cite{poveda-villalon_oops!_2014} to detect common pitfalls in the ontology design.
While most detected pitfalls were corrected in the ontology engineering process, some pitfalls were found to not be relevant or applicable and were ignored.
Finally, the ontology was compared against similar ontologies identified in the SotA to analyse its novelty, as well as strengths and weaknesses.

\subsubsection{Evaluating Information Validation Framework}
%% querying and validation framework
The framework developed for validating information utilises constraints derived from compliance questions.
Its evaluation consisted of generating a synthetic use-case based on the consent mechanism of a real-world website, where the constraints were validated against information related to consent and personal data activities within the use-case.
The use-case enabled representation of activities in both ex-ante and ex-post phases where ex-ante represented determining validity of the consent dialogue being presented, and ex-post represented determining validity of given consent.

The information regarding activities related to personal data and consent within the use-case was represented using GDPRov and GConsent to create a semantic web representation of the use-case.
SHACL was then used to define constraints derived from competency questions over the use-case data, with links to GDPR added to constraints using custom properties.

The evaluation consisted of determining suitability of SHACL and developed ontologies to express the constraints, and the ability to link the constraints and its validation results with the relevant clauses of the GDPR.
This was utilised to further determine the extent to which information could be queried regarding actionable tasks for compliance and validation results associated with clauses of GDPR, based on the data declared within the synthetic use-case.
Furthermore, the evaluation also analysed and demonstrated the advantage of combining ex-ante and ex-post approaches based on number of constraints required for validation of the use-case.
The framework was also compared against other approaches within the SotA to assess its novelty and utilisation towards addressing GDPR compliance.

\section{Contributions of this Thesis}\label{sec:intro:contributions}
The two major contributions of this thesis are, first - enabling linking of information with the text of GDPR , and second - creation of ontologies representing information about activities associated with personal data and consent (based on ontologies in $RO3$). Minor contributions include formulating an information model of entities and their relationships in GDPR (based on information in $RO1$ and $RO2$), and using semantic web technologies for querying and validating information required for compliance (based on approaches from $RO4$ and $RO5$). Resources associated with the contributions\footnote{\url{http://openscience.adaptcentre.ie/res/}}, including published papers\footnote{\url{https://openscience.adaptcentre.ie/publications/}}, have been made accessible under open licenses (MIT,  CC-by-4.0) for reproducibility, and to foster adoption and re-use by the community.

\subsection{GDPR as a Linked Data Resource}
The first major contribution of this thesis is the GDPRtEXT resource - which provides a linked data version of the text of the GDPR and a vocabulary of its concepts, and fulfils research objectives $RO3(a)$ and $RO5(b)$. By exposing each individual article or point within the text of the GDPR as a unique resource using semantic web, GDPRtEXT makes it possible for links to be established between information and the text of the GDPR. As these links are machine-readable, they can be used in approaches that automate the generation and querying of information associated with GDPR - such as for compliance, management of business processes, or generation of privacy policies. Furthermore, GDPRtEXT offers compatibility with the official documents produced by the European Publications Office by using ELI as its ontological base.

It is currently a common practice to refer to concepts within legal documents such as the GDPR by associating them with the relevant clause within the document. 
GDPRtEXT provides a thesauri or vocabulary of concepts defined or referred to within the GDPR to assist with use of concepts defined or mentioned within the GDPR. Each concept or term is associated with its definition or articles of relevance within the GDPR by using the linked data version of text provided by GDPRtEXT. This provides another way to link information to the GDPR through the use of concepts, and has been used to indicate the source in definitions of terms and relationships for vocabularies (see Section \ref{sec:contributions:ontologies}).

GDPRtEXT fills an important gap in the state of the art (as investigated in Chapter \ref{chapter:sota}) by providing a mechanism to link information with the text of the GDPR in a machine-readable manner, and it therefore the only vocabulary with its terms associated with their definition and usage in GDPR .
While there are other comparable and relevant methods to address such information \cite{agarwal_legislative_2018,palmirani_pronto:_2018-1}, GDPRtEXT is currently the only one  that uses and extends ELI \cite{noauthor_council_2012} which is the official metadata standard for European legislation documents, and is also the only openly accessible ontology regarding GDPR and its concepts \cite{leone_taking_2019}.

GDPRtEXT has been released\footnote{\url{https://w3id.org/GDPRtEXT}} under the open license (CC-by-4.0) and has been incorporated into Ireland's open data portal\footnote{\url{https://data.gov.ie/dataset/gdprtext}}.
The provision of machine-readable concepts and reference to clauses of the GDPR makes GDPRtEXT an important resource for use in legal knowledge graphs.

\subsection{Ontologies for representing activities about Personal Data and Consent}\label{sec:contributions:ontologies}
The second major contribution of this thesis are the two semantic web ontologies for representing information about activities associated with personal data and consent. Both ontologies define concepts and relationships using GDPRtEXT to associate their source within the GDPR.

The first of these is GDPRov, which enables representation of the processes and activities associated with the lifecycle of personal data and consent, and  fulfils the research objectives $RO3(b)$ and $RO3(c)$.
GDPRov extends PROV-O \cite{lebo_prov-o:_2013} - which is the W3C standard for defining provenance information - to define ex-post (activity logs indicating things that have happened) information, and P-Plan \cite{garijo_p-plan_2014} to define ex-ante (as an abstract model, template, or plan) representations of PROV activities based on scientific workflows.

The state of art contains ontologies for representing activities and their provenance related to the GDPR \cite{pasquier_data_2018,palmirani_pronto:_2018-1}, including those utilising PROV \cite{belhajjame_provenance_2018,bonatti_special_2018-1}, and holistic approaches combining ex-ante and ex-post compliance \cite{dullaert_d3.4_2019}.
In comparison, GDPRov provides the most exhaustive vocabulary of concepts based on the GDPR (based on comparisons demonstrated in Chapter \ref{chapter:vocabularies}), and is the only ontology to provide ex-ante and ex-post concepts within the same ontology.
GDPRov thus advances the state of the art by providing a larger vocabulary for modelling and representing activities based on GDPR concepts.

GDPRov represents the given consent as an artefact that is used and produced by activities concerned with its lifecycle. The determination of consent validity under the GDPR requires additional information \cite{politou_forgetting_2018,article_29_data_protection_working_party_guidelines_2018}, which is not captured by the scope of activities represented by GDPRov. Therefore, a separate modular ontology, termed GConsent, extends the research objective $RO3(c)$ and provides the necessary concepts and relationships to represent information for management and evaluating compliance of consent as governed by the obligations and requirements of the GDPR.

GConsent provides the necessary concepts and relationships to express information about consent in terms of entities such as individuals or agents, purposes and processing, involvement of third parties, medium and context of provision, relationship between instances (e.g. withdraws, updates), and the novel concept of `consent states' which enable the management of consent as an entity. 
In comparison with the state of the art, GConsent provides greater representation of information related to consent as compared to other approaches related to consent and data management for GDPR \cite{peras_guidelines_2018}, and is the most comprehensive ontology for representing consent (based on comparisons demonstrated in Chapter \ref{chapter:vocabularies}).

Together with GDPRtEXT, GDPRov and GConsent enable the representation of activities required to evaluate and validate compliance with the relevant articles of the GDPR. Apart from advancing the state of the art, the ontologies also provide a vocabulary of terms and concepts useful to adopters, and demonstrate the use of legal documents as a source for ontologies using linked data principles.
Their usefulness has also been demonstrated in approaches other than compliance - such as an ontological representation of privacy policies \cite{pandit_ontology_2018}, generation of privacy policies from metadata \cite{pandit_personalised_2018}, and automating evaluation of changes in activities \cite{pandit_gdpr-driven_2018}.
GDPRov\footnote{\url{https://w3id.org/GDPRov}} and GConsent\footnote{\url{https://w3id.org/GConsent}} are available under the open license (CC-by-4.0).

\subsection{Information Model of the GDPR}
A minor contribution of this thesis is an information model of GDPR representing entities and their interactions with respect to interoperability of information.
The model conceptualises the information and interactions between stakeholders based on information identified as part of $RO1$ and $RO2$, and provides an overview of requirements shaped by the GDPR.
This provides the categorisation of information requirements based on provenance, agreements, consent, certification, and compliance; and assists in the exploration of existing standards, including semantic web, by outlining the requirements and applications of information based on its interoperability between the entities.
The information model advances the state of the art by providing the first systemic analysis of information flows and interoperability between stakeholders, and serves to identify and evaluate potential applications of information and technology for stakeholders.

\subsection{Querying Information Related to Compliance}\label{sec:contributions:querying}
Another minor contribution of this thesis is the utilisation of semantic web technologies to query and validate information for GDPR compliance, which fulfils research objective $RO4$.
The use of developed ontologies, namely - GDPRtEXT, GDPRov, and GConsent - enable the use of SPARQL to represent compliance questions derived from state of the art (see Chapter \ref{chapter:information}) as queries capable of retrieving information relevant for assessment of compliance (see Chapter \ref{chapter:testing}).
Where approaches in the state of the art also use SPARQL to represent questions for compliance \cite{agarwal_legislative_2018,palmirani_pronto:_2018}, the work presented in this thesis derives such questions from templates provided by a supervisory authority for assisting organisations with the compliance readiness \cite{noauthor_gdpr_2017}.
The use of SPARQL thus assists the investigation process associated with compliance, and fulfils the research objective $RO4$.

\subsection{Validating Information for Compliance}\label{sec:contributions:validation}
The third minor contribution of this thesis is the utilisation of SHACL to validate information and link the results with the relevant clauses of the GDPR for compliance, which fulfils research objective $RO5$.
The approach uses constraints and assumptions derived from compliance questions (see Chapter \ref{chapter:information}) to create validation tests, which are represented by SHACL, and are associated with relevant clauses of the GDPR using GDPRtEXT (see Chapter \ref{chapter:testing}). The tests utilise concepts and relationships from GDPRov and GConsent to represent validation requirements, and re-use SPARQL queries created through $RO4$ to retrieve information.
The validation results are persisted and annotated with GDPRtEXT to link them with the GDPR, thereby providing a form of documentation for information validation associated with compliance.
Furthermore, the approach demonstrates how a combination of ex-ante and ex-post tests are more efficient by abstracting common constraints and validating them at the ex-ante stage, while only the specific constraints associated with individuals and provenance are validated at the ex-post stage based on ex-ante validation results.

Related work in the state of the art use a variety of approaches in the validation and assessment of compliance. The SPECIAL project has investigated the use of OWL2 reasoners to validate consent at ex-ante and ex-post stages \cite{bonatti_fast_2018,dullaert_d3.4_2019}, the application of ODRL policies as a compliance checking mechanism \cite{agarwal_legislative_2018,vos_odrl_2019}. The MIREL project has proposed the use of deontic logic for legal reasoning through the use of LegalRuleML \cite{palmirani_pronto:_2018,monica_modelling_2018}, while the BPR4GDPR project proposes checking provenance logs for conformance to predetermined processes (ex-post analysis) \cite{,mehr_compliance_2019}.
Compared to the state of the art, the approach presented in this thesis is novel in its utilisation of SHACL to validate information and link its results with the GDPR for compliance.
A similar use of SHACL utilising P-Plan workflows to validate policies expressed in ODRL for GDPR compliance has been proposed \cite{lieber_policy-compliant_2019}, which adds to the applicability of contributions of this thesis. 

\subsection{Participation in DPVCG}\label{sec:intro:dpvcg}
The Data Privacy Vocabularies and Controls Community Group\footnote{\url{https://www.w3.org/community/dpvcg/}} (DPVCG) is a W3C community group working towards developing a vocabulary associated with personal data processing based on relevant laws such as GDPR.
The group was created by members of the SPECIAL project, and currently consists of community members from diverse domains such as academia, legal experts, lawyers, and industry stakeholders.

Currently in operation for over 18 months, the work done within the DPVCG has produced the Data Privacy Vocabulary\footnote{\url{http://w3.org/ns/dpv}} (DPV) , an ontological resource for the representation of information associated with processing of data.
The DPV provides a community agreement in the form of a vocabulary of terms and concepts associated with the GDPR, and enables interoperability of compliance-related information.
The work conducted in the creation of DPV has been published in a peer-reviewed conference \cite{pandit_dpv_2019}, and has also been listed as a deliverable within the SPECIAL project \cite{pandit_d6.5_2019} - in both of which the author of this thesis is listed as an editor.

The research presented in this thesis had an impact in the creation of DPV through the use of developed ontologies as an input as well as through direct participation of the author as an active contributing member.
Comparing them both, the DPV provides a high-level abstraction of terms and concepts, whereas the ontologies in this thesis provide a more verbose model for the representation of information - which makes their usage with the DPV complimentary rather than contradictory.
Furthermore, the thesis provides a mapping between them to demonstrate similarities and potential for combined usage (see Chapter \ref{chapter:vocabularies}).

\subsection{Publications}\label{sec:intro:publications}
The following peer-reviewed publications present the research in this thesis:

\subsubsection{GDPR Ontologies}
The following publications are associated with $RO3$ - developing ontologies for representing the concepts and relationships within the GDPR.
\begin{enumerate}[start]
    \item \textbf{\bibentry{pandit_gdprtext_2018}}
        \newline
        This publication presents the GDPRtEXT resource consisting of a linked data representation of the text of GDPR, and a thesauri of its concepts. It also provides a mapping from clauses of the DPD to GDPR based on applicability of access control methods developed for DPD towards GDPR compliance. GDPRtEXT fulfils research objective $RO3(a)$, and is instrumental in providing semantic association between information and the GDPR for the approaches presented in this thesis.
    \item \textbf{\bibentry{pandit_modelling_2017}}
        \newline
        This publication presents the GDPRov ontology for representing the provenance of personal data and consent for GDPR, and also discusses the use of its concepts in SPARQL queries for retrieving information associated with compliance. GDPRov fulfils research objective $RO3(b)$, and provides ex-ante and ex-post representations for activities associated with personal data and consent for GDPR.
    \item \textbf{\bibentry{fatema_compliance_2017}}
        \newline
        This publication presents an early collaboration (pre-GDPR) for developing a preliminary ontology for representing consent, and a data management model using access control for GDPR. The early work was crucial towards understanding the complexities of consent, and provided valuable feedback towards the development of GConsent.
    \item \textbf{\bibentry{pandit_gconsent_2019}}
        \newline
        This publication presents the GConsent ontology for representing information about consent as required by the GDPR. GConsent fulfils research objective $RO3(c)$, and provides a verbose representation of consent for information management and documentation.
    \item \textbf{\bibentry{hadziselimovic_linked_2017}}
        \newline
        This publication presents an early collaboration (pre-GDPR) towards developing an ontology for representing data sharing agreements by extending the ODRL ontology based on GDPR. The ontology - called Data Protection Rights Language (DPRL) - enables representation of obligations associated with propagation of rights between parties that share or exchange data.
\end{enumerate}

\subsubsection{Querying and Evaluating Information for Compliance}
The following publications are associated with $RO4$ - querying for information, and $RO5$ - validating information for compliance.
\begin{enumerate}[resume]
    \item \textbf{\bibentry{pandit_queryable_2018}}
        \newline
        This publication presents the use of SPARQL queries to represent questions associated with compliance by using the GDPRtEXT and GDPRov ontologies.
        It demonstrates effectiveness of SPARQL in retrieving information for GDPR compliance, and fulfils research objective $RO4$.
    \item \textbf{\bibentry{pandit_test-driven_2019}} (in-press)
        \newline
        This publication presents an overview of the approach for validating information using SHACL and associating the results with specific articles of the GDPR. The approach proposes persistence of validation results to create a `compliance graph' that can itself by queried and validated for documenting information for compliance.
    \item \textbf{\bibentry{pandit_exploring_2018}}
        \newline
        This publication presents the implementation of the approach described above for validation of information by utilising the use-case of consent mechanism in a real-world website. It utilises SHACL to validate information represented by GDPRov and GConsent, and uses GDPRtEXT to associate the tests and their results with the GDPR. It also demonstrates the use of SPARQL to identify tasks and reports related to compliance by querying the validation results. The approach demonstrates the usefulness of combining ex-ante and ex-post approaches in terms of efficiency and compliance, and fulfils research objective $RO5$.
\end{enumerate}

\subsubsection{Model for Information Interoperability in GDPR}
These publications present a model of interaction between entities as defined by the GDPR, and explore information categories and their interoperability requirements based on existing standards, including those provided by the semantic web.
The model provides an overview of information flows between stakeholders, and the role of interoperability in facilitating information for compliance between them.
\begin{enumerate}[resume]
    \item \textbf{\bibentry{pandit_gdpr_2018}}
    \item \textbf{\bibentry{pandit_exploration_2018}}
\end{enumerate}

\subsubsection{Investigated Applications of Research - Information Management}
The following publications do not address the research question, but consist of applying the research presented in this thesis towards processes that assist with being compliant.
\begin{enumerate}[resume]
    \item \textbf{\bibentry{pandit_gdpr-driven_2018}}
    \newline This publication proposes an approach for detecting changes related to use of personal data and consent in activities by utilising the ex-ante component of GDPRov to represent activities and comparing them using a graph-based algorithm.
    \item \textbf{\bibentry{pandit_towards_2018}}
    \newline This publication explores the creation of a knowledge-based framework based on the utilisation of information associated with compliance using semantic web technologies, which can be utilised in applications such as creation of reports, documentation, and assessment of compliance for different stakeholders.
    \item \textbf{\bibentry{debruyne_towards_2019}}
    \newline This publication presents an approach for generating just-in-time datasets consisting of personal data based on given consent to ensure processes are compliant in their usage of consent as the legal basis.
\end{enumerate}

\subsubsection{Investigated Applications of Research - Privacy Policies}
The following publications do not address the research question, but consist of applying the research presented in this thesis towards privacy policies.
\begin{enumerate}[resume]
    \item \textbf{\bibentry{pandit_extracting_2018}}
    \newline This publication discusses the use of GDPRov to represent extracted extracting information about activities associated with personal data from within a privacy policy.
    \item \textbf{\bibentry{pandit_ontology_2018}}
    \newline This publication presents an ontology design pattern that uses GDPRov and GDPRtEXT to represent information about personal data and its processing in a privacy policy.
    \item \textbf{\bibentry{pandit_personalised_2018}}
    \newline This publication discusses the personalisation of privacy policies by using information about an individual's personal data processing to generate the text of the policy, and using GDPRtEXT and GDPRov to annotate it for a machine-readable representation.
\end{enumerate}

\subsubsection{Data Privacy Vocabulary}
The following publication presents the work related to the creation of the Data Privacy Vocabulary by the DPVCG, and describes the methodology used as well as relation to the existing vocabulary, including those presented in this thesis - namely GDPRtEXT, GDPRov, and GConsent.
\begin{enumerate}[resume]
    \item \textbf{\bibentry{pandit_creating_2019}}
\end{enumerate}

\section{Thesis Overview}
The rest of this thesis is structured as follows:

\subsubsection{Chapter 2: Background}
This chapter presents a summary of information necessary to understand the work presented in this thesis. The chapter consists of two sections, the first describes the concepts and requirements of the GDPR with particular focus on information required for compliance. The second section describes the semantic web technologies through an overview of standards and vocabularies, and the mechanisms used for querying and validation of information.

\subsubsection{Chapter 3: State of the Art}
This chapter reviews existing work and approaches regarding regulatory compliance with a specific focus on those addressing GDPR compliance. The chapter provides an overview of the different approaches in terms of technologies and methods, with an in-depth review of those utilising semantic web technologies to address GDPR compliance requirements. These are divided into three categories: first - regarding linking of information with the GDPR, second - vocabularies for representing information for GDPR, and third - those addressing evaluation of information for compliance. The chapter concludes with a summary of existing work and identification of gaps in the state of the art.

\subsubsection{Chapter 4: Provenance Information for GDPR Compliance}
This chapter describes information about provenance of personal data and consent through the use of questions whose answers retrieve the information necessary to address compliance with the GDPR. The questions are termed as `compliance queries' and are used to create a list of assumptions and constraints that the information must satisfy in order to be valid. The chapter also presents  an information model of entities and their interactions based on the GDPR, and the role of information flows in shaping interoperability requirements regarding GDPR compliance.

\subsubsection{Chapter 5: Ontologies for GDPR}
This chapter presents the OWL2-DL ontologies developed to represent information associated with the compliance queries presented in Chapter 4. The first ontology presented is GDPRtEXT, which provides a method to link information with the GDPR by providing a linked data version of its text and a vocabulary of concepts. The second ontology presented is GDPRov, which enables representation of provenance information regarding personal data and consent in the form of models or templates and their executions or activity logs. The third ontology presented is GConsent, which enables representation of information associated with consent. The chapter presents an overview of the concepts and relationships for each vocabulary, its relation with the GDPR, and how the vocabulary uses the compliance queries as competency questions to guide its development.

\subsubsection{Chapter 6: Querying and Validation of Information}
This chapter presents the use of SPARQL to express the compliance queries using the ontologies presented in Chapter 5. The chapter also presents a mechanism to validate information based on the constraints identified in Chapter 4 through the use of SHACL. The chapter further demonstrates how this approach enables persistence of information and documentation regarding compliance through an use-case based on a real-world scenario for a website's consent mechanism.

\subsubsection{Chapter 7: Conclusion}
This chapter concludes the thesis with a summary of key findings and outcomes of the presented work. It discusses the extent to which the thesis serves to address the research question(s) and objective(s), and outlines directions for future work in terms of potential applications and extension through related work.
