\chapter{Introduction}
\label{chapter:introduction}

\section{Background \& Motivation}\label{sec:intro:background}
% privacy laws across the world
% disconnect with technological progress
To date, 132 of the 206 states listed by the United Nations (UN) have a privacy law which regulates the usage of personal data \cite{greenleaf_global_2019-1}.
However, their intended application suffers from a disconnect with the rapid progress in technology. In particular, the use of internet as a medium for data exchange and its pervasiveness and connectivity to individuals via devices such as the smartphone has led to industrial data harvesting at large scales \cite{christl_networks_2016}. 
To counter this problem, lawmakers in the European Union (EU) passed the General Data Protection Regulation (GDPR) \cite{Regulation_GDPR} in 2016 with the aim of providing individuals with the right to information and control over use of their personal data, and to simplify requirements for organisations through a unified regulation across the EU.

The GDPR has received a large amount of attention due to its prospective fines which can be up to 4\% of an organisation's annual turnover or €20 million - whichever is greater. 
To date, there have been over 70 publicly known instances of fines associated with the GDPR \cite{GDPR_fines_tracker}, the largest of which was the disruptive €50 million fine to internet giant and Android developer Google \cite{CNIL_GOOGLE_2019}.
Being a regulation and replacing the Data Protection Directive (DPD) \cite{directive_DPD}, GDPR provides a uniform set of compliance requirements across the EU, and is the basis of national privacy laws implemented in its member states \cite{mccullagh_national_2019}.
Furthermore, GDPR has influenced other privacy laws, such as the California Consumer Protection Act (CCPA) \cite{CCPA}, thereby further expanding similarities in compliance requirements across the globe.

The most visible change of the GDPR for most individuals is the almost ubiquitous `consent dialogue' on websites, required for obtaining `consent' - one of the legal basis for processing of personal data in GDPR.
Despite being a legal requirement, the dialogue is notorious for being difficult to use effectively and subverting the spirit of the GDPR \cite{machuletz_multiple_2019,utz_informed_2019}.
The issue of consent itself has further received significant interest in terms of technological solutions developed for its compliance due to the right to withdraw consent afforded by the GDPR, which enables an individual to revoke their previously given consent, after which the processing of personal data based on it must be halted.
While there is no known public knowledge of practices regarding consent being found unlawful, opinion provided by legal authoritative has expressed the need for greater transparency regarding the activities associated with acquiring consent \cite{opinion_AG_2019}.

% information associated with gdpr compliance
Compared to other privacy laws, including its predecessor DPD, GDPR provides significantly stricter and verbose requirements for the processing of personal data and requires organisations to explicitly document information specific to its obligations in order to be compliant.
This information consists of identification of which clauses are applicable to the specific use-case of an organisation, and the steps taken to achieve compliance with their requirements and obligations.
From a technical or informational viewpoint, GDPR specifies interactions between entities in a clear manner, for example - a controller using the legal basis of consent is required to provide certain information about processing activities to the data subject, or how the controller needs to inform supervisory authorities of data breach. Furthermore, this information is also necessary to be maintained, evaluated, and documented in order to demonstrate compliance to a Supervisory Authority as part of an investigation. At the same time, this information is also associated with other stakeholders - such as through privacy policies, user agreements, terms and conditions, or even controller - processor agreements. This shows that information associated with GDPR compliance is also useful or used in other applications and involves several stakeholders.

As GDPR is a data protection law, its compliance is concerned primarily with information associated with the processing of personal data, its legality, and associated operations within an organisation. 
This includes processing in both tenses - past as well as future - where an organisation is obligated to first determine its requirements and activities involving processing of personal data will be valid under the GDPR, and to then maintain a record of such activities as the processing takes place.
These are defined in the legal domain by the terms `ex-ante' to specify compliance assessment before activity takes place (preventative) and `ex-post' to specify compliance assessment after the activity has taken place (corroborative).

The combination of new requirements and significant fines has provided an incentive to utilise technology in meeting the obligations and requirements stipulated by GDPR towards its compliance.
Existing efforts, such as the International Organization for Standardization (ISO), have addressed this change by updating standards to meet increased requirements with global privacy laws.
In the context of GDPR, ISO/IEC 27001 defines requirements for an information security management system, and its extension ISO/IEC 27701 defines a privacy information management system, which together provide a framework for managing privacy risks associated with personal data processing.
Adherence to such standards provides a commonality in the information management practices of an organisation, and assists in the compliance process by providing a structured interpretation of practices based on the standard specification, which is demonstrated through certification with the standard.

% challenges in developing technological solutions
Technological development of solutions for legal compliance face two problems in general: the first being algorithmic interpretation of the requirements associated with legal compliance - which is difficult as the text used in a legal document such as the GDPR does not readily lead to algorithmic compliance due to ambiguity and uncertainty in its legal interpretation, especially in domain specific use-cases.
In addition, because GDPR has been enforced for a comparatively short period, the interpretation of its clauses in terms of requirements and compliance relies on clarification through legal opinions and decisions by supervisory authorities and courts.
The second problem is that regardless of how technology is used in the compliance process, formal investigations of legal compliance require information to be documented and associated with the specifics of the law they intend to comply with - in this case the specific articles and clauses of GDPR.
Traditionally, this is carried out through creation of documentation by legal experts, lawyers, and legal departments.
Therefore, technological solutions addressing GDPR compliance should also incorporate information documentation in addition to assessment of compliance. 

% problem with existing text based approaches
Incorporating legal compliance into a requirements of a business has led to several approaches, which include use of symbolic (mathematical) logic, knowledge representation of legal text as logical rules, deontic rights specifying rights and obligations, defeasible logic based on exceptions, first order temporal logic, access control, markup based representations, and goal modelling of obligations \cite{otto_addressing_2007}.
While there has been significant work in the use of technology to adopt these approaches towards addressing and evaluating compliance in the last decade \cite{sadiq_modeling_2007,otto_addressing_2007,gordon_rules_2009,fellmann_state---art_2014,benyoucef_information_2015,elgammal_formalizing_2016,kirrane_access_2016}, the issue of associating information with legal documents has received relatively lesser focus.
While contemporary methods are sufficient to meet legal requirements, their use of text-based document formats prevents effective technological solutions that can be scaled, automated, or utilised in an information management system. To enable such approaches, information must be represented using machine-readable formats which can enable the use of querying to retrieve information as well as validation methods to check its correctness. Furthermore, the need to share information between stakeholders leads to the necessity of it being interoperable, which is also important towards transparency in the compliance process. By using open and interoperable standards, the commonality in representation and interpretation of information benefits stakeholders and reduces costs associated with innovation in the legal community.

% linked data principles and ELI, Akoma Ntoso
Governmental agencies across the globe have addressed this issue by adopting the principles of Linked Open Data \cite{bizer_linked_2011}, and have produced interoperable standards \cite{palmirani_akoma_2018,european_union_eli_2015,van_opijnen_european_2011} which facilitate use of information in technological solutions.
In addition to Linked Data, these standards also implement the principles of the Semantic Web \cite{semantic-web} by utilising the Resource Description Framework (RDF) \cite{RDF} to specify information in an interoperable and machine-readable manner.
This has paved the way for development of technologies that address challenges associated with legal compliance through greater use of automation and scale.
Consequently, the use of Linked Data and Semantic Web within the legal domain has resulted in the development of ontologies along the dimensions of organising and structuring information, reasoning and problem solving, semantic indexing and search, semantic integration and interoperability, and understanding the domain \cite{rodrigues_legal_2019}.
Semantic Web is also being used to address the challenges associated with GDPR compliance through commercial solutions as well as large-scale European research projects such as SPECIAL \cite{SPECIAL}, MIREL \cite{MIREL}, DAPRECO \cite{DAPRECO}, BPR4GDPR \cite{BPR4GDPR}, RestAssured \cite{RestAssured}.
The technological solutions developed within these utilise ontologies to represent the information required for compliance, and a corresponding approach that expresses and evaluates obligations for compliance.

%%% link between GDPR compliance and research question???
The work presented in this thesis concerns utilising the Semantic Web to address GDPR compliance by representing activities involved in the processing of personal data in both ex-ante and ex-post manner.
This includes activities associated with acquiring consent as well owing to the role of consent as a legal basis and the assertion that consent itself is also personal data.
In particular, the focus of the thesis is on representing information relevant for GDPR compliance in a manner that can be associated with the text of the GDPR following Linked Data Principles.
It uses existing standards such as Resource Description Framework (RDF) \cite{RDF} and Web Ontology Language (OWL2) \cite{OWL} to represent information as ontologies, SPARQL \cite{SPARQL} for querying information, and Shapes Constraint Language (SHACL) \cite{SHACL} to validate information.
The use of semantic web standards and technologies enables the information to be persisted in a machine-readable, interoperable, and queryable form - which readily lends itself to automation using technological solutions in the areas of legal compliance and its documentation.

% Research Scope 
In terms of scope, the work presented in this thesis addresses only the representation and management of information associated with GDPR compliance, and is not intended to provide an authoritative assessment of  compliance as the only entities with legal authority to do so are supervisory authorities and courts.
In the same vein, the research presented in this thesis is also not intended to replace professional opinions such as that offered by lawyers and legal experts.
Instead, the intention of the work is to demonstrate the applicability and feasibility of using technology as a tool to assist with the compliance process.

\section{Research Question}\label{sec:intro:RQ}
% The aim of the thesis is to enable representation of information associated with ex-ante and ex-post activities involving processing of personal data and consent by using semantic web technologies for GDPR compliance.
The research question investigated in this thesis is:
\begin{framed}
\small{Research Question}
\begin{quote}
\textbf{To what extent can information regarding activities associated with processing of personal data and consent be represented using Semantic Web technologies for GDPR compliance?}
\end{quote}
\end{framed}

\subsection{Definitions}\label{sec:intro:definitions}
The following definitions are used in the context of the research question outlined above and this thesis:
\begin{itemize}
    \item \textit{information regarding activities}: information about how processes, services, tasks, or other similar concepts were planned, executed or carried out, resulting outcomes, and the artefacts used or required
    \item \textit{activities associated with processing of personal data}: information about how personal data will be or has been obtained (its source), its usage - including storage, sharing, analysis, or other forms of processing 
    \item \textit{activities associated with consent}: information about consent will be or has been obtained, its usage as a legal basis, the information represented by consent, and its planned or recorded withdrawal
    \item \textit{querying}: retrieving information using a structured representation based on the underlying representation of information
    \item \textit{validation}: assessment of information to meet a constraint or requirement
    \item \textit{associate or link information with GDPR}: to establish an association or link between information and clauses or concepts of the text of GDPR
    \item \textit{subset of GDPR}: a subset of the clauses defined in the text of the GDPR
    \item \textit{ex-ante compliance}: compliance regarding processing before it has taken place, i.e. \textit{A priori}
    \item \textit{ex-post compliance}: compliance regarding processing after it has taken place, i.e. \textit{A posteriori}
    \item \textit{compliance questions}: questions that retrieve information relevant for determination of compliance
\end{itemize}

\subsection{Research Objectives}\label{sec:intro:RO}
% RO1
The GDPR is a legal document structured into 173 Recitals, 99 Articles, and 21 Citations. Of these, not all parts are relevant to provenance information of personal data and consent. Therefore, the first research sub-question concerns investigation and identification of the sub-set of GDPR regarding activities associated with personal data and consent, along with information on the ex-ante and ex-post aspects of such activities towards compliance.
\begin{framed}
$RO1$: Identify the subset of GDPR relevant for activities associated with processing of personal data and consent regarding ex-ante compliance.
\end{framed}

% RO2
Following the identification of a sub-set of relevance, information required to represent the activities is obtained using guided questions that identify the actors, entities, and relationships for compliance with the GDPR.
\begin{framed}
$RO2$: Identify information required to represent activities associated with processing of personal data and consent towards investigations of GDPR compliance.
\end{framed}

% RO3
The identified information is then represented using semantic web ontologies in the form of concepts and relationships. This representation acts as the information model upon which questions or queries are then  executed in an automated fashion to obtain the required information for determining compliance. The formalisation of information as an ontology also provides a controlled vocabulary for validation of information to determine its sufficiency and correctness before determining compliance. 

Rather than assimilating all information within a singular ontology, good practice dictates creation of modular ontologies specific to a particular task of domain. The information requirements can thus be divided into three distinct areas, each of which correspond to a specific domain, and lead towards the creation of an ontology within it. The first sub-objective concerns creating an ontology to associate information with the concepts and clauses within the text of the GDPR. The second objective utilises this ontology to represent activities associated with processing of personal data and consent. The third objective provides additional information regarding consent as required to determine its compliance. The distinction between the second and third objectives is based on the requirement of information other than those associated with activities in the determination of compliance for consent.
\begin{framed}
$RO3$: Create OWL2 ontologies for representation information about:
\newline\indent\indent\textbf{(a)}: concepts and text of the GDPR
\newline\indent\indent\textbf{(b)}: activities associated with processing of personal data and consent
\newline\indent\indent\textbf{(c)}: consent required to determine compliance
\end{framed}

% RO4
'Compliance questions' retrieve the relevant information required to determine compliance, and are important in the documentation process. These can be executed in an automated fashion by expressing the information using developed ontologies, and utilising SPARQL queries to represent compliance questions using corresponding concepts and relationships from the ontologies.
\begin{framed}
$RO4$: Represent compliance questions using SPARQL to query information about activities associated with processing of personal data and consent
\end{framed}

% RO5
The determination of compliance not only includes assessment whether a given information satisfies all the obligations and requirements, but also involves validation of the information itself in terms of correctness and completeness.
In software engineering processes, such assessments are automated in the form of tests that validate data and produce a report to record documentation.

The same principle can be utilised to assess a given information for correctness and completeness based on requirements of the GDPR.
This is done using SHACL which enables expressing validation requirements over developed ontologies and produces a report which can be persisted and linked back to the GDPR for documentation of compliance.
\begin{framed}
$RO5$: Utilise SHACL to:
\newline\indent\indent\textbf{(a)}: validate information for GDPR compliance regarding activities associated with processing of personal data and consent
\newline\indent\indent\textbf{(b)}: link validation results with GDPR
\end{framed}

\section{Research Methodology}\label{sec:intro:research-methodology}
% state of the art
\subsection{Reviewing the State of the Art}
A review of the state of the art (SotA) was conducted at several stages of the research given the evolving nature of approaches towards GDPR compliance.
The publications associated with the research objectives were driving factors in providing motivation to conduct a SotA review to capture the approaches and progress at that particular time.
In addition, a general review of legal models for compliance was also conducted to identify relevant approaches which could be reused towards addressing the requirements of the GDPR.

The review included understanding of the GDPR, which was obtained and studied from sources including the official text of the GDPR, its interpretation and clarification provided by authoritative bodies such as Data Protection Commissions, Article 29 Working Party (A29WP), and European Data Protection Board (EDPB).
In addition, guides and expert opinions provided by legal experts and organisations were also utilised to better understand the requirements for GDPR compliance.
Information requirements associated with compliance were based on these, as well as studying case law related to interpretation of the GDPR where accessible.

Approaches associated with GDPR compliance were reviewed where information was openly available, such as through academic publications and project deliverables.
Where such information was not accessible, such as in commercial products and some resources in academic projects, the available information was used to compile and review the SotA.
In particular, the review focused on the use of semantic web technologies, and the extent of their applicability towards addressing the requirements of the GDPR.

Publications and resources were discovered through Google Scholar, Scopus, IEEExplore, ACM Digital Library, as well as through events such as conferences and events, and by following relevant profiles on Twitter.
Zotero was used as a tool for managing references and related notes.

\subsection{Information Gathering}
The gathering of information regarding requirements of GDPR and its compliance was done through a literature review of official and authoritative documentation published by legal bodies and organisations.
In order to understand the requirements of GDPR and the stakeholders involved, a model was developed to understand the requirements for interoperability of information for each stakeholder.
Further, based on the use of information about GDPR compliance in the ontology engineering process, the analysed information was used to create `compliance questions' which guide the ontology development process by acting as `competency questions' (see \autoref{sec:intro:ontology-engineering}) and also act as queries for retrieving information relevant to the compliance process. The questions also provided the basis for creating information validation constraints to evaluate information.
This process is described in \autoref{chapter:information}.

% construction of ontologies
\subsection{Ontology Engineering}\label{sec:intro:ontology-engineering}
The ontologies developed to fulfil research objective $RO3$ used commonly adopted methodologies within the semantic web community. These consisted of a general introductory guide for creating ontologies \cite{noy_ontology_2001} which was used to familiarise with the process of ontology engineering.
The actual construction of ontologies followed a combination of NeON methodology \cite{suarez-figueroa_neon_2012} and UPON Lite \cite{de_nicola_lightweight_2016}, where NeON was used to identify existing scenarios and gather requirements and UPON Lite was used to derive actionable steps or tasks to build and test the ontology using an agile development process.
The combination of the two provided a methodology for identifying relevant information from the GDPR (using NeOn) and iteratively building and updating an ontology to represent it (using UPON Lite).
The methodology for ontology engineering is described in more detail in \autoref{sec:voc:methodology}.
A summary of the methodology is as follows:
\begin{enumerate}
	\item Identification of aims, objectives, scope
    \item Identify and analyse relevant information
    \item Create use-cases and competency questions
    \item Identify concepts and relationships
    \item Create Ontology
    \item Evaluate
    \item Progressive iterations following steps 2 to 6
    \item Dissemination
\end{enumerate}

Each ontology was documented with metadata based on best practices advocated by the community\footnote{\url{https://dgarijo.github.io/Widoco/doc/bestPractices/index-en.html}}, which enabled automatic generation of documentation using the WIDOCO tool \cite{garijo_widoco_2017}.
The namespace IRI was defined with persistent identifiers through the use of W3ID\footnote{\url{http://w3id.org/}}.
The ontology itself was archived in the public open repository Zenodo\footnote{\url{https://zenodo.org/}} which provided it with versioned DOIs.
All code and resources associated with the ontologies were also published on an open and public code repository.
The ontology and related resources were hosted in a SPARQL end-point on Trinity College Dublin servers to enable their use via their IRIs on the internet.

% querying and validation framework
\subsection{Querying Information for GDPR Compliance}
The querying of information utilised SPARQL and fulfilled the research objective $RO4$.
The methodology to represent the compliance questions as SPARQL queries utilised questions from a real-world document published by the Irish Data Protection Commission for assisting organisations in evaluation their readiness for GDPR - which was also one of the sources for gathering information relevant for GDPR compliance.
The querying was demonstrated by representing each question within the document as a SPARQL query using the developed ontologies and executed over a synthetic use-case.

\subsection{Information Validation Framework for GDPR Compliance}
In order to demonstrate the validation of information, a modular framework was proposed in \autoref{sec:testing:shacl} which consisted of creating a separate `compliance graph' for storing information relevant to compliance.
This facilitated the querying and validation of information associated with compliance in a modular approach using SPARQL and SHACL respectively.
The constraints and assumptions created from constraint questions in \autoref{chapter:information} were represented using SHACL and used to validate information based on obligations and requirements of GDPR compliance.
Its application was demonstrated through a use-case evaluating the validity of consent on a real-world website.

% evaluation strategy
\subsection{Evaluation Methodology}\label{sec:intro:evaluation}
\color{blue}
\todo{review Evaluation Methodology overview table}
A summary of the evaluations methods used in the thesis is presented in \autoref{table:intro:evaluation-methods}
\begin{table}[htbp]
\color{blue}
\footnotesize
\centering
\caption{Summary of Evaluation Methods}\label{table:intro:evaluation-methods}
\rowcolors{1}{}{gray!10}
\begin{tabularx}{\textwidth}{|l|X|X|X|X|X|}
\hline
Method & GDPRtEXT Ontology & GDPRov Ontology & GConsent Ontology & Querying using SPARQL & Validation using SHACL \\ \hline
Fulfilment of Competency Questions & \cmark & \cmark & \cmark & N/A & N/A \\ \hline
Semantic reasoner logical consistency & \cmark & \cmark & \cmark & \cmark & \cmark \\ \hline
OOPS! common pitfalls detection & \cmark & \cmark & \cmark & N/A & N/A \\ \hline
Documentation metadata and quality & \cmark & \cmark & \cmark & N/A & N/A \\ \hline
Demonstrate application to use-case & \cmark & \cmark & \cmark & \cmark & \cmark \\ \hline
External use-case & \xmark & \cmark & \cmark & \cmark & \cmark \\ \hline
Comparison with SotA & \cmark & \cmark & \cmark & \cmark & \cmark \\ \hline
Analysis of citations & \cmark & \cmark & N/A & \cmark & N/A \\ \hline
Peer-reviewed publication & \cmark & \cmark & \cmark & \cmark & \cmark \\ \hline
Reproducibility (open access resources) & \cmark & \cmark & \cmark & \cmark & \cmark \\ \hline
\end{tabularx}
\end{table}
\color{black}

\subsubsection{Evaluating Ontologies}
%% ontologies
The developed ontologies presented in \autoref{chapter:vocabularies} were assessed in their sufficiency and completeness to answer the competency questions. 
In addition, use-cases related to situations differing in compliance requirements were used to assess the ontology in terms of sufficient representation of related information. These use-cases were compiled from GDPR-related case law, SotA, and synthetic situations, and validated regarding information requirements with a legal expert.
Ontologies were also evaluated using best practices advocated by the community throughout its cycle of development by using a semantic reasoner to ensure logical consistency in expressed facts and axioms, and by using the OOPS! \cite{poveda-villalon_oops!_2014} online service to detect common pitfalls in ontology design.
While most detected pitfalls were corrected in the ontology engineering process, some pitfalls were found to not be relevant or applicable and were ignored.
Finally, the sections in this thesis describing each ontology present a comparison against similar ontologies identified in the SotA to analyse their novelty, as well as strengths and weaknesses.
The sections also present the relevant peer-reviewed publications where the ontologies were presented and discussed. Citations to these publications are used as an indication to identify relevant approaches, and to investigate criticisms and comparisons with other ontological representations.

An extended evaluation of the ontologies was also performed indirectly based on their use in the querying and validation of information for research objectives $RO4$ and $RO5$. This provided the opportunity to evaluate whether the ontology provided sufficient concepts to represent the required information, and whether such representations could be used in the querying and validation process.
Where an ontology was found to lack concepts, these were added in an updated iteration or - in cases where there was ambiguity in representations - noted for future updates.

\subsubsection{Evaluating Querying of Information}
The use of SPARQL to query information based on compliance question as presented in \autoref{sec:testing:sparql} was evaluated through its application over a document published by the Irish Data Protection Commission.
The SPARQL queries utilised the developed ontologies to represent concepts in the compliance question, which provided an opportunity to evaluate the extent to which the ontology could represent these concepts.
The approach itself was evaluated based on the extent to which the questions in the document could be expressed as SPARQL queries.
Where a question could not or was not expressed using SPARQL, an analysis was carried out to determine its reason - such as the ontology lacking concepts or the question not being in scope of the research question.
The developed application and queries were presented in a peer-reviewed publication.

\subsubsection{Evaluating Information Validation Framework}
%% querying and validation framework
The framework developed for validating information using constraints derived from compliance questions is is presented with its evaluation in \autoref{sec:testing:shacl}.
The evaluation consisted of generating a synthetic use-case based on the consent mechanism of a real-world website, where the constraints were validated against information related to consent and personal data activities within the use-case.
The section also mentions the relevant peer-reviewed publications where the framework and its application were published and presented.

The use-case enabled representation of activities in both ex-ante and ex-post phases where ex-ante represented validity of the consent dialogue being presented, and ex-post represented determining validity of given consent.
The information regarding activities related to personal data and consent within the use-case was represented using the developed ontologies to create a semantic web representation of the use-case.
SHACL was then used to define constraints derived from competency questions over the use-case data, with links to GDPR added to constraints using custom properties.

The evaluation consisted of demonstrating the use of SHACL and developed ontologies to express the constraints, and the ability to link the constraints and its validation results with the relevant clauses of the GDPR.
The approach was also utilised to demonstrate the potential for the validation results to be represented as actionable tasks for compliance associated with clauses of GDPR.
The framework and the application were compared against approaches within the SotA to assess novelty towards the use of SPARQL And SHACL addressing GDPR compliance.

\section{Contributions of this Thesis}\label{sec:intro:contributions}
The two major contributions of this thesis are, first - enabling association of information with the text of GDPR following linked data principles, and second - ontologies for representing information about activities associated with processing of personal data and consent (based on ontologies in $RO3$). Minor contributions include formulating an information model of entities and their relationships in GDPR (based on information in $RO1$ and $RO2$), and using semantic web technologies for querying and validating information required for compliance (based on approaches from $RO4$ and $RO5$). Resources associated with the contributions\footnote{\url{http://openscience.adaptcentre.ie/res/}}, including published papers\footnote{\url{https://openscience.adaptcentre.ie/publications/}}, have been made accessible under open licenses (MIT,  CC-by-4.0) for reproducibility, and to foster adoption and re-use by the community.

\subsection{GDPR as a Linked Data Resource}
The first major contribution of this thesis is the GDPRtEXT resource - which provides a linked data version of the text of the GDPR and a vocabulary of its concepts, and fulfils research objectives $RO3(a)$ and $RO5(b)$. By exposing each individual article or point within the text of the GDPR as a unique resource using semantic web, GDPRtEXT makes it possible for links to be established between information and the text of the GDPR. As these links are machine-readable, they can be used in approaches that automate the generation and querying of information associated with GDPR - such as for compliance, management of business processes, or generation of privacy policies. Furthermore, GDPRtEXT extends and is compatible with the ELI ontology \cite{thomas_european_2019} used by the European Publications Office to publish documents - including GDPR. ELI currently only provides representations at the document level, while GDPRtEXT extends its concepts for representing clauses at a granular level. GDPRtEXT thus provides its features in an interoperable and compatible manner with ELI.

It is currently common practice to refer to concepts within legal documents such as the GDPR by associating them with the relevant clause within the document. 
GDPRtEXT provides a glossary or vocabulary of concepts defined or referred to within the GDPR to assist with use of concepts defined or mentioned within the GDPR. Each concept or term is associated with its definition or articles of relevance within the GDPR by using the linked data version of text provided by GDPRtEXT. This provides another way to link information to the GDPR through the use of concepts, and has been used to indicate the source in definitions of terms and relationships for vocabularies (see \autoref{sec:contributions:ontologies}).

GDPRtEXT fills an important gap in the state of the art (as investigated in \autoref{chapter:sota}) by providing a mechanism to link information with the text of the GDPR in a machine-readable manner, and it therefore the only vocabulary with its terms associated with their definition and usage in GDPR .
While there are other comparable and relevant methods to address such information \cite{agarwal_legislative_2018,palmirani_pronto_2018-1}, GDPRtEXT is currently the only one  that uses and extends ELI \cite{ELI_2012} which is the official metadata standard for European legislation documents, and is also the only openly accessible ontology regarding GDPR and its concepts \cite{leone_taking_2019}.

GDPRtEXT has been released\footnote{\url{https://w3id.org/GDPRtEXT}} under the open license (CC-by-4.0) and has been incorporated into Ireland's open data portal\footnote{\url{https://data.gov.ie/dataset/gdprtext}}.
The provision of machine-readable concepts and reference to clauses of the GDPR makes GDPRtEXT an important resource for use in legal knowledge graphs.

\subsection{Ontologies for representing activities about Personal Data and Consent}\label{sec:contributions:ontologies}
The second major contribution of this thesis are the two semantic web ontologies - GDPRov for representing information about activities associated with processing of personal data and consent, and GConsent for representing information associated with determining compliance of consent. Both ontologies define concepts and relationships using GDPRtEXT to associate their source within the GDPR.

While GDPRov and GConsent both represent consent, the focus of GDPRov is on representing activities and artefacts, while GConsent represents information for the management of consent information relevant for compliance.
As the application of these ontologies within use-cases in \autoref{chapter:testing} show, both ontologies share some concepts and overlap, but are complimentary in their use and represent different aims in their representation of information.

Together with GDPRtEXT, GDPRov and GConsent enable the representation of activities required to evaluate and validate compliance with the relevant articles of the GDPR. Apart from advancing the state of the art, the ontologies also provide a vocabulary of terms and concepts useful to adopters, and demonstrate the use of legal documents as a source for ontologies using linked data principles.
Their usefulness has been demonstrated in approaches other than compliance - such as for representation of information in privacy policies \cite{pandit_ontology_2018}, generation of privacy policies from metadata \cite{pandit_personalised_2018}, and automating change-detection and its effects on activities \cite{pandit_gdpr-driven_2018}.
GDPRov\footnote{\url{https://w3id.org/GDPRov}} and GConsent\footnote{\url{https://w3id.org/GConsent}} are available under the open license (CC-by-4.0).

\subsubsection{GDPRov}
GDPRov enables representation of the processes and activities associated with the life-cycle of personal data and consent, and  fulfils the research objectives $RO3(b)$ and $RO3(c)$.
GDPRov extends PROV-O \cite{lebo_prov-o_2013} - which is the W3C standard for defining provenance information - to define ex-post (activity logs indicating things that have happened) information, and P-Plan \cite{garijo_p-plan_2014} to define ex-ante (as an abstract model, template, or plan) representations of PROV activities based on scientific workflows.
This enables it to represent planned activities as a model or template which is required to determine ex-ante compliance, and to associate it with its corresponding executions which are required to determine ex-post compliance.
The linking of information between ex-ante and ex-post phase in GDPRov comes from its basis in scientific workflows. It also provides the opportunity to exploit this association for a more efficient approach in evaluation of compliance, as proposed and demonstrated in \autoref{sec:testing:shacl}, and summarised as a contribution in the sections below.
% In this approach, ex-post activities are assumed to be compliant for information already found compliant in their ex-ante representation. Therefore, ex-post activities only need to be evaluated for information and validations specific to the ex-post phase, thus saving the repetition in evaluations of information. An application of this approach is demonstrated using SHACL and features prominent use of GDPRov and GConsent in \autoref{sec:testing:shacl}.

The state of art contains ontologies for representing activities and their provenance related to the GDPR \cite{pasquier_data_2018,palmirani_pronto_2018-1}, including those utilising PROV \cite{belhajjame_provenance_2018,bonatti_special_2018-1}, and holistic approaches combining ex-ante and ex-post compliance \cite{dullaert_d3.4_2019}.
In comparison, GDPRov provides the most exhaustive vocabulary of concepts based on the GDPR (based on comparisons demonstrated in \autoref{chapter:vocabularies}), and is the only ontology to provide ex-ante and ex-post concepts within the same ontology.
GDPRov thus advances the state of the art by providing the most comprehensive vocabulary for modelling and representing activities based on GDPR concepts.

\subsubsection{GConsent}
GDPRov provides representations for indicating activities that request consent and represent artefacts representing the request and given consent as inputs or outputs to such activities.
While these concepts are sufficient to represent information regarding given consent, they represent a specific semantic view based on the notion of capturing provenance of information in ex-ante and ex-post phases.

The determination of consent validity under the GDPR requires additional information \cite{politou_forgetting_2018,article_29_data_protection_working_party_guidelines_2018}, which is not captured by the scope of activities represented by GDPRov or does not relate to its representation of activities and artefacts.
Therefore, a separate modular ontology called GConsent was created based on the research objective $RO3(c)$ to provide the necessary concepts and relationships to represent information for management and evaluate compliance of consent.
GConsent aims to provide representation of \textit{only} consent information as required to evaluate its compliance. It acts as a distinct modular ontology which can be used by itself to represent consent, or in conjunction with GDPRov to represent consent and its related activities.

GConsent provides the necessary concepts and relationships to express information about consent in terms of entities such as individuals or agents, purposes and processing, involvement of third parties, medium and context of provision, relationship between instances (e.g. withdraws, updates), and the novel concept of `consent states' which enable the management of consent as an entity. 
In comparison with the state of the art, GConsent provides greater representation of information related to consent as compared to other approaches related to consent and data management for GDPR \cite{peras_guidelines_2018}, and is the most comprehensive ontology for representing consent (based on comparisons demonstrated in \autoref{chapter:vocabularies}).

\subsection{Querying Information Related to Compliance using SPARQL}\label{sec:contributions:querying}
A minor contribution of this thesis is the utilisation of SPARQL to query  information relevant for GDPR compliance, which fulfils research objective $RO4$.
The use of developed ontologies, namely - GDPRtEXT, GDPRov, and GConsent - provide representation of concepts associated with GDPR for use in SPARQL queries to represent compliance questions derived from state of the art (see \autoref{chapter:information}).

Where approaches in the state of the art also use SPARQL to represent questions for compliance \cite{agarwal_legislative_2018,palmirani_pronto_2018}, the work presented in \autoref{sec:testing:sparql} regarding querying using SPARQL is the only one within the state of the art to derive such queries from the questions used in an investigation of compliance, i.e. compliance questions as presented in \autoref{sec:info:compliance-questions}.
A practical application demonstrates SPARQL queries derived from questions provided by the Irish Data Protection Commission for assisting organisations with their compliance readiness \cite{GDPR_readiness_checklist}, and shows use of SPARQL in assisting the investigation process associated with compliance.
This application of SPARQL was published in a peer-reviewed publication \cite{pandit_queryable_2018} and was presented to members of the Irish Data Protection Commission as part of research developed in this thesis.

\subsection{Framework for Validating Information using SHACL Compliance}\label{sec:contributions:validation}
Another contribution of this thesis is the approach for using SHACL to validate information and linking the results with the relevant clauses of the GDPR for compliance, which fulfils research objective $RO5$.
While SPARQL is sufficient to query information, and in some cases, to determine compliance based on presence or absence of information, the use of SHACL provides a more formal framework for validation of information based on representing constraints and persisting the results of validation.

The validation using SHACL is part of a proposed framework presented in \autoref{sec:testing:shacl} which consists of creating a `compliance graph' for storing information relevant in the investigation and demonstration of compliance.
The validation requirements are derived from constraints and assumptions derived from compliance questions in \autoref{sec:info:constraints}, and are represented using SHACL with a link to relevant clauses of the GDPR established using GDPRtEXT to indicate their role in the compliance process.
The constraints expressed using SHACL utilise concepts and relationships from GDPRov and GConsent to represent validation requirements, and re-use SPARQL queries created for $RO4$ to retrieve information.
The validation results are persisted and annotated with GDPRtEXT to link them with the GDPR, thereby providing a form of documentation for information validation associated with compliance.

The framework suggests a more efficient form of validation by reusing ex-ante validation results in ex-post evaluations by abstracting common constraints belonging to ex-ante information and validating them at the ex-ante stage so that only specific constraints associated with instances in the ex-post stage - such as provenance information - need to be validated.
The demonstration of the framework and the approach consists of evaluating consent on a real-world website and generate a `compliance report' listing status of validations linked to GDPR.
The framework and approach have been published in peer-reviewed publications \cite{pandit_towards_2018,pandit_exploring_2018,pandit_test-driven_2019}

Related work in the state of the art uses a variety of approaches in the validation and assessment of compliance. The SPECIAL project has investigated the use of OWL2 reasoners to validate consent at ex-ante and ex-post stages \cite{bonatti_fast_2018,dullaert_d3.4_2019}, the application of ODRL policies as a compliance checking mechanism \cite{agarwal_legislative_2018,vos_odrl_2019}. The MIREL project has proposed the use of deontic logic for legal reasoning through the use of LegalRuleML \cite{palmirani_pronto_2018,monica_modelling_2018}, while the BPR4GDPR project proposes checking provenance logs for conformance to predetermined processes (ex-post analysis) \cite{mehr_compliance_2019}.
A similar use of SHACL utilising P-Plan workflows to validate policies expressed in ODRL for GDPR compliance has been proposed \cite{lieber_policy-compliant_2019}, which provides future directions for application of this research.
Compared to the state of the art, the approach presented in this thesis is novel in its utilisation of SHACL to validate information and link its results with the GDPR for compliance. It is also novel in its combination and reuse of ex-ante and ex-post validations for compliance.

\subsection{Information Interoperability Model of the GDPR}
A minor contribution of this thesis also consists of an information interoperability model based on representing categories of entities (stakeholders) as defined by the GDPR and their interactions with respect to interoperability of information shaped by compliance requirements of the GDPR.
The model, described in \autoref{sec:info:model}, conceptualises interactions between stakeholders based on information identified as part of $RO1$ and $RO2$, and provides an overview of requirements regarding information and interoperability shaped by GDPR.

The model provides the categorisation of information requirements based on provenance, agreements, consent, certification, and compliance; and assists in the exploration of existing standards, including semantic web, by outlining the requirements and applications of information based on its interoperability between the entities.
It advances the state of the art by providing the first systemic analysis of information flows and interoperability between stakeholders, and serves to provide a framework for developing and evaluating potential consensus on interoperability of information for compliance between stakeholders.
The model, its analysis, and application in the context of right to data portability was published in peer-reviewed publications \cite{pandit_modelling_2017,pandit_exploration_2018} and as a book chapter \cite{pandit_standardisation_2020}.

\subsection{Participation in DPVCG}\label{sec:intro:dpvcg}
The Data Privacy Vocabularies and Controls Community Group\footnote{\url{https://www.w3.org/community/dpvcg/}} (DPVCG) is a W3C community group working towards developing a vocabulary associated with personal data processing based on relevant laws such as GDPR.
The group was created by members of the SPECIAL project, and currently consists of community members from diverse domains such as academia, legal experts, lawyers, and industry stakeholders.

Currently in operation for over 18 months, the work done within the DPVCG has produced the Data Privacy Vocabulary\footnote{\url{http://w3.org/ns/dpv}} (DPV) , an ontological resource for the representation of information associated with processing of data.
The DPV represents a community agreement for the vocabulary and semantics of terms and concepts associated with the GDPR, which provides a degree of interoperability in representing information for legal compliance.
The work conducted in the creation of DPV has been published in a peer-reviewed conference \cite{pandit_dpv_2019}, and has also been listed as a deliverable within the SPECIAL project \cite{pandit_d6.5_2019} - in both of which the author of this thesis is listed as an editor.
Further, the author of this thesis is acting as the chair of DPVCG since the start of 2020.

The research presented in this thesis had an impact in the creation of DPV through the use of developed ontologies as an input as well as through direct participation of the author as an active contributing member.
An overview of the DPV is therefore presented in \autoref{sec:voc:DPV}, along with comparisons to the developed ontologies (GDPRtEXT, GDPRov, GConsent) and the SotA.
To summarise the comparison, DPV provides a high-level abstraction of terms and concepts, whereas the ontologies in this thesis provide representations of information with more granularity and detail - which makes their usage with the DPV complimentary rather than contradictory.

\subsection{Publications}\label{sec:intro:publications}
The following peer-reviewed publications present the research in this thesis (grouped by relevance, ordered reverse chronologically):

\subsubsection{Ontologies representing information for GDPR compliance}
The following publications are associated with $RO3$ - developing ontologies for representing the concepts and relationships within the GDPR.
\begin{enumerate}[start]
    \item ``\textbf{GConsent - A Consent Ontology Based on the GDPR}'' \cite{pandit_gconsent_2019} \\
    \textit{\textbf{H. J. Pandit}, C. Debruyne, D. O’Sullivan, and D. Lewis.} \\ 
    \textit{16\textsuperscript{th} European Semantic Web Conference (ESWC), 2019.}
        \vspace{0.1cm} \newline
        This publication presents the GConsent ontology for representing information about consent as required by the GDPR. GConsent fulfils research objective $RO3(c)$, and provides a verbose representation of consent for information management and documentation. GConsent is described in \autoref{sec:voc:GConsent}.
    \item ``\textbf{GDPRtEXT - GDPR as a Linked Data Resource}'' \cite{pandit_gdprtext_2018} \\
    \textit{\textbf{H. J. Pandit}, K. Fatema, D. O’Sullivan, and D. Lewis.} \\
    \textit{15\textsuperscript{th} European Semantic Web Conference (ESWC), 2018.}
        \vspace{0.1cm} \newline
        This publication presents the GDPRtEXT resource consisting of a linked data representation of the text of GDPR, and a thesauri of its concepts. It also provides a mapping from clauses of the DPD to GDPR based on applicability of access control methods developed for DPD towards GDPR compliance. GDPRtEXT fulfils research objective $RO3(a)$, and is instrumental in providing semantic association between information and the GDPR for the approaches presented in this thesis. GDPRtEXT is described in \autoref{sec:voc:GDPRtEXT}.
    \item ``\textbf{Modelling Provenance for GDPR Compliance using Linked Open Data Vocabularies}'' \cite{pandit_modelling_2017} \\ 
    \textit{\textbf{H. J. Pandit}, and D. Lewis.} \\
    \textit{5\textsuperscript{th} Workshop on Society, Privacy and the Semantic Web - Policy and Technology (PrivOn2017), co-located with the 16\textsuperscript{th} International Semantic Web Conference (ISWC), 2017. }
        \vspace{0.1cm} \newline
        This publication presents the GDPRov ontology for representing the provenance of personal data and consent for GDPR, and also discusses the use of its concepts in SPARQL queries for retrieving information associated with compliance. GDPRov fulfils research objective $RO3(b)$, and provides ex-ante and ex-post representations for activities associated with personal data and consent for GDPR. GDPRov is described in \autoref{sec:voc:GDPRov}.
    \item \textbf{Compliance through Informed Consent: Semantic Based Consent Permission and Data Management Model} \cite{fatema_compliance_2017} \\
    \textit{K. Fatema, E. Hadziselimovic, \textbf{H. J. Pandit}, C. Debruyne, D. Lewis, and D. O’Sullivan.} \\
    \textit{5\textsuperscript{th} Workshop on Society, Privacy and the Semantic Web - Policy and Technology (PrivOn2017), co-located with the 16\textsuperscript{th} International Semantic Web Conference (ISWC), 2017. }
        \vspace{0.1cm} \newline
        This publication presents an early (pre-GDPR enforcement) collaboration in developing a preliminary ontology for representing consent and a data management model for GDPR. The early work was crucial towards understanding the complexities of consent, and provided valuable feedback towards the development of GConsent.
    \item ``\textbf{Linked Data Contracts to Support Data Protection and Data Ethics in the Sharing of Scientific Data}'' \cite{hadziselimovic_linked_2017} \\ 
    \textit{E. Hadziselimovic, K. Fatema, \textbf{H. J. Pandit}, and D. Lewis.} \\ 
    \textit{Workshop on Enabling Open Semantic Science (SemSci), co-located with the 16\textsuperscript{th} International Semantic Web Conference (ISWC), 2017.}
        \vspace{0.1cm} \newline
        This publication presents an early collaboration (pre-GDPR) towards developing an ontology for representing data sharing agreements by extending the ODRL ontology based on GDPR. The ontology - called Data Protection Rights Language (DPRL) - enables representation of obligations associated with propagation of rights between parties that share or exchange data.
\end{enumerate}

\subsubsection{Querying and validating information for GDPR compliance}
The following publications are associated with $RO4$ - querying for information, and $RO5$ - validating information for compliance.
\begin{enumerate}[resume]
    \item ``\textbf{Test-driven Approach Towards GDPR Compliance}'' \cite{pandit_test-driven_2019} \\
    \textit{\textbf{H. J. Pandit}, D. O’Sullivan, and D. Lewis.} \\ 
    \textit{14\textsuperscript{th} International Conference on Semantic Systems (SEMANTiCS), 2019.}
        \vspace{0.1cm} \newline
        This publication presents an overview of the approach for validating information using SHACL and associating the results with specific articles of the GDPR. The approach proposes persistence of validation results to create a `compliance graph' that can itself by queried and validated for documenting information for compliance. This research fulfils research objective $RO5$ and is presented in \autoref{sec:testing:shacl}.
    \item ``\textbf{Queryable Provenance Metadata For GDPR Compliance}'' \cite{pandit_queryable_2018} \\
    \textit{\textbf{H. J. Pandit}, D. O’Sullivan, and D. Lewis.} \\ 
    \textit{14\textsuperscript{th} International Conference on Semantic Systems (SEMANTiCS), 2018.}
        \vspace{0.1cm} \newline
        This publication presents the use of SPARQL queries to represent questions associated with compliance by using the GDPRtEXT and GDPRov ontologies.
        It demonstrates effectiveness of SPARQL in retrieving information for GDPR compliance, and fulfils research objective $RO4$. This work is presented in \autoref{sec:testing:sparql}.
    \item ``\textbf{ Exploring GDPR Compliance Over Provenance Graphs Using SHACL}'' \cite{pandit_exploring_2018} \\
    \textit{\textbf{H. J. Pandit}, D. O’Sullivan, and D. Lewis.} \\ 
    \textit{14\textsuperscript{th} International Conference on Semantic Systems (SEMANTiCS) - Posters track, 2018.}
        \vspace{0.1cm} \newline
        This publication presents the implementation of the approach described above for validation of information by utilising the use-case of consent mechanism in a real-world website. It utilises SHACL to validate information represented by GDPRov and GConsent, and uses GDPRtEXT to associate the tests and their results with the GDPR. It also demonstrates the use of SPARQL to identify tasks and reports related to compliance by querying the validation results. The approach demonstrates the usefulness of combining ex-ante and ex-post approaches in terms of efficiency and compliance, and fulfils research objective $RO5$. This work is presented in \autoref{sec:testing:shacl:approach}.
    \item ''\textbf{Towards Knowledge-based Systems for GDPR Compliance}'' \cite{pandit_towards_2018} \\
    \textit{\textbf{H. J. Pandit}, C. Debruyne, D. O’Sullivan, and D. Lewis.} \\ 
    \textit{International Workshops on Contextualized Knowledge Graphs (CKG), co-located with 17\textsuperscript{th} International Semantic Web Conference (ISWC), 2018.}
        \vspace{0.1cm} \newline This publication explores the creation of a knowledge-based framework based on the utilisation of information associated with compliance using semantic web technologies, which can be utilised in applications such as creation of reports, documentation, and assessment of compliance for different stakeholders. The approach was used in conjunction with the above mentioned publication in addressing research objective $RO5$.
\end{enumerate}

\subsubsection{Model for information interoperability based on requirements of GDPR compliance}
These publications present a model of interaction between entities as defined by the GDPR, and explore information categories and their interoperability requirements based on existing standards, including those provided by the semantic web.
The model provides an overview of information flows between stakeholders, and the role of interoperability in facilitating information for compliance between them. This research is presented in \autoref{sec:info:model}.
\begin{enumerate}[resume]
    \item ``\textbf{Standardisation, Data Interoperability, and GDPR}'' \cite{pandit_exploration_2018} \\
    \textit{\textbf{H. J. Pandit}, C. Debruyne, D. O’Sullivan, and D. Lewis.} \\ 
    \textit{Book Chapter in Shaping the Future Through Standardization, 2019}
    \item ``\textbf{An Exploration of Data Interoperability for GDPR}'' \cite{pandit_standardisation_2020} \\
    \textit{\textbf{H. J. Pandit}, C. Debruyne, D. O’Sullivan, and D. Lewis.} \\ 
    \textit{International Journal of Standardization Research (IJSR) , Vol. 16 Issue. (1), 2018}
    \item ``\textbf{GDPR Data Interoperability Model}'' \cite{pandit_gdpr_2018} \\
    \textit{\textbf{H. J. Pandit}, D. O’Sullivan, and D. Lewis.} \\ 
    \textit{23\textsuperscript{rd} European Academy for Standardisation Annual Standardisation Conference (EURAS), 2018}
\end{enumerate}

\subsubsection{Investigated applications of research - Information Management}
The following publications do not address the research question, but consist of applying the research presented in this thesis towards processes that assist with being compliant.
\begin{enumerate}[resume]
    \item ``\textbf{Towards Generating Policy- Compliant Datasets}'' \cite{debruyne_towards_2019} \\
    \textit{C. Debruyne, \textbf{H. J. Pandit}, D. O’Sullivan, and D. Lewis.} \\ 
    \textit{13\textsuperscript{th} IEEE International Conference on Semantic Computing (ICSC), 2019.}
    \vspace{0.1cm} \newline This publication presents an approach for generating just-in-time datasets consisting of personal data based on given consent to ensure processes are compliant in their usage of consent as the legal basis.
    \item ``\textbf{GDPR-driven Change Detection in Consent and Activity Metadata}'' \cite{pandit_gdpr-driven_2018} \\
    \textit{\textbf{H. J. Pandit}, D. O’Sullivan, and D. Lewis.} \\ 
    \textit{4\textsuperscript{th} Workshop on Managing the Evolution and Preservation of the Data Web (MEPDaW), co-located with 15\textsuperscript{th} European Semantic Web Conference (ESWC), 2018.}
    \vspace{0.1cm} \newline This publication proposes an approach for detecting changes related to use of personal data and consent in activities by utilising the ex-ante component of GDPRov to represent activities and comparing them using a graph-based algorithm.
\end{enumerate}

\subsubsection{Investigated Applications of Research - Privacy Policies}
The following publications do not address the research question, but consist of applying the research presented in this thesis towards privacy policies.
\begin{enumerate}[resume]
    \item ``\textbf{Extracting Provenance Metadata from Privacy Policies}'' \cite{pandit_extracting_2018} \\
    \textit{\textbf{H. J. Pandit}, D. O’Sullivan, and D. Lewis.} \\ 
    \textit{7\textsuperscript{th} International Provenance \& Annotation Workshop (IPAW), par t of Provenance Week, 2018.}
    \vspace{0.1cm} \newline This publication discusses the use of GDPRov to represent extracted extracting information about activities associated with personal data from within a privacy policy.
    \item ``\textbf{An Ontology Design Pattern for Describing Personal Data in Privacy Policies}'' \cite{pandit_ontology_2018} \\
    \textit{\textbf{H. J. Pandit}, D. O’Sullivan, and D. Lewis.} \\ 
    \textit{9\textsuperscript{th} Workshop on Ontology Design and Patterns (WOP), co-located with 17\textsuperscript{th} International Semantic Web Conference (ISWC), 2018.}
    \vspace{0.1cm} \newline This publication presents an ontology design pattern that uses GDPRov and GDPRtEXT to represent information about personal data and its processing in a privacy policy.
    \item ``\textbf{Personalised Privacy Policies}'' \cite{pandit_personalised_2018} \\
    \textit{\textbf{H. J. Pandit}, D. O’Sullivan, and D. Lewis.} \\ 
    \textit{4\textsuperscript{th} International Workshop on TEchnical and LEgal aspects of data pRIvacy and SEcurity (TELERISE), co-located with 22\textsuperscript{nd} European Conference on Advances in Databases and Information Systems, 2018.}
    \vspace{0.1cm} \newline This publication discusses the personalisation of privacy policies by using information about an individual's personal data processing to generate the text of the policy, and using GDPRtEXT and GDPRov to annotate it for a machine-readable representation.
\end{enumerate}

\subsubsection{Data Privacy Vocabulary}
The following publication presents the work related to the creation of the Data Privacy Vocabulary by the DPVCG, and describes the methodology used as well as relation to the existing vocabulary, including those presented in this thesis - namely GDPRtEXT, GDPRov, and GConsent. The DPV is described in \autoref{sec:voc:DPV}.
\begin{enumerate}[resume]
    \item ``\textbf{Creating A Vocabulary for Data Privacy}'' \cite{pandit_creating_2019} \\
    \textit{\textbf{H. J. Pandit}, A. Polleres, B. Bos, R. Brennan, B. Bruegger, F. J. Ekaputra, J. D. Fernández, R. G. Hamed, E. Kiesling, M. Lizar, E. Schlehahn, S. Steyskal, R. Wenning} \\
    \textit{18\textsuperscript{th} International Conference on Ontologies, DataBases, and Applications of Semantics (ODBASE), 2019.}
\end{enumerate}


\section{Thesis Overview}
The rest of this thesis is structured as follows:

\subsubsection*{\autoref{chapter:background}: Background on GDPR and Semantic Web}
This chapter presents a summary of information required to understand the work presented in this thesis. The chapter consists of two sections, the first describes the concepts and requirements of the GDPR with particular focus on information about activities associated personal data and consent. The second section describes the semantic web technologies through an overview of standards and vocabularies, and the mechanisms used for querying and validation of information.

\subsubsection*{\autoref{chapter:sota}: State of the Art}
This chapter reviews existing work and approaches regarding regulatory compliance with a specific focus on those addressing GDPR compliance. The chapter starts by providing an overview of approaches used for legal compliance. It then presents an in-depth review of approaches utilising semantic web technologies to address GDPR compliance requirements, followed by other approaches for GDPR compliance. Approaches which do not directly address the GDPR, but are relevant to the domain of legislative compliance and semantic web are also presented. The chapter then presents an analysis of the state of the art, and concludes with a discussion on identified gaps and limitations.

\subsubsection*{\autoref{chapter:information}: Information Required for GDPR Compliance}
This chapter presents the information required for GDPR compliance of activities associated with processing of personal data and consent in ex-ante and ex-post phases.
The chapter starts by presenting an information model for the interoperability of information between stakeholders defined by the GDPR.
The model provides an analysis of the information interoperability requirements based on the requirements of GDPR compliance, and the role of existing standards in addressing them.
This is followed by expressing the information requirements in the form of analytical questions, termed `compliance questions', whose answers provide the information necessary to evaluate compliance. 
The chapter then concludes with the use of compliance questions in the ontology engineering process as `competency questions', and the identification of constraints and assumptions which can be used to validate the information for GDPR compliance.

\subsubsection*{\autoref{chapter:vocabularies}: Representing Information for GDPR Compliance using Ontologies}
This chapter presents the OWL2 ontologies developed to represent information associated processing of personal data and consent for GDPR compliance.
The ontologies present concepts for answering the compliance queries presented in Chapter 4. The first ontology presented is GDPRtEXT, which provides a method to link information with the concepts and clauses of GDPR through a linked data version of its text and a vocabulary of concepts. The second ontology presented is GDPRov, which enables representation of provenance information regarding personal data and consent in the form of models or templates and their executions or activity logs. The third ontology presented is GConsent, which enables representation of information associated with consent. The chapter presents an overview of the concepts and relationships for each vocabulary, its relation with the GDPR, and how the vocabulary uses the compliance queries as competency questions to guide its development.

\subsubsection*{\autoref{chapter:testing}: Querying and Validating Information for GDPR Compliance}
This chapter presents the use of SPARQL to express the compliance queries using the ontologies presented in Chapter 5. The chapter also presents a framework to validate information based on the constraints identified in Chapter 4 through the use of SHACL. The framework demonstrates the use of semantic web technologies in validating information for GDPR compliance by utilising a combination of ex-ante and ex-post validations, and the linking of results with the GDPR for documentation of information for compliance.

\subsubsection*{\autoref{chapter:conclusion}: Conclusion}
This chapter concludes the thesis with a summary of key findings and outcomes of the presented work. It discusses the extent to which the thesis serves to address the research question(s) and objective(s), and outlines directions for future work in terms of potential applications and extension through related work.
