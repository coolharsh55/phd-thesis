\chapter{Conclusion}
This chapter presents the conclusion for the work presented in this thesis with \autoref{sec:conclusion-RO} presenting the extent to which the four research objectives defined in \autoref{sec:intro-RO} address the research question, with the resulting contributions summarised in \autoref{sec:conclusion-contributions}. Avenues for further work are discussed in Section \autoref{sec:conclusion-future-work}, with final remarks presented in \autoref{sec:conclusion-final-remarks} concluding the thesis.

% RESEARCH OBJECTIVES
\section{Fulfilment of Research Objectives}\label{sec:conclusion-RO}
% RO1
\subsection*{RO1: Identifying information required for GDPR compliance involving provenance of personal data and consent}
The first research objective (\textbf{RO1}) was to identify information required for GDPR compliance which involves provenance of personal data and consent. This was fulfilled by work presented in \autoref{chapter:information}.

In order to analyse the information requirements of the GDPR, an information model was created to explore the entities and their relationships with respect to information exchange. The model provided an analysis of GDPR  in the form of requirements and processes associated with information for compliance, and was used to categorise information as Provenance, Agreements, Consent, Certification, and Compliance. These were used to identify the nature and source of information and its relationship with entities.

The research objective $RO1$ was achieved through creation of `compliance queries' - questions whose answers provide information associated with evaluating compliance. These questions were compiled from the authoritative guidelines provided by European Data Protection Commissioner's offices, reports and opinions produced by Article 29 Working Party regarding interpretation of GDPR, and documents published by institutions providing legal services. The questions were used to formulate constraints which information was required to satisfy in order to be valid based on assumptions which always held true. Finally, concepts and relationships were identified by interpreting the questions with its constraints and assumptions as competency questions.

% RO2
\subsection*{RO2: Creating semantic web ontologies for expressing information required for GDPR compliance}
The second research objective (\textbf{RO2}) was to create ontologies to represent information required for GDPR compliance identified in $RO1$. Specifically, $RO2$ involved creation of ontologies for representing information associated with compliance regarding - (a)  concepts and text of GDPR, (b) provenance of consent and personal data, and (c) consent. This was achieved through work presented in \autoref{chapter:vocabularies} consisting of GDPRtEXT, GDPRov, and GConsent ontologies for each of respective sub-objectives.

GDPRtEXT enabled unambiguous and machine-readable linking of information to concepts and text of GDPR. GDPRtEXT provided an OWL2 ontology to represent the structured text of GDPR as individual Recitals, Chapters, Sections, Articles, Points, Sub-Points, and Citations, by extending the European Legislation Identifier (ELI) ontology. ELI is the authoritative ontology used by the European Publication Office to define metadata for all published documents. The extension mechanism used by GDPRtEXT maintains formal compatibility with ELI. Using GDPRtEXT, the text of GDPR was re-defined as linked data in machine-readable representations by assigning an unique identifier for individual resources. GDPRtEXT also provides a SKOS vocabulary of terms and concepts defined or referenced by the GDPR to create a thesauri of concepts relevant to its compliance. Thus, GDPRtEXT fulfils $RO2(a)$ regarding provision of a mechanism for associating information with concepts and text of the GDPR.

GDPRov is an OWL2 ontology which addresses $RO2(b)$ regarding representing provenance of personal data and consent. It extends the PROV-O and P-Plan ontologies with terms and relationships relevant for GDPR, where PROV-O is the W3C standard for representing provenance information, and P-Plan is its extension for defining abstract models as plans which then get instantiated into activities having provenance. GDPRov exploits the PROV-O and P-Plan relationship to represent a model or plan of how processes are supposed to interact with personal data and consent, such as for collection, use, storage, and sharing. The model or plan can then be used as the template for activities to be carried out whose provenance is linked to the model. Apart from providing terms for addressing personal data and consent, GDPRov also enables representation of other processes defined by GDPR, such as handling rights and data breaches, which can similarly be depicted using a model or plan. GDPRov addresses competency questions identified in $RO1$ regarding provenance and fulfils $RO2(b)$.

The third research sub-objective was addressed by creating GConsent - an OWL2 ontology to represent information associated with consent. GConsent expands upon the use of consent as an abstract entity in GDPRov by providing representation of contextual information associated with actors, state, relationships, and provenance of consent in verbose detail. In particular, it enables the association of purpose, processing, personal data, data subject, third parties, and delegates with a specific instance of consent. It also enables the capture of contextual information such as the medium the consent was given, timestamp,  and location. States in GConsent reflect the status of consent and provide an indication of consent, such as `requested' or `explicitly given' or `invalidated', which are categorised based on whether they can be used as a valid legal basis for processing. GConsent thus provides a verbose vocabulary for the representation of information associated with consent and fulfils $RO2(c)$.

% evaluation, best practices, CQ, w3id
Each ontology was constructed by gathering requirements from competency questions, creating associated ontological representations, and evaluating based on use-cases and competency questions. For GDPRtEXT, requirements were gathered from an analysis of textual references to legal documents. For GDPRov and GConsent, competency questions were constructed from compliance queries identified through $RO1$. Each ontology incorporated best practices and guidelines advocated by the semantic web community. The ontologies were updated periodically based on identified deficiencies and requirements through explored applications and use-cases. Finally, each ontology was made open and accessible under CC-by-4.0 license, and published in an academic venue for exposure and validation through peer-review.

\subsection*{RO3: Creating SPARQL queries for retrieving information about provenance of personal data and consent relevant for GDPR compliance}
The third research objective (\textbf{RO3}) was to create SPARQL queries that retrieve information about provenance of personal data and consent. These SPARQL queries are semantic representations of compliance queries identified in $RO1$ and utilise ontologies created in $RO2$ to define concepts and relationships pertaining to GDPR. The SPARQL queries demonstrate the linking of retrieved information with relevant concepts and parts of the GDPR, as well as the creation of knowledge graphs for use in compliance processes. This work was presented in \autoref{chapter:testing}.

\subsection*{RO4: Creating a framework using semantic web technologies for validating and evaluating information about provenance of personal data and consent and linking results to relevant concepts and clauses of the GDPR}
The fourth and final research objective (\textbf{RO4}) was the creation of a framework utilising semantic web technologies to validate and evaluate provenance of personal data and consent, and linking the results to relevant concepts and clauses of GDPR. This was fulfilled by the work presented in \autoref{chapter:testing}.

The framework utilised SHACL to validate and evaluate information based on the constraints and assumptions identified in $RO1$. In this context, validation was defined as testing conformance of information to constraints, while evaluation was defined as testing adherence to requirements of compliance. By utilising SHACL, tests and their results were associated with GDPR using the mechanism provided by GDPRtEXT. The framework was demonstrated through a use-case generated from the consent mechanism on a real-world website where data was defined using ontologies created for $RO2$ and evaluated using constraints from $RO1$. The evaluations utilised SPARQL queries from $RO3$ to retrieve and validate information by using SHACL-SPARQL.

The framework provided three types of evaluations based on the use of approach advocated through GDPRov regarding models and provenance. The first type evaluated individual instances of consent from provenance log (ex-post), while the second type evaluated the model of consent (ex-ante). The third type combined both ex-ante and ex-post types by testing common requirements on the model, and reusing its results to evaluate only  instance-specific requirements over provenance log.
The framework demonstrated usefulness of created ontologies towards linking information, and showed the advantages of representing provenance as a model for evaluation of information.

\section{Contributions}\label{sec:conclusion-contributions}
This section provides a summary of contributions of this thesis, initially presented in \autoref{sec:intro-contributions}, along with its impact to date. The thesis yielded two major contributions - enabling linking of information with concepts and text of GDPR, and semantic web ontologies for representing information associated with provenance of personal data and consent. The thesis also yielded minor contributions in the form of an information model of entities and their relationships within GDPR, and a framework for validating and evaluating information for compliance using semantic web technologies.

\subsection*{Major Contributions}
\subsubsection{GDPR as a Linked Data Resource}
The first major contribution, represented by GDPRtEXT, enables association of information with the concepts and text of GDPR using linked data principles. It provides machine-readable unique identifiers for each specific part (Chapter, Article, etc.) of the GDPR by representing its text in RDF using an extension of the European Legislation Identifier (ELI) ontology. It also provides a SKOS vocabulary of concepts and terms defined or represented within the text of the GDPR. The usefulness of GDPRtEXT has been demonstrated in its use to define the source of terms in the ontologies presented in this thesis, as well as in linking information related to compliance with the relevant concepts and clauses of the GDPR. GDPRtEXT advances the state of the art in its provision of unambiguous and machine-readable representations of concepts and text of GDPR.

\subsubsection{Ontologies for representing provenance of personal data and consent}
The second major contribution is the GDPRov and GConsent ontologies, which together enable representation of information associated with provenance of personal data and consent relevant for investigation of GDPR compliance.

GDPRov expands the existing ontologies of PROV-O and P-Plan with concepts and relationships specific to GDPR in order to represent provenance of personal data and consent at ex-ante and ex-post stages. Where ex-post representations are common as provenance logs, the ex-ante representations act as a model or plan of intended activities for evaluation of compliance. Furthermore, provenance logs (ex-post) can be linked to their models (ex-ante) to represent the relationship between planning and implementation. GDPRov also enables representation of processes associated with the GDPR such as the handling of various rights and data breach.

GConsent expands upon the abstract representation of consent in GDPRov to provide more verbose information regarding entities and contextual information relevant for the management of consent. It also provides the concept of `consent states' which reflect the use of consent as a valid legal basis and are useful in the  representation and management of consent in information systems. To date, GConsent is the most comprehensive vocabulary regarding consent associated with the GDPR.

\subsection*{Minor Contributions}
The minor contributions of this thesis are - an information model of entities and their relationships defined by the GDPR, and a framework utilising semantic web technologies for validating and evaluating information for GDPR compliance. The minor contributions complement the previously described major contributions by providing a theoretical basis in the form of an information model, and demonstrate the feasibility and usability of developed ontologies through an application for validating information for compliance.

The first minor contribution is an information model, which was presented in \autoref{sec:information-model}, and provides an analysis of information exchanged between entities based on requirements of GDPR, along with a categorisation of information as provenance, agreements, consent, certification, and compliance. The model furthers the state of the art by enabling exploration of existing standards of interoperability for information exchange under GDPR, and provides the argument for semantic web as a suitable representation based on the notion of semantic interoperability.

The second minor contribution, as presented in \autoref{chapter:testing}, is a framework that utilises semantic web technologies in order to validate and evaluate information for GDPR compliance. The framework associates results of validation with text of GDPR by using GDPRtEXT to enable the creation of machine-readable documentation and coverage scores. The validation itself is done using SHACL, which enables representing the validation tests as machine-readable data using RDF, as well as associating tests with specific articles and concepts within the GDPR. 
The evaluations carried out in the framework utilise ex-ante and ex-post tests while also demonstrating the advantages of a combined approach.
The framework demonstrates the use of compliance queries by representing them in SPARQL and utilising them in validation using SHACL-SPARQL. 
Furthermore, the framework is applicable for all validation methods, as the outcome of any validation method can be linked with the text of GDPR by annotating it with GDPRtEXT.



% \subsection*{Impact of Research}
% A significant portion of the work presented in this thesis has been disseminated to the community through peer-reviewed academic papers published in journals, conferences, and workshops. 

\section{Opportunities for Further Work}\label{sec:conclusion-future-work}
Due to the novelty of GDPR and increased interest in its compliance, there are several opportunities where the work presented in this thesis can be further developed and applied, as categorised in the following three areas -
\subsection*{Align approaches for Regulatory Compliance}
% \subsubsection{Semantic analysis of ontologies targeting GDPR}
% - Consolidate all vocabularies to create a cohesive domain ontology aligned to GDPR
Differences in domain ontologies offer varying perspectives on the modelling of relationships and concepts within the same domain. In the case of GDPR, these ontologies can be compared using the commonality of concepts and aims. For example, `consent' is represented in the ontologies GDPRtEXT, GDPRov, GConsent, SPECIAL \cite{kirrane_scalable_2018}, and PrOnto \cite{palmirani_pronto:_2018} - where each representation is based on the same concept of consent, and yet differs in its modelling of the relationships associated with consent. A comparison of ontologies based on semantics of concepts is useful to establish compatibility in their usage and approaches, and to evaluate their usefulness for a given use-case.

The state of the art, presented in \autoref{chapter:sota}, describes existing work outlining such an analysis \cite{leone_taking_2019} and its use in a tool \cite{leone_legal_2018} to compare approaches in the legal domain. It also presents approaches involving application of deontic logic to address regulatory compliance for GDPR, where the text is interpreted using ODRL \cite{agarwal_modelling_2017}, and  PrOnto \cite{palmirani_pronto:_2018} which models deontic operations and uses LKIF \cite{hoekstra_lkif_2007} to model actions and roles.
This can be further expanded to align existing approaches and ontologies through semantics of concepts provided by GDPRtEXT.
% , and to establish a library of design patterns for use of concepts for GDPR compliance.

\subsection*{Expand Scope of Ontologies}
\subsubsection{Incorporate future updates to ELI into GDPRtEXT}
GDPRtEXT addresses the aim of linking to specific parts of the GDPR by extending the ELI ontology. The EU Publications Office, as the official developers and maintainers of ELI, are currently working on updating the ELI ontology to enable such linking for all published documents. Their work will provide authoritative URIs for all aspects of a legal document, and will also enable identification of definitions. Once published, the updated ELI ontology will make the GDPRtEXT extension redundant. However, GDPRtEXT will still have uses as a SKOS vocabulary of concepts that is used by ontologies such as GDPRov and GConsent to define the source of their concepts and relationships. By updating GDPRtEXT to use the updated ELI ontology, the interpretation of GDPR as a linked data resource can be provided using the authoritative URIs for use with the provided SKOS vocabulary.

\subsubsection{Create vocabulary for expressing GDPR Compliance}
The vocabularies associated with GDPR, including those presented in the state of the art in \autoref{chapter:sota} and as contributions of the thesis in \autoref{chapter:vocabularies}, address compliance by associating information with its requirements. This establishes the opportunity to create a vocabulary that represents compliance itself by describing the state of information in fulfilling requirements. Such a vocabulary would be of use to supervisory authorities as well as organisations in generating documentation demonstrating the compliance of information as well as the degree to which it was fulfilled or achieved.

% \subsubsection{Enable use of mappings to align existing information systems}
% - Use the ontologies and queries to create an abstraction over existing information system so that existing systems can use the linked queries using SPARQL and mappings

\subsubsection{Expand GConsent to capture real-world interactions on the web}
The aim of GConsent, as presented in this thesis, is to represent information about consent. While it is a comprehensive and verbose ontology compared to the state of the art, it currently is not sufficient to express the nuances and complexities of real-world interactions - such as those found in the consent mechanisms on websites. More specifically, it lacks a way to describe the intricate relationships of different organisations, including third parties, and the combined collection and dissemination of consent which happens via real-time bidding online. The real-time bidding mechanism operates on a large scale and is complex to interpret \cite{eijk_web_2019} due to the large scale interactions as well as dynamic nature of entities involved. Representing this would enable GConsent to be of assistance in legal analysis of online privacy and consent.

\subsection*{Generate Assistive Systems for Compliance}
\subsubsection{Incorporate GDPRov and GConsent in the SPECIAL compliance checker}
The compliance checker developed by the SPECIAL project \cite{kirrane_scalable_2018} uses a controlled vocabulary consisting of personal data, processing, purpose, storage, and recipients. It is possible to use the SPECIAL compliance checker to check the compliance of information defined using GDPRov and GConsent by modifying the checker to target these vocabularies or by alignment of SPECIAL vocabularies with GDPRov and GConsent. This would enable the work presented in this thesis to take advantage of large scale analysis and transparent log mechanisms provided by the SPECIAL architecture. Evaluation of the approach would be based on analysis of scalability and performance to ascertain extent of its benefit.

\subsubsection{Privacy Policy annotation and automatic generation}
A privacy policy fulfils the legal requirement for dissemination of information concerning the processing of personal data. Existing approaches for annotating privacy policies \cite{oltramari_privonto:_2018,harkous_polisis:_2018} do not take into account the semantics of associated information, nor effects of GDPR on privacy policy as a document. The argument for a privacy policy dataset specifically annotated for GDPR \cite{galle_case_2019} consists of using concepts relevant to the legislation in the annotation process. This can be achieved through use of GDPRtEXT as a vocabulary of GDPR concepts. In addition, workflows represented using GDPRov provide the necessary information in order to generate a partial privacy policy, and can be used to automate generation of text by converting the processing workflows into natural language text. An early exploration of this work regarding annotation of  privacy policy and personalising was presented in \cite{pandit_personalised_2018}.

\subsubsection{Design patterns for GDPR compliance}
While there are verbose ontologies to represent information associated with compliance, their specific usage is dependant on the applied use-case. To facilitate adoption and usage, a library of design patterns can be created where each pattern is concerned with representing information associated with compliance for a specific concept or clause of the GDPR. For example: a design pattern representing periodic collection of GPS data from smartphone devices, which is linked with applicable clauses of GDPR as well as requirements or constraints it must fulfil in order to be compliant. Such design patterns can be used as the basis for assistive tools that generate and assess information for compliance.

% \subsubsection{Tool to validate and assess compliance documentation}
% Create a set of constraints that validate model of the system and assist organisations in ensuring their processes are documented. e.g. process to handle data breach exists

\section{Final Remarks}\label{sec:conclusion-final-remarks}
GDPR is the subject of scrutiny due to its impending interpretation by supervisory authorities and courts of law and the possibility of incurring large amount of fines. Consequently, there is significant interest in approaches associated with its compliance, particularly those that involve technological means as they promise algorithmic solutions that can be automated.

Technological solutions towards addressing compliance are dependant on the underlying information model, and have a range of approaches to choose from - as is evident in the state of the art regarding regulatory compliance. However, it can be argued that the law ultimately only deals with legal documentation where information is invariably linked with specific clauses of the law.

% The work presented in this thesis is a step towards enabling technological solutions that assist in the linking of information for compliance in an interoperable and machine-readable form through semantic web.

Within this context, the work presented in this thesis is useful for all involved stakeholders - organisations, supervisory authorities, and data subjects - by enabling creation of tools and services to assist in the representation, querying, and validation of information. In particular, the thesis establishes advantages of using semantic web technologies and provides an argument towards their adoption in the regulatory compliance domain.

With an increased need and focus on the intersection of technology and privacy, approaches based on semantic web can foster transparency and accountability by enabling an open medium for knowledge interaction for all stakeholders. It is therefore the author's hope that this thesis and the work presented therein is of benefit to society for meeting the expectations demanded by privacy laws such as GDPR as well those arising from social obligations. 
