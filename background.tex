\chapter{Background: GDPR \& Semantic Web}
\label{chapter:background}

This chapter presents the necessary background information related to understanding the research presented in this thesis. In particular, it presents a short introduction to the General Data Protection Regulation (GDPR) in \autoref{sec:background:GDPR} and to the semantic web in \autoref{sec:background:semweb}. The information only represents a summary of the topic, and is accompanied with links for further information.

\section{General Data Protection Regulation (GDPR)}\label{sec:background:GDPR}
\todo{copy background info from DPC website}

\subsection*{Further Reading}

\subsubsection{PRIPARE}
PRIPARE\footnote{\url{http://pripareproject.eu/}} (PReparing Industry to Privacy-by-design by supporting its Application in REsearch) is an European research project that aimed to provide a set of documents regarding privacy engineering by covering activities such as privacy risk management, requirement analysis, design strategies, maintenance and compliance. To this end, it published the PRIPARE methodology handbook \cite{noauthor_pripare-methodology-handbook-final-feb-24-2016.pdf_nodate} providing guidelines for privacy and security by design. The handbook incorporates information based on a draft of GDPR\footnote{\url{The handbook was published in 2015, and incorporated known information about the GDPR up to that point in time.}} It provides foundational methodologies and reference models for carrying out privacy analysis, designing and implementing privacy enhancing systems, with templates for impact assessments and conformance. 

\subsubsection{STAR \& STAR II}
STAR\footnote{\url{https://projectstareu.wordpress.com/}} (Support Training Activities on the data protection Reform) is an European project that had the aim to provide materials to support training of DPAs and DPOs for the GDPR. To this end, it has created and published its resources consisting of training materials in an open and publicly accessible form\footnote{\url{http://www.project-star.eu/training-materials}}. The project has also produced a handbook for assisting stakeholders in understanding the GDPR and preparing for its compliance. The project also provides an evaluation questionnaire and compliance checklist \cite{noauthor_gdpr_2019-1}consisting of a list of questions and criterion to assess preparedness with the requirements of the GDPR. 

The resources published by the STAR projects provide documentation regarding GDPR compliance that is adopted by organisations and regulatory authorities. As such, it acts as a source of information about GDPR in the requirements gathering phase associated with information representation and evaluation of compliance.

\section{Semantic Web Technologies}\label{sec:background:semweb}
This section provides background information on the semantic web. It introduces the semantic web standards of RDF, RDFS, OWL, SPARQL, and SHACL.
Through these concepts, the application of linked data towards information management is described.

\subsection{Resource Description Framework (RDF)}

\subsection{RDF Schema (RDFS)}

\subsection{Web Ontology Language (OWL)}

\subsection{SPARQL Protocol and RDF Query Language (SPARQL)}

\subsection{Shapes Constraint Language (SHACL)}

\subsection{Semantic Web Applications and Technologies}

\subsubsection{Semantic Web Stack}

\subsubsection{Knowledge Graphs}

\subsection*{Further Reading}