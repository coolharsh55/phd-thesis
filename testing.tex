\chapter{Querying and Validating Information for GDPR Compliance}
\label{chapter:testing}
% chapter introduction
This chapter presents the application of semantic web technologies to query and validate information for GDPR compliance.
It represents information using the developed vocabularies of GDPRtEXT, GDPRov, and GConsent presented in \autoref{chapter:vocabularies}.
The queries are represented using SPARQL - a W3C standard for querying RDF - and are based on the compliance questions presented in \autoref{sec:info:compliance-questions}.
Validation is carried out based on the identified assumptions and constraints presented in \autoref{sec:info:constraints} and is expressed using SHACL which is the W3C standard for representing constraints.
The presented work represents minor contributions of this thesis, and fulfils the research objective $RO4$ regarding querying of information and $RO5$ regarding validation of information for GDPR compliance.
The chapter ends with conclusions drawn from the research in terms of re-usability and interoperability are presented in \autoref{sec:sparql:conclusion}.

\section{Querying Information for GDPR Compliance using SPARQL}\label{sec:testing:sparql}
This section presents the creation and utilisation of SPARQL queries to retrieve information relevant for GDPR compliance.
The creation of the queries is dependant on the ontological representation of information being retrieved, which in this case includes the use of GDPRov and GDPRtEXT ontologies. This is further explained in \autoref{sec:testing:sparql:relation}
A GDPR preparation guide published by the Irish Data Protection Commission was used as the source of compliance questions for which corresponding the SPARQL queries were created. The methodology used for this is presented in \autoref{sec:testing:sparql:methodology} with an demonstration of the developed queries presented in \autoref{sec:testing:sparql:demo}.
A note on the evaluation of this work is presented in \autoref{sec:testing:sparql:evaluation}.

\subsection{SPARQL queries and ontological representation of information}\label{sec:testing:sparql:relation}
The research regarding querying presented here is based on the task of retrieving information for answering questions relevant to the assessment of compliance.
It represents utilisation of technical solutions to automate the task of information retrieval and requires machine-readable data (or metadata).
For this, the developed vocabularies provide the ontological concepts and relationships necessary to express the information using GDPR terminology and enable the association of information with clauses and concepts of GDPR.

The compliance questions, as presented in \autoref{sec:info:compliance-questions}, do not use a specific ontology or vocabulary but instead are based in the natural language and use legal terminology. In order to utilise technological solutions for answering them, it is necessary to first convert these questions into queries which can be executed over the compliance representation.
Since the work presented in this thesis derives motivation from the use of semantic web technologies which use RDF for representing information, the task of querying utilises the SPARQL standard to retrieve this information.

The ontology used in the SPARQL queries must match the ontology used in the representation of information associated with compliance. Differences in ontologies would hamper the effective of execution of queries and have the potential to return invalid results or empty results.
The creation of SPARQL queries is therefore specific to the ontologies used in a particular information representation, and is therefore also specific to the use-case it is used in.

\subsection{Methodology}\label{sec:testing:sparql:methodology}
The methodology used for creation of SPARQL queries is based on utilisation of GDPRov to represent the concepts and GDPRtEXT to link the information to the GDPR.
The SPARQL queries thus created aim to retrieve information relevant to answering the question rather than present an evaluation or assessment of compliance.
While the compliance questions presented in \autoref{sec:info:compliance-questions} provide a basis for construction of semantic queries using SPARQL, we focused on identifying a real-world use-case of the use of such compliance questions to provide a demonstration of the application of this research.

\subsubsection{Utilising compliance questions from GDPR readiness guide published by DPC}
We utilised the guide titled ``Preparing Your Organisation for the GDPR – A Guide for SMEs'' published by Data Protection Commission of Ireland (DPC) as the basis for compliance questions which were represented using SPARQL as semantic queries.
The guide was published by the DPC in 2017 to help organisations in assessing their readiness towards GDPR compliance requirements.
It is accessible online\footnote{\url{http://gdprandyou.ie/wp-content/uploads/2017/12/A-Guide-to-help-SMEs-Prepare-for-the-GDPR.pdf}} and consists of a `table' (see Fig. \ref{fig:sparql:guide}) containing questions regarding information about an organisations processing activities.
The guide divides the questions into contextual sections based on addressing specific GDPR articles and obligations.
\begin{figure}[htbp]
\centering
\fbox{\includegraphics[width=\textwidth]{img/GDPR_guide_page_10.png}}
\caption{Questions for information required to assess compliance - Page 10 of ``Preparing Your Organisation for the GDPR - A Guide for SMEs'' published by Ireland's Data Protection Commission}
\label{fig:sparql:guide}
\end{figure}

\subsubsection{Steps of the methodology}
The steps followed in utilising the questions in the guide to create SPARQL queries and demonstrate their application were as follows:
\begin{enumerate}
    \item Analyse questions within the document to identify corresponding concepts and relationships in GDPRov and GDPRtEXT (see below)
    \item Represent questions as SPARQL queries using GDPRov and GDPRtEXT (see below)
    \item Create a synthetic use-case based on processing of personal data with GDPRov and GDPRtEXT used to represent information (see \autoref{sec:testing:sparql:demo})
    \item Execute SPARQL queries over use-case to retrieve answers for compliance questions (see \autoref{sec:testing:sparql:demo})
    \item Evaluate the queries based on the subjective criteria of - a) Extent of answering compliance questions b) Suitability of retrieved results in answering compliance questions (see \autoref{sec:testing:sparql:evaluation})
\end{enumerate}

\subsubsection{Analysis of GDPR Readiness Guide}
The guide contains 63 questions across 13 pages that are presented in 9 sections.
Its analysis consisted of categorising the questions based on requirements of information, relation to phases of compliance, and whether they were suitable to be implemented as SPARQL queries.
The analysis was recorded and published online\footnote{\url{https://w3id.org/GDPRep/checklist-demo/notes}} in the form of a spreadsheet with comments describing the thought process in interpreting each question's information requirements.

The first set of questions on page 1 concern consent and personal data and are structurally different than the other set of questions in that they are more abstract and generic and concern the overall practices concerning processing of personal data by the organisation.
These questions are described under the `general' category with other groups of questions having their category mentioned explicitly within the document.
Questions in the general category require information and practices associated with consent and personal data. Other categories contain questions which enquire explicitly about activities and mechanisms regarding compliance to specific obligations.

The questions were analysed and categorised into three categories based on their intended requirements towards information required for compliance. The three categories identified through this exercise were - demonstrative, evaluative, and assistive - based on the requirements of information associated with them. 
Demonstrative questions require answers that satisfy the compliance question and do not need further actions or processing based on the information. 
Assistive questions provide information that can be directly evaluated for compliance, with the term `assistive' indicating information that assists in the evaluation.
Evaluative questions retrieve information whose evaluation requires further information retrieved through additional questions based on the provided information.
The primary difference assistive and evaluative questions is whether they retrieve information which can be evaluated as is for compliance or whether it requires additional questions to retrieve further information.
These terms used for categorisation are arbitrary and do not relate to any specific methodology for legal compliance, but are useful to analyse the question from an information management perspective.

Questions were also analysed based on whether they relate to or require information regarding activities in ex-ante and ex-post phases.
The questions do not explicitly provide an indication of whether they enquire about a model of processing (ex-ante) or the execution (ex-post). The distinction was based on whether the question concerned information about practices, plans, or intentions regarding processing of personal data - in which case it was deemed to enquire about ex-ante information.
Similarly, if a question concerned past execution of activities or records of activities - it was said to enquire about ex-post information.
In some cases, questions were specified to enquire about both ex-ante and ex-post information based on potential application in both phases.

An overview of the questions is provided in \autoref{table:sparql:dpc-1}.
It assigns an ID for each question to enable associating it with the corresponding SPARQL query and for linking related questions in the analysis.
The column `\textit{Category}' reflects the category of question mentioned within the guide, with general used for the initial generic questions.
`\textit{Title}' refers to the title of the text within the guide, and the column `\textit{GDPR}' refers to an explicit mention of a GDPR clause within the question or its description.
\begin{table}[htbp]
\footnotesize
\centering
\caption{Questions provided in the GDPR Readiness Guide}
\begin{tabularx}{\textwidth}{|l|l|X|l|}
\hline
ID & Category & Title & GDPR \\ \hline
G1 & General & Categories of personal data and data subjects &  \\ \hline
G2 & General & Elements of personal data included within each data category &  \\ \hline
G3 & General & Source of the personal data &  \\ \hline
G4 & General & Purposes for which personal data is processed &  \\ \hline
G5 & General & Legal basis for each processing purpose (non-special categories of personal data) &  \\ \hline
G6 & General & Special categories of personal data &  \\ \hline
G7 & General & Legal basis for processing special categories of personal data &  \\ \hline
G8 & General & Retention period &  \\ \hline
G9 & General & Action required to be GDPR compliant? &  \\ \hline
P1 & PersonalData & Validity of Consent & 7,8,9 \\ \hline
P2 & PersonalData & Retrospective Consent & 7,8,9 \\ \hline
P3 & PersonalData & Demonstration of Consent & 7,8,9 \\ \hline
P4 & PersonalData & Withdraw consent for processing & 7.8.9 \\ \hline
P5 & PersonalData & Children's Personal Data & 8 \\ \hline
P6 & PersonalData & Legitimate interest based data processing &  \\ \hline
R1 & Rights & Subject Access Requests (SARs) & 15 \\ \hline
R2 & Rights & Subject Access Requests (SARs) Response Time & 15 \\ \hline
R3 & Rights & Data Portability & 20 \\ \hline
R4 & Rights & Deletion and Rectification & 16,17 \\ \hline
R5 & Rights & Right to restriction of processing & 18 \\ \hline
R6 & Rights & Right to object to processing & 21 \\ \hline
R7 & Rights & Halt processing after right to object & 21 \\ \hline
R8 & Rights & Profiling and automated processing & 22 \\ \hline
R9 & Rights & Right to obtain human intervention & 22 \\ \hline
R10 & Rights & Restrictions to data subject rights & 23 \\ \hline
A1 & AccuracyRetention & Purpose Limitation &  \\ \hline
A2 & AccuracyRetention & Data minimisation &  \\ \hline
A3 & AccuracyRetention & Accuracy &  \\ \hline
A4 & AccuracyRetention & Retention &  \\ \hline
A5 & AccuracyRetention & Retention Legal Obligations &  \\ \hline
A6 & AccuracyRetention & Destroy data securely &  \\ \hline
A7 & AccuracyRetention & Duplication of records &  \\ \hline
T1 & Transparency & Transparency to customers and employees & 12,13,14 \\ \hline
T2 & Transparency & Provide Information listed in Article 13 & 13 \\ \hline
T3 & Transparency & Provide Information listed in Article 14 & 14 \\ \hline
T4 & Transparency & Provide information when engaging &  \\ \hline
T5 & Transparency & Provide information on facilitating rights &  \\ \hline
C1 & ControllerObligations & Supplier Agreements & 27,28,29 \\ \hline
C2 & ControllerObligations & Data Protection Officers & 37,38,39 \\ \hline
C3 & ControllerObligations & Reasons for not having a DPO & 37,38,39 \\ \hline
C4 & ControllerObligations & Escalation procedures & 37,38,39 \\ \hline
C5 & ControllerObligations & Escalation procedures through a DPO & 37,38,39 \\ \hline
C6 & ControllerObligations & Data Protection Impact Assessments (DPIAs) & 35 \\ \hline
S1 & DataSecurity & Risks involved in processing data & 32 \\ \hline
S2 & DataSecurity & Documented Security Program & 32 \\ \hline
S3 & DataSecurity & Resolving security related issues & 32 \\ \hline
S4 & DataSecurity & Designated individual for security & 32 \\ \hline
S5 & DataSecurity & Encryption & 32 \\ \hline
S6 & DataSecurity & Removing information & 32 \\ \hline
S7 & DataSecurity & Restoring access & 32 \\ \hline
B1 & DataBreach & Documented incident plans & 33,34 \\ \hline
B2 & DataBreach & Regular reviews & 33,34 \\ \hline
B3 & DataBreach & Notifying authorities & 33,34 \\ \hline
B4 & DataBreach & Notifying data subjects & 33,34 \\ \hline
B5 & DataBreach & Documentation of data breaches & 33,34 \\ \hline
B6 & DataBreach & Co-operation procedures for data breach & 33,34 \\ \hline
I1 & InternationalDataTransfer & Data transfer outside EEA & 44,45,46,47,48,49,50 \\ \hline
I2 & InternationalDataTransfer & Special category of Personal Data in Transfer & 44,45,46,47,48,49,50 \\ \hline
I3 & InternationalDataTransfer & Purpose of Transfer & 44,45,46,47,48,49,50 \\ \hline
I4 & InternationalDataTransfer & Transfer Recipients & 44,45,46,47,48,49,50 \\ \hline
I5 & InternationalDataTransfer & Transfer Details & 44,45,46,47,48,49,50 \\ \hline
I6 & InternationalDataTransfer & Legality of international transfers &  \\ \hline
I7 & InternationalDataTransfer & Transparency &  \\ \hline

\end{tabularx}
\label{table:sparql:dpc-1}
\end{table}

\autoref{table:sparql:dpc-2} presents a summarised view of the analysis of compliance questions presented in \autoref{table:sparql:dpc-1}.
The complete information along with additional comments and fields is available in the online published version of the analysis.
In the table, column `\textit{Type}' indicate the type of query based on categorisation as demonstrative, assistive, evaluative based on the description in the sections above.
The column `\textit{Data}' provides information on information required for the question, including results of other queries.
The relation of question to the ex-ante phase of compliance is reflected by the column `\textit{E/A}' and ex-post phase by the column '\textit{E/P}' using boolean \texttt{Y/N} values.
The column `\textit{SPARQL}' indicates whether a SPARQL query was constructed for the corresponding question.
The column `\textit{GDPRov}' indicates whether the current iteration of GDPRov provides the necessary concepts and relationships to answer the compliance question, with a value of \texttt{Y} indicating that it does, \texttt{N} indicating it does not, and \texttt{S} indicating the information to be out of scope for GDPRov.
\begin{table}[htbp]
\footnotesize
\centering
\caption{Analysis of compliance questions specified in \autoref{table:sparql:dpc-1}}
\begin{tabularx}{\textwidth}{|l|l|X|l|l|l|l|}
\hline
ID & Type & Data & E/A & E/P & SPARQL & GDPRov \\ \hline
G1 & Demonstrative & personal data, data subjects & Y & N & Y & Y \\ \hline
G2 & Demonstrative & personal data & Y & N & Y & Y \\ \hline
G3 & Demonstrative & personal data, steps that collect data, entities that provde data & Y & Y & Y & Y \\ \hline
G4 & Demonstrative & results of G1, processes acting on data & Y & N & Y & Y \\ \hline
G5 & Demonstrative & results of G4, processes acting on data & Y & N & Y & Y \\ \hline
G6 & Demonstrative & special category personal data & Y & N & Y & Y \\ \hline
G7 & Demonstrative & results of G6, steps that collect data, steps that store data & Y & N & Y & Y \\ \hline
G8 & Not-Implemented & results of G1, steps that store data &  &  & N & N \\ \hline
G9 & Not-Implemented &  &  &  & N & S \\ \hline
P1 & Assistive & consent, steps that acquire consent & Y & N & Y & Y \\ \hline
P2 & Not-Implemented &  &  &  & N & S \\ \hline
P3 & Evaluative & consent & Y & Y & Y & Y \\ \hline
P4 & Evaluative & steps that withdraw consent & Y & N & Y & Y \\ \hline
P5 & Evaluative & steps that acquire consent, steps for age verification & Y & N & Y & Y \\ \hline
P6 & Assistive & steps that process personal data & Y & N & Y & Y \\ \hline
R1 & Assistive & steps that handle SAR & Y & N & Y & Y \\ \hline
R10 & Not-Implemented &  &  &  & N & S \\ \hline
R2 & Assistive & steps that handle SAR & N & Y & N & Y \\ \hline
R3 & Evaluative & steps that address right to data portability & Y & N & Y & Y \\ \hline
R4 & Evaluative & steps that address right to rectification & Y & N & Y & Y \\ \hline
R5 & Assistive & data subjet request, steps that process personal data & N & Y & N & Y \\ \hline
R6 & Not-Implemented &  &  &  & N & Y \\ \hline
R7 & Evaluative & steps that process personal data & Y & N & Y & Y \\ \hline
R8 & Assistive & steps that make automated decisions, consent & Y & Y & Y & Y \\ \hline
R9 & Assistive & steps that make automated decisions, right to contest automated decisions & Y & N & Y & Y \\ \hline
A1 & Evaluative & personal data, consent, steps that involve personal data through use, share, store & Y & Y & Y & Y \\ \hline
A2 & Assistive & personal data, steps that process personal data & Y & Y & Y & Y \\ \hline
A3 & Not-Implemented &  &  &  & N & S \\ \hline
A4 & Not-Implemented &  &  &  & N & S \\ \hline
A5 & Not-Implemented &  &  &  & N & S \\ \hline
A6 & Assistive & steps that delete data & Y & N & Y & Y \\ \hline
A7 & Not-Implemented &  &  &  & N & S \\ \hline
T1 & Not-Implemented &  &  &  & N & S \\ \hline
T2 & Assistive & steps that collect personal data & Y & N & Y & Y \\ \hline
T3 & Assistive & steps that collect personal data & Y & N & Y & Y \\ \hline
T4 & Not-Implemented &  &  &  & N & S \\ \hline
T5 & Not-Implemented &  &  &  & N & S \\ \hline
C1 & Not-Implemented &  &  &  & N & S \\ \hline
C2 & Not-Implemented &  &  &  & N & Y \\ \hline
C3 & Not-Implemented &  &  &  & N & S \\ \hline
C4 & Not-Implemented &  &  &  & N & S \\ \hline
C5 & Not-Implemented &  &  &  & N & S \\ \hline
C6 & Assistive & steps part of the DPIA process & Y & N & Y & Y \\ \hline
S1 & Assistive & steps that process data & Y & N & Y & Y \\ \hline
S2 & Not-Implemented &  &  &  & N & S \\ \hline
S3 & Not-Implemented &  &  &  & N & S \\ \hline
S4 & Not-Implemented &  &  &  & N & S \\ \hline
S5 & Not-Implemented & steps that share data &  &  & N & N \\ \hline
S6 & Not-Implemented &  &  &  & N & Y \\ \hline
S7 & Not-Implemented &  &  &  & N & N \\ \hline
B1 & Evaluative & processes or plan that address security incidents & Y & N & Y & Y \\ \hline
B2 & Not-Implemented &  &  &  & N & S \\ \hline
B3 & Evaluative & processes or plans for notifying DPC & Y &  & Y & Y \\ \hline
B4 & Evaluative & processes or plans for notifying data subjects of a data breach &  &  & Y & Y \\ \hline
B5 & Not-Implemented &  &  &  & N & Y \\ \hline
B6 & Not-Implemented &  &  &  & N & S \\ \hline
I1 & Evaluative & steps that share data & Y & Y & Y & Y \\ \hline
I2 & Evaluative & results from I1, category of personal data & Y & N & Y & Y \\ \hline
I3 & Assistive & steps that share data & Y & Y & Y & Y \\ \hline
I4 & Evaluative & steps that share data & Y & Y & Y & Y \\ \hline
I5 & Not-Implemented &  &  &  & N & Y \\ \hline
I6 & Not-Implemented &  &  &  & N & Y \\ \hline
I7 & Not-Implemented & steps that share data &  &  & N & S \\ \hline
\end{tabularx}
\label{table:sparql:dpc-2}
\end{table}

% Where the questions could not be answered due to missing concepts and relationships in GDPRov, or due to uncertain interpretations of a legal concept or ambiguity, a note was made to identify solutions in the future.
% This was used to update the GDPRov at a later date with additional concepts, such as for legal bases, data sharing, data transfers, or documentation of data breaches.
The information regarding GDPRov is also provided since the creation of SPARQL queries from GDPR readiness guide was carried out in the earlier stages of GDPRov's iterations and before the enforcement of the GDPR in May 2018. Therefore, some questions were deemed to be ambiguous or lacking legal information on information necessary for compliance.
The queries and constraints presented in \autoref{sec:testing:shacl} were developed at a later stage when GDPR had seen significant attention and interpretation and do not suffer from similar absences of implementation.

\subsubsection{Creation of SPARQL queries}
The creation of SPARQL queries involved analysis of the text of a question to identify relevant concepts and relationships in GDPRov useful towards expressing the question as a semantic query in SPARQL as well as representing the information required to answer the question.
In this, some questions were found to be subjective or qualitative and thus could not be expressed as SPARQL queries. These are indicated as not implemented in \autoref{table:sparql:dpc-2} and were also included in the webpage of implementation presented in \autoref{sec:testing:sparql:demo}.

A total of 33 SPARQL queries were created based on the above analysis of compliance questions and their requirements.
The queries utilised GDPRov and GDPRtEXT ontologies to specify the information associated with the question.
The SPARQL queries were published online\footnote{\url{https://w3id.org/GDPRep/checklist-demo/sparql-queries}}
with a seperate files for each query associated with a compliance question, and a common file containing the common prefixes used in all queries.

As an example, Listing \autoref{code:sparql:dpc-G5} contains the corresponding SPARQL query for compliance question \texttt{G5} which concerns the legal basis used to justify processing of personal data. 
The query retrieves information about steps and the processes along with the legal basis for their operation in ex-ante phase using the vocabulary provided by GDPRov.
Within this, the query specifically retrieves steps which are defined as being part of a process and use some form of personal data. 
The legal bases can be associated with individual steps or with the process they are associated with.
\begin{listing}[ht]
\begin{minted}[
    frame=single,
    framesep=5mm,
    baselinestretch=1,
    linenos
]{text}
PREFIX rdfs:     <http://www.w3.org/2000/01/rdf-schema#>
PREFIX gdprov:   <http://purl.org/adaptcentre/openscience/ontologies/gdprov#>
PREFIX gdprtext: <http://purl.org/adaptcentre/openscience/ontologies/GDPRtEXT#>

SELECT DISTINCT ?process ?legal WHERE {
  ?data a ?data_type .
  ?data_type rdfs:subClassOf gdprov:PersonalData .
  ?step a ?step_type .
  ?step_type rdfs:subClassOf gdprov:DataStep .
  ?step gdprov:usesData ?data . 
  ?step gdprov:isPartOfProcess ?process .
  OPTIONAL { ?step gdprov:hasLegalBasis ?legal } .
  OPTIONAL { ?process gdprov:hasLegalBasis ?legal } .
} ORDER BY ?process
\end{minted}
\caption{SPARQL query representing compliance question \texttt{G5} concerning legal basis for processing}
\label{code:sparql:dpc-G5}
\end{listing}

\subsection{Demonstration using synthetic use-case} \label{sec:testing:sparql:demo}
To demonstrate the application of queries, a synthetic use-case was created using GDPRov and GDPRtEXT to represent information.
The use-case is based on the scenario of an online shopping service that allows users to order products.
RDF representations of the processes and personal data associated with the use-case were created and queried using the created SPARQL queries to retrieve information regarding compliance.
The implementation was provided online\footnote{\url{https://w3id.org/GDPRep/checklist-demo}} along with its data and code in a public repository\footnote{\url{http://openscience.adaptcentre.ie/GDPR-checklist-demo/demo/}}.

\subsubsection{Use-case: Online shopping service that shows ads}
Within the use-case, users can shop for products using the online service i.e. a website. Users have the option to establish an account with the service to receive discounts and special offers for the products offered on the service.
Ads are served to users and are generated by a Third Party.
The sign-up process collects personal data such as name, address, email, and contact number.
While ordering products, users are requested to provide sensitive information for transactions about their bank account or credit cards.

The personal data is represented by extending \texttt{gdprov:PersonalData} as \texttt{CustomerInfo} for representing information about the user, and extending \texttt{SensitiveData} for representing banking and financial information as \texttt{gdprov:BankingInfo}.
Processes for handling obligations and rights are expressed using the terms provided by GDPRov.
The sign-up process enables an user to provide information which is used for personalisation and ads and collects the user's consent.
As a final step, the Fact++\footnote{\url{http://owl.cs.manchester.ac.uk/tools/fact/}}
reasoner was used to derive additional facts about the use-case and to ensure its consistency in terms of correct use of GDPRov (and by extension PROV-O and P-Plan) and GDPRtEXT.

\subsubsection{Implementation}
The online demo provides an execution of the created SPARQL queries over the data defined for the use-case.
This represents automation of answering compliance questions using retrieved information.
The demo intended to showcase how the static GDPR readiness checklist can be made more interactive and automated using semantic web technologies.

The demo is provided as a single web page, with the questions from GDPR readiness checklist provided in their natural language form and ID followed by the corresponding SPARQL query.
The results for each query are retrieved whenever the page loads from a
SPARQL endpoint\footnote{\url{http://openscience.adaptcentre.ie/sparql}}
containing the RDF data about the use-case.
The demo uses YASQE\footnote{\url{http://yasqe.yasgui.org/}} to represent the SPARQL queries with syntax highlighting, and YASR\footnote{\url{http://yasr.yasgui.org/}} to represent the results of queries in an interactive fashion.

The results of each query contain the information associated with answering the relevant compliance questions. For the SPARQL query regarding question \texttt{G5} presented in Listing. \autoref{code:sparql:dpc-G5} which enquired about legal obligations, the results express the steps and processes along with their legal obligations.
The query and the results as presented in the demo are depicted in \autoref{fig:sparql:demo}.
The results consist of five rows - of which three are processes that handle the various rights and therefore are not accompanied with any legal basis\footnote{The processes handling rights should utilise the legal basis of requirements specified by law since GDPR mandates the provision of rights.}.
The remaining two results represent processes associated with the provision of the service, of which \textit{OrderProcess} reprents `ordering a product' and uses legitimate interest as its legal basis, and \textit{NewUserSignUpProcess} collects information about an user and uses the legal basis of given consent.
\begin{figure}[htbp]
\centering
\fbox{\includegraphics[width=0.75\textwidth]{img/sparql_query_demo.png}}
\caption{Retrieving information using SPARQL for query G5 in GDPR readiness checklist}
\label{fig:sparql:demo}
\end{figure}

\subsection{Evaluation}\label{sec:testing:sparql:evaluation}
The aim of this work was to represent compliance questions using SPARQL in order to retrieve information represented in RDF regarding the processing of personal data.
The demonstration using a synthetic use-case provided the basis for exploring the application of created SPARQL queries by using GDPRov and GDPRtEXT were utilised as the ontological representations of data.
The evaluation of this work, while not being exhaustive, concerns the creation of SPARQL queries and their application over a given use-case.

In terms of coverage of compliance questions represented as SPARQL queries, the exercise could not represent all of the questions within the GDPR readiness guide. The reasons were these lack of knowledge (at the time of implementation) regarding interpretation of GDPR compliance requirements, absence of concepts in GDPRov, and the context of the query being out of scope for the research question of this thesis - namely activities associated with personal data and consent.
Of the total 63 questions within the GDPR readiness guide, 32 questions have corresponding SPARQL queries created and used in the demo.

Of the 31 questions that were not implemented, 20 questions were considered out of scope, with the other 3 questions lacking corresponding concepts in GDPRov to create SPARQL queries.
Of these, question \texttt{G8} concerning retention periods for personal data can be expressed using the Time ontology \cite{cox_time_2017}.
The other two, \texttt{S5} and \texttt{S7} require specification of information associated with information management and governance procedures utilised within an organisation. While these are technically not outside the scope of GDPRov, they require a larger understanding of how such processes are specified and managed, and commonly involve use of ISO/IEC specifications to denote practices.
Others, such as question \texttt{C1} concerning agreements between entities or question \texttt{C4} concerning escalation procedures involving DPO can be represented using the concepts provided by GDPRov, especially the representation of processes involving steps, but are not considered within the scope of the ontology.

An evaluation of the results retrieved by the queries requires a legal expert who can interpret the extent to which the results are useful in the exercise of deciding or interpreting compliance.
In addition to these, the requirements of legal compliance differ from that of organisational management of processes, and therefore also require a domain expert - regarding management of business processes - to analyse the results from the perspective of an organisation.
Since neither of them could be availed of, the exercise of evaluating the queries within a real-world use case of an organisation and the process of evaluation of compliance was not carried out.

% ---------------------------------------------------------------------------------

\section{Validating Information for GDPR Compliance using SHACL}\label{sec:testing:shacl}
This section presents the application of SHACL to validate information based on requirements of GDPR compliance.
SHACL is utilised as a validation mechanism to create a test-driven approach where information regarding processing activities is represented using semantic representations, and is first checked for correctness and then for compliance.
In this, the constraints presented in \autoref{sec:info:constraints} are utilised to determine the correctness and compliance of information, and questions presented in \autoref{sec:info:compliance-questions} are used to retrieve information for compliance.
Both the constraints and the queries are linked to the GDPR using GDPRtEXT.

The results of SHACL validations are persisted to create a `compliance graph' which enables querying for information regarding compliance, and provides more efficient testing based on reuse of ex-ante tests in ex-post testing.
The approach is demonstrated using a proof-of-concept implementation based on the evaluation of consent information on a real-world website and using GDPRov, GConsent, and GDPRtEXT to represent the information.
This approach and the implementation have been published in peer-reviewed publications concerning the conceptual model \cite{pandit_exploring_2018}, construction of a knowledge graph from information about GDPR compliance \cite{pandit_towards_2018}, and implementation testing compliance of given consent on a real-world website \cite{pandit_test-driven_2019}.
All resources regarding this work have been published online\footnote{\url{https://w3id.org/GDPRep/semantic-tests}} under an open and permissive license (CC-by-4.0).

The work presented in this section fulfils the research objectives $RO5$ and demonstrates the following:
\begin{enumerate}
    \item Utilise SHACL to validate information for GDPR compliance.
    \item Express compliance as a test-driven exercise similar to the concept of unit-testing in software engineering.
    \item Utilise results of testing ex-ante information for testing of ex-post information in order to reduce the number of tests required.
    \item Construct a compliance graph storing test results based on the concept of knowledge-graph.
    \item Demonstrate the use of compliance graph in retrieving and documenting information regarding GDPR compliance.
\end{enumerate}

A description of the approach is provided in \autoref{sec:testing:shacl:approach} which presents the role of SHACL as a validation mechanism and the creation of a compliance graph from the validation test results. The creation of SHACL constraints to represent the natural language constraints in \autoref{sec:info:constraints} is presented in \autoref{sec:testing:shacl:constraints}.
The utilisation of ex-ante validations for evaluation of ex-post information is presented in \autoref{sec:testing:shacl:combine}.
The proof-of-concept implementation demonstrating application of the approach is presented in \autoref{sec:testing:shacl:demo} with a discussion on it presented in \autoref{sec:testing:shacl:eval}.

\subsection{Validation Model}\label{sec:testing:shacl:approach}
\autoref{fig:shacl:model} presents a visual overview of the approach for the validation model.
In this ....
% what is the data graph
% role of compliance queries
% validate using SHACL
% create compliance graph
% validate obligations and articles by checking the compliance graph, insert results back in to the compliance graph
% generate report using information from compliance graph to answer compliance queries and get an overview of compliance
\begin{figure}[htbp]
    \centering
    \fbox{\includegraphics[width=\textwidth]{img/SHACL-model.png}}
    \caption{Overview of approach utilising SHACL to validate information for GDPR compliance}
    \label{fig:shacl:model}
\end{figure}

\subsection{Creation of SHACL constraints}\label{sec:testing:shacl:constraints}

\subsection{Utilising ex-ante test results for ex-post validations}\label{sec:testing:shacl:combine}

\subsection{Proof-of-concept implementation}\label{sec:testing:shacl:demo}

\subsection{Discussion}\label{sec:testing:shacl:eval}

\section{Conclusion}\label{sec:sparql:conclusion}


\subsubsection{Summary of work presented}
% one line covering both, same as intro
\autoref{sec:testing:sparql} presented the use of SPARQL queries for representing compliance queries and retrieving information associated with compliance that was represented using GDPRov and GDPRtEXT ontologies.
The queries were based on questions from GDPR readiness guide published by the Data Protection Commission of Ireland in 2017 to assist organisations in assessing their adherence to compliance requirements of the GDPR.
The methodology regarding the analysis of the questions from the document and representation using GDPRov was presented in \autoref{sec:testing:sparql:methodology}.
An application of SPARQL queries was presented through an online demo using a synthetic use-case based on the scenario of an online shopping service.
The queries and the demo were published in a peer-reviewed publication \cite{pandit_queryable_2018} at the SEMANTiCS conference.

\subsubsection{Re-usability of developed resources}
The interpretation of compliance questions in the GDPR readiness document using SPARQL and its application in the synthetic use-case demonstrates the potential application and usefulness of SPARQL queries to retrieve information relevant for compliance.
However, the creation of SPARQL is highly dependant on utilising the same ontological concepts as the data it is querying. Therefore, such SPARQL queries are use-case dependant, and by definition do not have re-usability beyond the data they were created for.

\subsubsection{Towards standardisation and interoperable resources}
To counter this, it is essential to have a standardised interoperable representation which can be utilised to either 