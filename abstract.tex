\chapter*{Abstract}
The General Data Protection Regulation (GDPR) is a landmark law regarding privacy and data protection.
GDPR stipulates potentially large fines if an organisation is found to not be compliant.
This has resulted in research involving use of technological resources to meet and evaluate compliance requirements.
Such approaches involve representing information about processing activities in a machine-readable format, verifying its correctness, and evaluating whether it meets the obligations of GDPR.

Under GDPR, an organisation is required to maintain and demonstrate documentation showing its compliance to the obligations and requirements regarding processing activities.
This documentation involves information on how the activities were planned, evaluated for compliance, and executed.
In addition to these, if consent is used as a legal basis to justify the processing, then information about how that consent was obtained also needs to be recorded in order to evaluate and demonstrate its adherence to requirements specified by GDPR.

Utilising semantic web technologies provides a machine-readable and interoperable representation of information that can be queried and verified based on open standards such as RDF, OWL, SPARQL, and SHACL.
This thesis presents the use of semantic web technologies to represent and associate information regarding processing of personal data and consent with GDPR for assistance with its compliance.
In particular, it addresses three deficits within the current state of the art for utilising linked data approaches for GDPR compliance.
The first of these is regarding associating information with the text and concepts of GDPR which would enable the adoption of a linked data approach to automation and management of compliance documentation.
The second concerns representations of activities regarding the planning and execution of processes concerning personal data and consent.
The third involves representing information required to evaluate and demonstrate compliance with the requirements of consent.

The outcomes of the work are presented in the thesis in the form of major contributions of GDPRtEXT - a linked data representation of the text of GDPR and a glossary of concepts relevant for its compliance, GDPRov - an OWL2 ontology based on PROV-O for modelling activities associated with personal data and consent in ex-ante (planning) and ex-post (execution) phases, and GConsent - an OWL2 ontology for representing information regarding consent.
The thesis also presents minor contributions describing application of semantic web technologies in the form of querying and validation of information using the SPARQL and SHACL standards. 